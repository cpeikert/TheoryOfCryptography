\documentclass[11pt]{article}

\usepackage{fullpage}
\usepackage{newtxtext}
\usepackage[T1]{fontenc}
\usepackage{hyperref,microtype,pdfsync}
\usepackage{amssymb}
\usepackage{algorithm,algorithmic}

\RequirePackage{mathtools}

%%% BLACKBOARD SYMBOLS

% hyperref package defines C and G, undefined to avoid conflicts
\let\C\relax
\let\G\relax
\newcommand{\C}{\ensuremath{\mathbb{C}}}
\newcommand{\D}{\ensuremath{\mathbb{D}}}
\newcommand{\F}{\ensuremath{\mathbb{F}}}
\newcommand{\G}{\ensuremath{\mathbb{G}}}
\newcommand{\J}{\ensuremath{\mathbb{J}}}
\newcommand{\N}{\ensuremath{\mathbb{N}}}
\newcommand{\Q}{\ensuremath{\mathbb{Q}}}
\newcommand{\R}{\ensuremath{\mathbb{R}}}
\newcommand{\T}{\ensuremath{\mathbb{T}}}
\newcommand{\Z}{\ensuremath{\mathbb{Z}}}
\newcommand{\QR}{\ensuremath{\mathbb{QR}}}

\newcommand{\Zt}{\ensuremath{\Z_t}}
\newcommand{\Zp}{\ensuremath{\Z_p}}
\newcommand{\Zq}{\ensuremath{\Z_q}}
\newcommand{\ZN}{\ensuremath{\Z_N}}
\newcommand{\Zps}{\ensuremath{\Z_p^*}}
\newcommand{\ZNs}{\ensuremath{\Z_N^*}}
\newcommand{\JN}{\ensuremath{\J_N}}
\newcommand{\QRN}{\ensuremath{\QR_{N}}}
\newcommand{\QRp}{\ensuremath{\QR_{p}}}

%%% THEOREM COMMANDS

\RequirePackage{amsthm}

\theoremstyle{plain}            % following are "theorem" style
\newtheorem{theorem}{Theorem}[section]
\newtheorem{lemma}[theorem]{Lemma}
\newtheorem{corollary}[theorem]{Corollary}
\newtheorem{proposition}[theorem]{Proposition}
\newtheorem{claim}[theorem]{Claim}
\newtheorem{fact}[theorem]{Fact}

\theoremstyle{definition}       % following are def style
\newtheorem{definition}[theorem]{Definition}
\newtheorem{conjecture}[theorem]{Conjecture}
\newtheorem{example}[theorem]{Example}
\newtheorem{protocol}[theorem]{Protocol}

\theoremstyle{remark}           % following are remark style
\newtheorem{remark}[theorem]{Remark}
\newtheorem{note}[theorem]{Note}

% equation numbering style
\numberwithin{equation}{section}

%%% GENERAL COMPUTING

\newcommand{\bit}{\ensuremath{\set{0,1}}}
\newcommand{\pmone}{\ensuremath{\set{-1,1}}}

% asymptotics
\DeclareMathOperator{\poly}{poly}
\DeclareMathOperator{\polylog}{polylog}
\DeclareMathOperator{\negl}{negl}
\newcommand{\Otil}{\ensuremath{\tilde{O}}}

% probability/distribution stuff
\DeclareMathOperator*{\E}{E}
\DeclareMathOperator*{\Var}{Var}

% sets in calligraphic type
\newcommand{\calD}{\ensuremath{\mathcal{D}}}
\newcommand{\calF}{\ensuremath{\mathcal{F}}}
\newcommand{\calG}{\ensuremath{\mathcal{G}}}
\newcommand{\calH}{\ensuremath{\mathcal{H}}}
\newcommand{\calX}{\ensuremath{\mathcal{X}}}
\newcommand{\calY}{\ensuremath{\mathcal{Y}}}

% types of indistinguishability
\newcommand{\compind}{\ensuremath{\stackrel{c}{\approx}}}
\newcommand{\statind}{\ensuremath{\stackrel{s}{\approx}}}
\newcommand{\perfind}{\ensuremath{\equiv}}

% font for general-purpose algorithms
\newcommand{\algo}[1]{\ensuremath{\mathsf{#1}}}
% font for general-purpose computational problems
\newcommand{\problem}[1]{\ensuremath{\mathsf{#1}}}
% font for complexity classes
\newcommand{\class}[1]{\ensuremath{\mathsf{#1}}}

% complexity classes and languages
\renewcommand{\P}{\class{P}}
\newcommand{\BPP}{\class{BPP}}
\newcommand{\NP}{\class{NP}}
\newcommand{\coNP}{\class{coNP}}
\newcommand{\AM}{\class{AM}}
\newcommand{\coAM}{\class{coAM}}
\newcommand{\IP}{\class{IP}}

%%% "LEFT-RIGHT" PAIRS OF SYMBOLS


% inner product
\DeclarePairedDelimiter\inner{\langle}{\rangle}
% absolute value
\DeclarePairedDelimiter\abs{\lvert}{\rvert}
% a set
\DeclarePairedDelimiter\set{\{}{\}}
% parens
\DeclarePairedDelimiter\parens{(}{)}
% tuple, alias for parens
\DeclarePairedDelimiter\tuple{(}{)}
% square brackets
\DeclarePairedDelimiter\bracks{[}{]}
% rounding off
\DeclarePairedDelimiter\round{\lfloor}{\rceil}
% floor function
\DeclarePairedDelimiter\floor{\lfloor}{\rfloor}
% ceiling function
\DeclarePairedDelimiter\ceil{\lceil}{\rceil}
% length of some vector, element
\DeclarePairedDelimiter\length{\lVert}{\rVert}
% "lifting" of a residue class
\DeclarePairedDelimiter\lift{\llbracket}{\rrbracket}
\DeclarePairedDelimiter\len{\lvert}{\rvert}

%%% CRYPTO-RELATED NOTATION

% KEYS AND RELATED

\newcommand{\key}[1]{\ensuremath{#1}}

\newcommand{\pk}{\key{pk}}
\newcommand{\vk}{\key{vk}}
\newcommand{\sk}{\key{sk}}
\newcommand{\mpk}{\key{mpk}}
\newcommand{\msk}{\key{msk}}
\newcommand{\fk}{\key{fk}}
\newcommand{\id}{id}
\newcommand{\keyspace}{\ensuremath{\mathcal{K}}}
\newcommand{\msgspace}{\ensuremath{\mathcal{M}}}
\newcommand{\ctspace}{\ensuremath{\mathcal{C}}}
\newcommand{\tagspace}{\ensuremath{\mathcal{T}}}
\newcommand{\idspace}{\ensuremath{\mathcal{ID}}}

\newcommand{\concat}{\ensuremath{\|}}

% GAMES

% advantage
\newcommand{\advan}{\ensuremath{\mathbf{Adv}}}

% different attack models
\newcommand{\attack}[1]{\ensuremath{\text{#1}}}

\newcommand{\atk}{\attack{atk}} % dummy attack
\newcommand{\indcpa}{\attack{ind-cpa}}
\newcommand{\indcca}{\attack{ind-cca}}
\newcommand{\anocpa}{\attack{ano-cpa}} % anonymous
\newcommand{\anocca}{\attack{ano-cca}}
\newcommand{\euacma}{\attack{eu-acma}} % forgery: adaptive chosen-message
\newcommand{\euscma}{\attack{eu-scma}} % forgery: static chosen-message
\newcommand{\suacma}{\attack{su-acma}} % strongly unforgeable

% ADVERSARIES
\newcommand{\attacker}[1]{\ensuremath{\mathcal{#1}}}

\newcommand{\Adv}{\attacker{A}}
\newcommand{\AdvA}{\attacker{A}}
\newcommand{\AdvB}{\attacker{B}}
\newcommand{\Dist}{\attacker{D}}
\newcommand{\Sim}{\attacker{S}}
\newcommand{\Ora}{\attacker{O}}
\newcommand{\Inv}{\attacker{I}}
\newcommand{\For}{\attacker{F}}

% CRYPTO SCHEMES

\newcommand{\scheme}[1]{\ensuremath{\text{#1}}}

% pseudorandom stuff
\newcommand{\prg}{\algo{PRG}}
\newcommand{\prf}{\algo{PRF}}
\newcommand{\prp}{\algo{PRP}}

% symmetric-key cryptosystem
\newcommand{\skc}{\scheme{SKC}}
\newcommand{\skcgen}{\algo{Gen}}
\newcommand{\skcenc}{\algo{Enc}}
\newcommand{\skcdec}{\algo{Dec}}

% public-key cryptosystem
\newcommand{\pkc}{\scheme{PKC}}
\newcommand{\pkcgen}{\algo{Gen}}
\newcommand{\pkcenc}{\algo{Enc}} % can also use \kemenc and \kemdec
\newcommand{\pkcdec}{\algo{Dec}}

% digital signatures
\newcommand{\sig}{\scheme{SIG}}
\newcommand{\siggen}{\algo{Gen}}
\newcommand{\sigsign}{\algo{Sign}}
\newcommand{\sigver}{\algo{Ver}}

% message authentication code
\newcommand{\mac}{\scheme{MAC}}
\newcommand{\macgen}{\algo{Gen}}
\newcommand{\mactag}{\algo{Tag}}
\newcommand{\macver}{\algo{Ver}}

% key-encapsulation mechanism
\newcommand{\kem}{\scheme{KEM}}
\newcommand{\kemgen}{\algo{Gen}}
\newcommand{\kemenc}{\algo{Encaps}}
\newcommand{\kemdec}{\algo{Decaps}}

% identity-based encryption
\newcommand{\ibe}{\scheme{IBE}}
\newcommand{\ibesetup}{\algo{Setup}}
\newcommand{\ibeext}{\algo{Ext}}
\newcommand{\ibeenc}{\algo{Enc}}
\newcommand{\ibedec}{\algo{Dec}}

% hierarchical IBE (as key encapsulation)
\newcommand{\hibe}{\scheme{HIBE}}
\newcommand{\hibesetup}{\algo{Setup}}
\newcommand{\hibeext}{\algo{Extract}}
\newcommand{\hibeenc}{\algo{Encaps}}
\newcommand{\hibedec}{\algo{Decaps}}

% binary tree encryption (as key encapsulation)
\newcommand{\bte}{\scheme{BTE}}
\newcommand{\btesetup}{\algo{Setup}}
\newcommand{\bteext}{\algo{Extract}}
\newcommand{\bteenc}{\algo{Encaps}}
\newcommand{\btedec}{\algo{Decaps}}

% trapdoor functions
\newcommand{\tdf}{\scheme{TDF}}
\newcommand{\tdfgen}{\algo{Gen}}
\newcommand{\tdfeval}{\algo{Eval}}
\newcommand{\tdfinv}{\algo{Invert}}
\newcommand{\tdfver}{\algo{Ver}}

%%% PROTOCOLS

\newcommand{\out}{\text{out}}
\newcommand{\view}{\text{view}}

%%% COMMANDS FOR LECTURES/HOMEWORKS

\RequirePackage{fancyhdr}

\newcommand{\lecheader}{%
  \chead{\large \textbf{Lecture \lecturenum\\\lecturetopic}}

  \lhead{\small \textbf{Theory of Cryptography}\\}

  \rhead{\small \textbf{Instructor:
      \href{http://www.eecs.umich.edu/~cpeikert/}{Chris Peikert}\\Scribe:
      \scribename}}

  \setlength{\headheight}{20pt}
  \setlength{\headsep}{16pt}
}


%%% xsim settings

\RequirePackage{xsim}
\RequirePackage{needspace}

\DeclareExerciseEnvironmentTemplate{QRef}
{%
    \par\vspace{\baselineskip}
    \Needspace*{2\baselineskip}
    \noindent
    \hyperref[sol:\ExerciseID]{\textbf{\XSIMmixedcase{\GetExerciseName}~\GetExerciseProperty{counter}.}}
    \label{ques:\ExerciseID}
}{}

\DeclareExerciseEnvironmentTemplate{ARef}{%
        \Needspace*{2\baselineskip}
        \noindent
        \hyperref[ques:\ExerciseID]{\textbf{\XSIMmixedcase{\GetExerciseParameter{exercise-name}}~\GetExerciseProperty{counter}.}}
        \label{sol:\ExerciseID}
        \GetExerciseBody{exercise}
        \newline
        \textbf{\XSIMmixedcase{\GetExerciseName}.}
}{\par\vspace{\baselineskip}}

\DeclareExerciseHeadingTemplate{simple}{%
    \section*{\XSIMmixedcase{\GetExerciseParameter{solution-name}s}}
}

\DeclareExerciseType{question}{%
    exercise-env = question,
    solution-env = answer,
    exercise-name = Question,
    solution-name = Answer,
    exercise-template = QRef,
    solution-template = ARef,
}

\xsimsetup{%
  path = {questions},
  file-extension = {aux},
  print-solutions/headings-template=simple
}

\AtEndDocument{\pagebreak\printsolutions}


% VARIABLES

\newcommand{\lecturenum}{5}
\newcommand{\lecturetopic}{Indistinguishability, Pseudorandomness}
\newcommand{\scribename}{Abhinav Shantanam}

% END OF VARIABLES

\lecheader

\pagestyle{plain}               % default: no special header

\begin{document}

\thispagestyle{fancy} % first page should have special header

% LECTURE MATERIAL STARTS HERE

\section{Indistinguishability}
\label{sec:indistinguishability}

We'll now start a major unit on \emph{indistinguishability} and
\emph{pseudorandomness}.  These concepts are a cornerstone of modern
cryptography, underlying several foundational applications such as
pseudorandom generators, secure encryption, ``commitment'' schemes,
and much more.

For example, our most immediate application of indistinguishability
will be to construct cryptographically strong \emph{pseudorandom bit
  generators}.  These are algorithms that produce many
``random-looking'' bits, while using very little ``true'' randomness.
(In particular, the bits they output will necessarily be ``very far''
from truly random, in a statistical sense.)  One easy-to-imagine
application would be to use the pseudorandom bit stream as a one-time
encryption pad, which would allow the shared secret key to be much
smaller than the message.  But what does it \emph{mean} for a string
of bits to be ``random-looking''?  And how can we be confident that
using such bits does not introduce any unforeseen weaknesses in our
system?

More generally, the motivating question for our study is:
\begin{center}
  When can two (possibly different) objects be considered
  \emph{effectively the same}?
\end{center}
The answer:
\begin{center}
  \emph{When they can't be told apart!}
\end{center}
Though seemingly glib, this answer encapsulates a very powerful
mindset that will serve us well as we go forward.

\subsection{Statistical Indistinguishability}
\label{sec:stat-indist}

We use probability theory to model (in)distinguishability.  If two
distributions are identical, then they certainly should be considered
indistinguishable.  We relax this condition to define
\emph{statistical} indistinguishability, for when the
\emph{statistical distance} between the two distributions is
negligible.  The statistical distance between two distributions $X$
and $Y$ over a domain $\Omega$ is defined as\footnote{For a set~$S$ of
  real numbers, its \emph{supremum} $\sup(S)$ is the least upper bound
  of the elements of~$S$. The suprenum can be thought of as a
  generalization of the maximum element to (infinite) sets where no
  such element may exist. For example, there is no maximum element of
  the set $[0,1) \subseteq \R$, but the suprenum is 1 since it is the
  smallest real number that is larger than every element of
  $[0,1)$. When~$S$ is finite, the supremum is simply the maximum.}
\[ \Delta(X,Y) := \sup_{A \subseteq \Omega} \abs{X(A) - Y(A)}, \]
where $X(A) = \sum_{w \in A} \Pr[X=w]$ is the probability that a draw
from~$X$ lands in~$A$, and likewise for~$Y(A)$. We can
view~$A \subseteq \Omega$ as a statistical ``test'' that has some
probability of ``passing'' when given an element drawn from~$X$, or
from~$Y$; the statistical distance is essentially the maximum
difference between these two probabilities, taken over all tests.
Note that $A$ and $\bar{A}$ are effectively the same test, since
\[ \abs{X(\bar{A})-Y(\bar{A})} = \abs{1-X(A)-(1-Y(A))} =
  \abs{Y(A)-X(A)} = \abs{X(A)-Y(A)}. \]

\begin{lemma}
  \label{lem:stat-dist}
  For distributions $X,Y$ over a finite domain $\Omega$,
  \[ \Delta(X,Y) = \frac12 \sum_{w \in \Omega} \abs{X(w) - Y(w)}. \]
\end{lemma}

\begin{proof}
  Let the test $A = \set{w \in \Omega\; :\; X(w) > Y(w)}$.  This makes
  $X(A) - Y(A)$ as large as possible, so
  $\Delta(X,Y) = X(A)-Y(A) = \sum_{w \in A} \abs{X(w)-Y(w)}$.  As
  noted above, we also have
  $\Delta(X,Y) = Y(\bar{A})-X(\bar{A}) = \sum_{w \in \bar{A}}
  \abs{X(w)-Y(w)}$.  Summing the two equations, we have
  $2\Delta(X,Y) = \sum_{w \in \Omega} \abs{X(w)-Y(w)}$, as desired.
\end{proof}

Statistical distance is very robust, which enhances its usefulness.
Using Lemma~\ref{lem:stat-dist}, the following facts are
straightforward to prove.

\begin{lemma}
  Let $f$ be a function (or more generally, randomized procedure) on
  the domain of $X,Y$. Then $\Delta(f(X), f(Y)) \leq \Delta(X,Y)$.
\end{lemma}
In other words, statistical distance cannot be increased by the
application of any (randomized) procedure.

\begin{lemma}
  Statistical distance is a metric; in particular, it satisfies the
  following three properties:
  \begin{itemize}
  \item Identity of indiscernibles: $\Delta(X,Y) = 0 \iff X = Y$
  \item Symmetry: $\Delta(X,Y) = \Delta(Y,X)$
  \item Subadditivity (triangle inequality):
    $\Delta(X,Z) \leq \Delta(X,Y) + \Delta(Y, Z)$
  \end{itemize}
\end{lemma}

\begin{question}
  Justify why it is the case that $\Delta(X,Y) = 0 \iff X = Y$.
\end{question}

\begin{answer}
  In the forward direction, because $\Delta(X,Y) = 0$ and all the
  terms in the sum from Lemma~\ref{lem:stat-dist} are non-negligible,
  we have that $\abs{X(w) - Y(w)} = 0$ for all $w \in \Omega$, i.e.,
  $X(w) = Y(w)$. Hence, $X = Y$. In the other direction, if $X = Y$,
  then by definition $X(A) - Y(A) = 0$ for all $A$, so
  $\Delta(X,Y) = \sup(\set{0}) = 0$.
\end{answer}

Statistical distance lets us say when two (sequences of) distributions
are ``essentially the same,'' in an asymptotic sense.

\begin{definition}
  \label{def:stat-ind}
  Let $\calX = \set{X_{n}}_{n \in \N}$ and
  $\calY = \set{Y_{n}}_{n \in \N}$ be sequences of probability
  distributions, called \emph{ensembles}.  We say that $\calX$ and
  $\calY$ are \emph{statistically indistinguishable}, written
  $\calX \statind \calY$, if \[ \Delta(X_{n}, Y_{n}) = \negl(n). \]
\end{definition}

\begin{example}
  Let $X_{n}$ be the uniform distribution over $\bit^{n}$, and let
  $Y_{n}$ be the uniform distribution over the nonzero strings
  $\bit^{n} \backslash \set{0^{n}}$.  An optimal test $A$ is the
  singleton set $A = \set{0^{n}}$, yielding
  $\Delta(X_{n}, Y_{n}) = 2^{-n} = \negl(n)$, so
  $\calX \statind \calY$.  (This can also be seen by calculating the
  summation in Lemma~\ref{lem:stat-dist}.)  The analysis extends
  similarly to any $Y_{n}$ that leaves out a $\negl(n)$ fraction of
  $\bit^{n}$.  From this we can say that such ensembles $\calY$ are
  ``essentially uniform,'' or \emph{statistically pseudorandom}.
\end{example}

Question: can a statistically peudorandom generator exist?  This
depends on the definition of ``generator'' (which we give below), but
for any meaningful definition of the term, it isn't possible!  This
can be shown by explicitly demonstrating an appropriate subset (or
test) that distinguishes strings output by the generator from
uniformly random ones; see Question~\ref{ques:3} below.

\subsection{Computational Indistinguishability}
\label{sec:comp-indist}

We can define a natural analogue of statistical distance in the
computational setting, where the ``test'' is implemented by an
\emph{efficient algorithm}.  Namely, for distributions $X$ and $Y$ and
an algorithm $\Adv$ (possibly randomized), define $\Adv$'s
\emph{distinguishing advantage} between $X$ and $Y$ as
\[ \advan_{X,Y}(\Adv) = \abs*{\Pr[\Adv(X)=1] - \Pr[\Adv(Y)=1]}. \]
(The output of $\Adv$ can be arbitrary, but we interpret $1$ as a
special output indicating that the test implemented by $\Adv$ is
``satisfied,'' and any other output as ``not satisfied.'')  We extend
this to ensembles $\calX$ and $\calY$, making
$\advan_{\calX,\calY}(\Adv)$ a function of $n \in \N$.

\begin{definition}
  \label{def:comp-ind}
  Let $\calX = \set{X_{n}}$ and $\calY = \set{Y_{n}}$ be ensembles,
  where $X_{n}$ and $Y_{n}$ are distributions over $\bit^{l(n)}$ for
  $l(n) = \poly(n)$.  We say that $\calX$ and $\calY$ are
  \emph{computationally indistinguishable}, written
  $\calX \compind \calY$, if $\advan_{\calX,\calY}(\Adv) = \negl(n)$
  for all non-uniform PPT algorithms $\Adv$.  We say that $\calX$ is
  (computationally) \emph{pseudorandom} if
  $\calX \compind \set{U_{l(n)}}$, the ensemble of uniform
  distributions over $\bit^{l(n)}$.
\end{definition}

The basic facts about statistical distance also carry over to
computational indistinguishability, where all functions/tests are
restricted to be efficient.

\begin{lemma}[Composition lemma]
  \label{lem:closure}
  Let $\AdvB$ be a non-uniform PPT algorithm.  If
  $\set{X_{n}} \compind \set{Y_{n}}$, then
  $\set{\AdvB(X_{n})} \compind \set{\AdvB(Y_{n})}$.
\end{lemma}

\begin{proof}
  Let $\Dist$ be any non-uniform PPT algorithm attempting to
  distinguish $\set{\AdvB(X_{n})}$ from $\set{\AdvB(Y_{n})}$; we wish
  to show that its advantage must be negligible.  Consider an
  algorithm $\Adv$ that, given input $x$, runs $\Dist(\AdvB(x))$ and
  outputs whatever $\Dist$ outputs.  Clearly $\Adv$ is non-uniform
  PPT.  By construction, we have
  \[ \advan_{X_{n},Y_{n}}(\Adv) =
    \advan_{\AdvB(X_{n}),\AdvB(Y_{n})}(\Dist).
  \] The left-hand side is $\negl(n)$ by hypothesis, hence so is the
  right-hand side, as desired.
\end{proof}

\begin{lemma}[Hybrid lemma]
  \label{lem:hybrid}
  Let $\calX^{i} = \set{X^{i}_{n}}$ for $i \in [m]$, where
  $m=\poly(n)$.  If $\calX^{i} \compind \calX^{i+1}$ for every
  $i \in [m-1]$, then $\calX^{1} \compind \calX^{m}$.
\end{lemma}

\begin{proof}
  Let $\Dist$ be any non-uniform PPT algorithm attempting to
  distinguish $\calX^{1}$ from $\calX^{m}$.  Let
  $p_{i} = p_{i}(n) = \Pr[\Dist(X^{i}_{n}) = 1]$.  By the triangle
  inequality, we can write $\Dist$'s advantage as
  \[ \advan_{\calX^{1},\calX^{m}}(\Dist) = \abs{p_{1} - p_{m}} \leq
    \sum_{i \in [m-1]} \abs{p_{i} - p_{i+1}} = \sum_{i \in [m-1]}
    \advan_{\calX^{i}, \calX^{i+1}}(\Dist). \] Now by assumption, each
  $\advan_{\calX^{i}, \calX^{i+1}}(\Dist) = \nu_{i}(n)$, where
  $\nu_{i}(n)$ is a negligible function---which may be different for
  each $i$.  The sum of $\poly(n)$-many negligible functions is indeed
  negligible: letting $\nu(n) = \sum_{i} \nu_{i}(n)$, we need to show
  that for all $c > 0$, there exists some $n_{0}$ such that
  $\nu(n) \leq n^{-c}$ for all $n \geq n_{0}$.  By assumption, we know
  that for each $i$, there is some $n_{i}$ such that
  $\nu_{i}(n) \leq n^{-c}/m$ for \emph{all} $n \leq n_{i}$.  Letting
  $n_{0}$ be the largest of these, it follows that
  $\nu(n) \leq n^{-c}$ for all $n \geq n_{0}$, as desired.
\end{proof}

\begin{remark}
  In the proof of the lemma, we used the triangle inequality on the
  quantities $\advan_{\calX^{i},\calX^{i+1}}(\Dist)$ to conclude
  something about $\advan_{\calX^{1}, \calX^{m}}(\Dist)$.
  Syntactically this is unremarkable, but observe closely what we have
  done: even though $\Dist$'s ``goal in life'' is to distinguish
  between $\calX^{1}$ and $\calX^{m}$, by referring to the quantities
  $\advan_{\calX^{i},\calX^{i+1}}(\Dist)$, we are implicitly
  considering how $\Dist$ behaves on \emph{all} the hybrid ensembles
  $\calX^{i}$ --- these are distributions on which $\Dist$ was never
  ``designed'' to run!  Yet because $\Dist$ is ``just an algorithm,''
  we can run it and use it for whatever purposes we like.  The hybrid
  lemma says that in order for $\Dist$ to distinguish between
  $\calX^{1}$ and $\calX^{m}$, it must also distinguish between
  $\calX^{i}$ and $\calX^{i+1}$ for some $i$, which is impossible by
  hypothesis.
\end{remark}

\section{Pseudorandom Generators}
\label{sec:prgs}

\begin{definition}
  \label{def:prg}
  A \emph{deterministic} function $G : \bit^{*} \to \bit^{*}$ is a
  \emph{pseudorandom generator} (PRG) with output length $\ell(n) > n$
  if:
  \begin{itemize}
  \item $G$ can be computed by a polynomial-time algorithm,
  \item $\len{G(x)} = \ell(\len{x}) > \len{x}$ for all
    $x \in \bit^{*}$, and
  \item the ensemble $\set{G(U_{n})}$ is (computationally)
    pseudorandom.
  \end{itemize}
  
  This last property essentially says that $\set{G(U_{n})}$ and
  $\set{U_{\ell(n)}}$ are computationally indistinguishable. By the
  composition lemma, it follows that $G(U_{n})$ can be used in place
  of $U_{\ell(n)}$ in \emph{any} (efficient) application!
\end{definition}

\begin{question}
  Let \(G\) be a PRG. Is \(H(x) := \overline{G(x)}\)
  \emph{necessarily} a PRG as well?  \emph{Hint:} think about the
  composition lemma.
\end{question}
  
\begin{answer}
  Yes. The intuition here is that if \(G(x)\) looks random, then its
  complement \(\overline{G(x)}\) should look like the complement of a
  uniformly random string, which itself is uniformly random.

  The formal proof is a straightforward application of the composition
  lemma. Since \(G\) is a PRG, we have
  \(G(U_n) \compind U_{\ell(n)}\), where $\ell(n)$ is $G$'s
  expansion. Consider the efficient algorithm
  \(\AdvB(x) := \overline{x}\). Observe that
  \(H(x) = \overline{G(x)} = \AdvB(G(x))\) for any~$x$. By the
  composition lemma, we have
  \[ H(U_{n}) = \AdvB(G(U_n)) \compind \AdvB(U_{\ell(n)}) \equiv
    U_{\ell(n)}, \] where the last equivalence holds because any
  bijective function applied to the uniform distribution yields the
  uniform distribution. Thus, \(H(U_n) \compind U_{\ell(n)}\), as
  needed.
\end{answer}

\begin{question}
  Prove (or at least sketch a proof) that a \emph{statistically}
  pseudorandom generator cannot exist. \emph{Hint:} think about the
  sizes of $G(\bit^n)$ and $\bit^{\ell(n)}$, and define an appropriate
  statistical test (i.e., subset of $\bit^{\ell(n)}$).
\end{question}

\begin{answer}
  Since $G$ has $n$-bit seeds (of which there are $2^n$), there are at
  most $2^n$ possible outputs of $G$ (which reside in
  $\bit^{\ell(n)}$). However, $\bit^{\ell(n)}$ is significantly larger
  than this; it has $2^{\ell(n)} \geq 2^{n+1}$ elements, making it at
  least twice as big as the image of \(G\). Informally, we can think
  of our statistical test as checking whether or not its given string
  is in the image of $G$. When the string is an output of~$G$, the
  check always succeeds; when the string is uniformly random, there is
  a significant (at least $1/2$) probability that the check fails.

  More formally and in the language of statistical indistinguishably:
  let $A = G(\bit^n) \subseteq \bit^{\ell(n)}$, i.e., the image of
  $G$, as suggested above. Because~$G$ is deterministic, we have
  $\abs{A} \leq 2^{n}$. Then
  \begin{align*}
    \abs{G(U_n)(A) - U_{\ell(n)}(A)} &= \abs{1 - U_{\ell(n)}(A)} \\
                                     &= 1 - \abs{A} \cdot 2^{-\ell(n)}  \\
                                     &\geq 1 - 2^{n-\ell(n)} \\
                                     &\geq 1 - 2^{-1} = 1/2,
  \end{align*}
  which is (very much) non-negligible. Since we have demonstrated a
  particular~$A$ for which
  $\abs{G(U_n)(A) - U_{\ell(n)}(A)} \geq 1/2$, the suprenum over all
  possible~$A$ cannot be smaller, so
  $\Delta(G(U_n), U_{\ell(n)}) \geq 1/2$.
\end{answer}

\subsection{Expansion of a PRG}
\label{sec:properties}

From the definition, it is easy to see that the ``weakest'' PRG we
could ask for would be one that stretches its input by just 1 bit,
i.e., $\ell(n) = n+1$.  Is there an upper limit on how much a PRG can
stretch?  The following theorem says that there is (effectively)
\emph{no limit}: if you can stretch by even just 1 bit, then you can
stretch by essentially any (polynomial) amount!

\begin{theorem}
  \label{thm:expansion}
  Suppose there exists a PRG $G$ with expansion $\ell(n)=n+1$.  Then
  for any polynomial $t(\cdot) = \poly(n)$, there exists a PRG
  $G_t: \bit^n \to \bit^{t(n)}$.
\end{theorem}

\noindent
We will prove this theorem in the next lecture.

\begin{remark}
  This theorem says something extremely strong.  Observe that the
  image $\set{G_{t}(s) : s \in \bit^{n}}$ of $G_t$ is an extremely
  small fraction $2^{n-t(n)}$ of its range set $\bit^{t(n)}$.  Yet no
  computationally bounded algorithm can distinguish a random element
  from this small subset, from a truly random one over the whole
  space!
\end{remark}

\end{document}

%%% Local Variables: 
%%% mode: latex
%%% TeX-master: t
%%% End: 
