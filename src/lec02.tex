\documentclass[11pt]{article}

\usepackage{fullpage}
\usepackage{newtxtext}
\usepackage[T1]{fontenc}
\usepackage{hyperref,microtype,pdfsync}
\usepackage{amssymb}
\usepackage{algorithm,algorithmic}

\RequirePackage{mathtools}

%%% BLACKBOARD SYMBOLS

% hyperref package defines C and G, undefined to avoid conflicts
\let\C\relax
\let\G\relax
\newcommand{\C}{\ensuremath{\mathbb{C}}}
\newcommand{\D}{\ensuremath{\mathbb{D}}}
\newcommand{\F}{\ensuremath{\mathbb{F}}}
\newcommand{\G}{\ensuremath{\mathbb{G}}}
\newcommand{\J}{\ensuremath{\mathbb{J}}}
\newcommand{\N}{\ensuremath{\mathbb{N}}}
\newcommand{\Q}{\ensuremath{\mathbb{Q}}}
\newcommand{\R}{\ensuremath{\mathbb{R}}}
\newcommand{\T}{\ensuremath{\mathbb{T}}}
\newcommand{\Z}{\ensuremath{\mathbb{Z}}}
\newcommand{\QR}{\ensuremath{\mathbb{QR}}}

\newcommand{\Zt}{\ensuremath{\Z_t}}
\newcommand{\Zp}{\ensuremath{\Z_p}}
\newcommand{\Zq}{\ensuremath{\Z_q}}
\newcommand{\ZN}{\ensuremath{\Z_N}}
\newcommand{\Zps}{\ensuremath{\Z_p^*}}
\newcommand{\ZNs}{\ensuremath{\Z_N^*}}
\newcommand{\JN}{\ensuremath{\J_N}}
\newcommand{\QRN}{\ensuremath{\QR_{N}}}
\newcommand{\QRp}{\ensuremath{\QR_{p}}}

%%% THEOREM COMMANDS

\RequirePackage{amsthm}

\theoremstyle{plain}            % following are "theorem" style
\newtheorem{theorem}{Theorem}[section]
\newtheorem{lemma}[theorem]{Lemma}
\newtheorem{corollary}[theorem]{Corollary}
\newtheorem{proposition}[theorem]{Proposition}
\newtheorem{claim}[theorem]{Claim}
\newtheorem{fact}[theorem]{Fact}

\theoremstyle{definition}       % following are def style
\newtheorem{definition}[theorem]{Definition}
\newtheorem{conjecture}[theorem]{Conjecture}
\newtheorem{example}[theorem]{Example}
\newtheorem{protocol}[theorem]{Protocol}

\theoremstyle{remark}           % following are remark style
\newtheorem{remark}[theorem]{Remark}
\newtheorem{note}[theorem]{Note}

% equation numbering style
\numberwithin{equation}{section}

%%% GENERAL COMPUTING

\newcommand{\bit}{\ensuremath{\set{0,1}}}
\newcommand{\pmone}{\ensuremath{\set{-1,1}}}

% asymptotics
\DeclareMathOperator{\poly}{poly}
\DeclareMathOperator{\polylog}{polylog}
\DeclareMathOperator{\negl}{negl}
\newcommand{\Otil}{\ensuremath{\tilde{O}}}

% probability/distribution stuff
\DeclareMathOperator*{\E}{E}
\DeclareMathOperator*{\Var}{Var}

% sets in calligraphic type
\newcommand{\calD}{\ensuremath{\mathcal{D}}}
\newcommand{\calF}{\ensuremath{\mathcal{F}}}
\newcommand{\calG}{\ensuremath{\mathcal{G}}}
\newcommand{\calH}{\ensuremath{\mathcal{H}}}
\newcommand{\calX}{\ensuremath{\mathcal{X}}}
\newcommand{\calY}{\ensuremath{\mathcal{Y}}}

% types of indistinguishability
\newcommand{\compind}{\ensuremath{\stackrel{c}{\approx}}}
\newcommand{\statind}{\ensuremath{\stackrel{s}{\approx}}}
\newcommand{\perfind}{\ensuremath{\equiv}}

% font for general-purpose algorithms
\newcommand{\algo}[1]{\ensuremath{\mathsf{#1}}}
% font for general-purpose computational problems
\newcommand{\problem}[1]{\ensuremath{\mathsf{#1}}}
% font for complexity classes
\newcommand{\class}[1]{\ensuremath{\mathsf{#1}}}

% complexity classes and languages
\renewcommand{\P}{\class{P}}
\newcommand{\BPP}{\class{BPP}}
\newcommand{\NP}{\class{NP}}
\newcommand{\coNP}{\class{coNP}}
\newcommand{\AM}{\class{AM}}
\newcommand{\coAM}{\class{coAM}}
\newcommand{\IP}{\class{IP}}

%%% "LEFT-RIGHT" PAIRS OF SYMBOLS


% inner product
\DeclarePairedDelimiter\inner{\langle}{\rangle}
% absolute value
\DeclarePairedDelimiter\abs{\lvert}{\rvert}
% a set
\DeclarePairedDelimiter\set{\{}{\}}
% parens
\DeclarePairedDelimiter\parens{(}{)}
% tuple, alias for parens
\DeclarePairedDelimiter\tuple{(}{)}
% square brackets
\DeclarePairedDelimiter\bracks{[}{]}
% rounding off
\DeclarePairedDelimiter\round{\lfloor}{\rceil}
% floor function
\DeclarePairedDelimiter\floor{\lfloor}{\rfloor}
% ceiling function
\DeclarePairedDelimiter\ceil{\lceil}{\rceil}
% length of some vector, element
\DeclarePairedDelimiter\length{\lVert}{\rVert}
% "lifting" of a residue class
\DeclarePairedDelimiter\lift{\llbracket}{\rrbracket}
\DeclarePairedDelimiter\len{\lvert}{\rvert}

%%% CRYPTO-RELATED NOTATION

% KEYS AND RELATED

\newcommand{\key}[1]{\ensuremath{#1}}

\newcommand{\pk}{\key{pk}}
\newcommand{\vk}{\key{vk}}
\newcommand{\sk}{\key{sk}}
\newcommand{\mpk}{\key{mpk}}
\newcommand{\msk}{\key{msk}}
\newcommand{\fk}{\key{fk}}
\newcommand{\id}{id}
\newcommand{\keyspace}{\ensuremath{\mathcal{K}}}
\newcommand{\msgspace}{\ensuremath{\mathcal{M}}}
\newcommand{\ctspace}{\ensuremath{\mathcal{C}}}
\newcommand{\tagspace}{\ensuremath{\mathcal{T}}}
\newcommand{\idspace}{\ensuremath{\mathcal{ID}}}

\newcommand{\concat}{\ensuremath{\|}}

% GAMES

% advantage
\newcommand{\advan}{\ensuremath{\mathbf{Adv}}}

% different attack models
\newcommand{\attack}[1]{\ensuremath{\text{#1}}}

\newcommand{\atk}{\attack{atk}} % dummy attack
\newcommand{\indcpa}{\attack{ind-cpa}}
\newcommand{\indcca}{\attack{ind-cca}}
\newcommand{\anocpa}{\attack{ano-cpa}} % anonymous
\newcommand{\anocca}{\attack{ano-cca}}
\newcommand{\euacma}{\attack{eu-acma}} % forgery: adaptive chosen-message
\newcommand{\euscma}{\attack{eu-scma}} % forgery: static chosen-message
\newcommand{\suacma}{\attack{su-acma}} % strongly unforgeable

% ADVERSARIES
\newcommand{\attacker}[1]{\ensuremath{\mathcal{#1}}}

\newcommand{\Adv}{\attacker{A}}
\newcommand{\AdvA}{\attacker{A}}
\newcommand{\AdvB}{\attacker{B}}
\newcommand{\Dist}{\attacker{D}}
\newcommand{\Sim}{\attacker{S}}
\newcommand{\Ora}{\attacker{O}}
\newcommand{\Inv}{\attacker{I}}
\newcommand{\For}{\attacker{F}}

% CRYPTO SCHEMES

\newcommand{\scheme}[1]{\ensuremath{\text{#1}}}

% pseudorandom stuff
\newcommand{\prg}{\algo{PRG}}
\newcommand{\prf}{\algo{PRF}}
\newcommand{\prp}{\algo{PRP}}

% symmetric-key cryptosystem
\newcommand{\skc}{\scheme{SKC}}
\newcommand{\skcgen}{\algo{Gen}}
\newcommand{\skcenc}{\algo{Enc}}
\newcommand{\skcdec}{\algo{Dec}}

% public-key cryptosystem
\newcommand{\pkc}{\scheme{PKC}}
\newcommand{\pkcgen}{\algo{Gen}}
\newcommand{\pkcenc}{\algo{Enc}} % can also use \kemenc and \kemdec
\newcommand{\pkcdec}{\algo{Dec}}

% digital signatures
\newcommand{\sig}{\scheme{SIG}}
\newcommand{\siggen}{\algo{Gen}}
\newcommand{\sigsign}{\algo{Sign}}
\newcommand{\sigver}{\algo{Ver}}

% message authentication code
\newcommand{\mac}{\scheme{MAC}}
\newcommand{\macgen}{\algo{Gen}}
\newcommand{\mactag}{\algo{Tag}}
\newcommand{\macver}{\algo{Ver}}

% key-encapsulation mechanism
\newcommand{\kem}{\scheme{KEM}}
\newcommand{\kemgen}{\algo{Gen}}
\newcommand{\kemenc}{\algo{Encaps}}
\newcommand{\kemdec}{\algo{Decaps}}

% identity-based encryption
\newcommand{\ibe}{\scheme{IBE}}
\newcommand{\ibesetup}{\algo{Setup}}
\newcommand{\ibeext}{\algo{Ext}}
\newcommand{\ibeenc}{\algo{Enc}}
\newcommand{\ibedec}{\algo{Dec}}

% hierarchical IBE (as key encapsulation)
\newcommand{\hibe}{\scheme{HIBE}}
\newcommand{\hibesetup}{\algo{Setup}}
\newcommand{\hibeext}{\algo{Extract}}
\newcommand{\hibeenc}{\algo{Encaps}}
\newcommand{\hibedec}{\algo{Decaps}}

% binary tree encryption (as key encapsulation)
\newcommand{\bte}{\scheme{BTE}}
\newcommand{\btesetup}{\algo{Setup}}
\newcommand{\bteext}{\algo{Extract}}
\newcommand{\bteenc}{\algo{Encaps}}
\newcommand{\btedec}{\algo{Decaps}}

% trapdoor functions
\newcommand{\tdf}{\scheme{TDF}}
\newcommand{\tdfgen}{\algo{Gen}}
\newcommand{\tdfeval}{\algo{Eval}}
\newcommand{\tdfinv}{\algo{Invert}}
\newcommand{\tdfver}{\algo{Ver}}

%%% PROTOCOLS

\newcommand{\out}{\text{out}}
\newcommand{\view}{\text{view}}

%%% COMMANDS FOR LECTURES/HOMEWORKS

\RequirePackage{fancyhdr}

\newcommand{\lecheader}{%
  \chead{\large \textbf{Lecture \lecturenum\\\lecturetopic}}

  \lhead{\small \textbf{Theory of Cryptography}\\}

  \rhead{\small \textbf{Instructor:
      \href{http://www.eecs.umich.edu/~cpeikert/}{Chris Peikert}\\Scribe:
      \scribename}}

  \setlength{\headheight}{20pt}
  \setlength{\headsep}{16pt}
}


%%% xsim settings

\RequirePackage{xsim}
\RequirePackage{needspace}

\DeclareExerciseEnvironmentTemplate{QRef}
{%
    \par\vspace{\baselineskip}
    \Needspace*{2\baselineskip}
    \noindent
    \hyperref[sol:\ExerciseID]{\textbf{\XSIMmixedcase{\GetExerciseName}~\GetExerciseProperty{counter}.}}
    \label{ques:\ExerciseID}
}{}

\DeclareExerciseEnvironmentTemplate{ARef}{%
        \Needspace*{2\baselineskip}
        \noindent
        \hyperref[ques:\ExerciseID]{\textbf{\XSIMmixedcase{\GetExerciseParameter{exercise-name}}~\GetExerciseProperty{counter}.}}
        \label{sol:\ExerciseID}
        \GetExerciseBody{exercise}
        \newline
        \textbf{\XSIMmixedcase{\GetExerciseName}.}
}{\par\vspace{\baselineskip}}

\DeclareExerciseHeadingTemplate{simple}{%
    \section*{\XSIMmixedcase{\GetExerciseParameter{solution-name}s}}
}

\DeclareExerciseType{question}{%
    exercise-env = question,
    solution-env = answer,
    exercise-name = Question,
    solution-name = Answer,
    exercise-template = QRef,
    solution-template = ARef,
}

\xsimsetup{%
  path = {questions},
  file-extension = {aux},
  print-solutions/headings-template=simple
}

\AtEndDocument{\pagebreak\printsolutions}


% VARIABLES

\newcommand{\lecturenum}{2}
\newcommand{\lecturetopic}{Computational Hardness}
\newcommand{\scribename}{George P.~Burdell}

% END OF VARIABLES

\lecheader

\pagestyle{plain}               % default: no special header

\begin{document}

\thispagestyle{fancy}           % first page should have special header

% LECTURE MATERIAL STARTS HERE

\section{Recap: Perfect Secrecy}
\label{sec:recap:-perf-secr}

Recall from last time: 
\begin{itemize}
\item Shannon secrecy says that the \textit{a posteriori} distribution
  over the message, given the ciphertext, is identical to the
  \textit{a priori} distribution.  (See the lecture notes for formal
  definition.)  Informally, ``seeing the ciphertext is only as good as
  seeing nothing at all.''
\item Perfect secrecy is the property that the distribution of the
  ciphertext $c \in \ctspace$ (over the choice of key $k \in
  \keyspace$) is exactly the same, no matter what message $m
  \in \msgspace$ was encrypted.
\item Shannon secrecy and perfect secrecy are equivalent.
\item The one-time pad scheme enjoys Shannon/perfect secrecy.
\end{itemize}

\section{Limits of Shannon/Perfect Secrecy}
\label{sec:limits-perf-secr}

For reference, here again is the definition of Shannon secrecy:

\begin{definition}[Shannon secrecy]
  \label{def:shannon-secrecy}
  A shared-key encryption scheme $(\skcgen,\skcenc,\skcdec)$ with
  message space $\msgspace$ and ciphertext space $\ctspace$ is
  \emph{Shannon secret with respect to a probability distribution $D$}
  over $\msgspace$ if for all $\bar{m} \in \msgspace$ and all $\bar{c}
  \in \ctspace$,
  \begin{equation}
    \label{eq:shannon}
    \Pr_{m \gets D,\; k \gets \skcgen}[m = \bar{m}\; |\;
    \skcenc_{k}(m) = \bar{c} ] = \Pr_{m \gets D}[ m = \bar{m} ].
  \end{equation}
  The scheme is \emph{Shannon secret} if it is Shannon secret with
  respect to every distribution $D$ over $\msgspace$.
\end{definition}

Now that we know that the one-time pad is Shannon secret, what more is
left to do?  (Cancel the course?)  Well, first notice that the
one-time pad scheme is not necessarily very usable: the key length is
as large as the message length, and for long messages it may be
unrealistic for Alice and Bob to establish (and keep secret!)  a huge
key before communicating.  (This is not to mention the additional
distinct keys needed for when Carol, Dan, Edith, and Fred enter the
picture\ldots) Also notice that the key can by used only \emph{one
  time}: encrypting more than one message with the same key is
``abusing the model,'' and causes the security proof to break down.
In fact, using the key more than once often makes the scheme
\emph{completely insecure}.  (The USSR~learned this the hard way as a
result of the US/UK-led ``Venona'' project during the Cold War.)

It turns out that these drawbacks are not really a deficiency of the
one-time pad itself, but are \emph{inherent} in the strong
requirement of perfect secrecy.

\begin{theorem}[Shannon's theorem]
  \label{thm:shannon}
  If a shared-key encryption scheme (with key space $\keyspace$ and
  message space $\msgspace$) is Shannon secret, then
  $\abs{\keyspace} \geq \abs{\msgspace}$.
\end{theorem}

\begin{proof}
  We prove the contrapositive: supposing that $\abs{\keyspace} <
  \abs{\msgspace}$ for some scheme $(\skcgen, \skcenc, \skcdec)$, we
  will show that the scheme is not Shannon secret.  It suffices to
  show the existence of a distribution $D$ over $\msgspace$, a fixed
  message $\bar{m} \in \msgspace$, and a ciphertext $\bar{c} \in
  \ctspace$ such that the two sides of~\eqref{eq:shannon} are unequal.

  Let $D$ be the uniform distribution over $\msgspace$, let $m \in
  \msgspace$ be arbitrary, let $\bar{k} \in \keyspace$ be an arbitrary
  key in the support of $\skcgen$ (i.e., $\skcgen$ outputs $\bar{k}$
  with probability $> 0$), and let $\bar{c}$ be any ciphertext in the
  support of $\skcenc_{\bar{k}}(m)$ (keeping in the mind that
  $\skcenc$ may be randomized).  Now assume without loss of generality
  that $\skcdec$ is deterministic (exercise: prove that it's really
  w.l.o.g., using the correctness of decryption), and define $\calD =
  \set{ \skcdec_{k}(\bar{c})\; :\; k \in \keyspace } \subseteq
  \msgspace$ to be the set of all possible decryptions of $\bar{c}$
  under all possible keys.  We must have $\abs{\calD} \leq
  \abs{\keyspace} < \abs{\msgspace}$, which implies that there exists
  some message $\bar{m} \in \msgspace \setminus \calD$.  By
  correctness of decryption, and because $D$ is the uniform
  distribution over $\msgspace$, we have
  \[ 0 = \Pr_{m \gets D, k \gets \skcgen}[m=\bar{m}\; |\;
  \skcenc_{k}(m)=\bar{c}] < \Pr_{m \gets D}[m=\bar{m}] =
  \frac{1}{\abs{\msgspace}}, \] as desired.
\end{proof}

Notice that the proof above is ``effective'' in the sense that
$\bar{m}$ can be computed by explicitly enumerating
$\skcdec_{k}(\bar{c})$ for every $k \in \keyspace$.  However, if
$\keyspace$ is huge, this task may take too long to be feasible.

\section{Roadmap to Computational Security}
\label{sec:roadmap-comp-sec}

We have seen that perfect secrecy is obtainable, but impractical ---
the key must be as long as the message.  Moreover, the proof of this
fact provides an actual attack, which enumerates all possible keys.
On the one hand, if the key space is huge (say, $2^{1000}$), the
attack may need to run for a long, long time.  On the other hand,
maybe some schemes fall to faster attacks that don't have to
enumerate all keys...

Today (and throughout most of the rest of course), we will be focused
on schemes that are meaningfully secure (though not ``perfectly'' so)
\emph{as long as the adversary does not employ an absurdly large
  amount of computation}.  We have good reason to believe that there
are inherent limits on the amount of computation that can be performed
in the physical universe --- for example, the number of atoms in the
observable universe is estimated at around $10^{80}$, and we have only
about $10^{10}$ years until our sun runs out of fuel and collapses.

The \emph{computationally secure} schemes we develop will have keys
much shorter than the message length, but this is really just the
beginning: they will also have absurdly strong (seemingly even
paradoxical) security and functionality properties that go far beyond
just keeping a message secret.

To get there from here, there are many questions to which we must give
precise mathematical answers.  here are just a few:
\begin{itemize}
\item How do we \emph{model} ``bounded computation''?
\item What does it mean for a problem to be \emph{hard} for bounded
  computation?
\item Where might we hope to \emph{find} such hard problems?
\item How do we \emph{define} security against computationally bounded
  adversaries?
\item How do we \emph{use} hard problems to design schemes that
  satisfy our security definitions?
\end{itemize}

To keep a concrete example in mind, perhaps you have heard the
(informal) conjecture that ``factoring is hard.''  We will see how to
state this conjecture (and others like it) precisely, and start to use
it in cryptography.

\section{Computational Model}
\label{sec:computational-model}

\subsection{Algorithms}
\label{sec:algorithms}

Informally, we all know what an algorithm is: something that you can
implement in your favorite programming language, and run on any kind
of computer (or network of computers) that you might like.  As a
formal mathematical object, an algorithm is a \emph{Turing machine}.
In this course we will not need the formal definition too often, and
the informal notion will usually suffice.

The \emph{running time} of an algorithm on an input is the number of
``basic'' steps it performs before terminating.  Basic steps include
reading or writing to a location in memory, performing an arithmetic
operation (on fixed-size pieces of data), etc.  For our purposes, the
exact set of basic operations will not be too important, because it is
possible to translate between different sets of basic operations with
only polynomial slowdown: an algorithm (using one set of basic ops)
with running time $T$ can be converted into an algorithm (using
another set) with running time $d T^{c}$, for some fixed constants
$c,d$.  The running time $T(n)$ of an algorithm as a function of input
length $n$ is its \emph{maximum} running time over all inputs of
length $n$.  We say that an algorithm is \emph{polynomial-time} if its
running time $T(n)=O(n^{c})$ for some fixed constant $c \geq 0$.

Two aspects of our model of an algorithm require a bit more precision.
The first is \emph{randomization}, that is, we grant algorithms the
ability to ``flip coins.''  In the Turing machine view, the machine is
augmented with a read-only ``random tape,'' which is initialized with
an infinite string of uniformly random and independent bits.
Alternatively, we can view the algorithm's randomness as an extra
argument $r \in \bit^{*}$, and we denote the output of $\Adv$ on input
$x$ with ``coins'' $r$ as $\Adv(x; r)$.  (When omitted, $r$ is taken
to be a uniformly random string.)  Note that $\Adv(x)$ defines a
probability distribution over its outputs, where the probability is
taken over the random coins.  The running time $T$ of a probabilistic
algorithm $\Adv$ on an input $x$ is the \emph{maximum} number of steps
performed by $\Adv(x;r)$, over all random strings $r$.  The notions of
running time as a function of the input length $n = \len{x}$, and
\emph{probabilistic polynomial-time} (PPT), are defined similarly.
Note that a PPT algorithm can only ``look at'' polynomially many bits
of its random string.

In this course, PPT algorithms will be our main model of ``efficient
computation'' for cryptographic schemes.  This is primarily to give us
a \emph{useful, well-behaved abstraction}, which allows us not to get
too caught up in the details of implementations.  Of course, an
algorithm with running time $n^{100}$ is not very useful in practice,
and at times we will take a more precise view of running times.

The second aspect of our model, which we use only for modelling
\emph{adversaries}, is \emph{non-uniformity}.  The idea is that we
allow an adversary to have some extra ``advice'' that depends only on
the \emph{length} of its input.  This advice can only increase what
the algorithm is capable of computing, so security against non-uniform
adversaries is potentially stronger than against only uniform ones.
The ability to have advice will also simplify some of our security
proofs.\footnote{It also makes Turing machines equivalent to ``circuit
  families,'' which leads to more robust security definitions.}  More
formally, we say that $\Adv$ is a non-uniform algorithm if there
exists an infinite sequence $w_{1}, w_{2}, \ldots \in \bit^{*}$ so
that on input $x$, $\Adv$ is also given the extra argument
$w_{\len{x}}$.  The notion of PPT for non-uniform algorithms is as
above; in particular, note that a non-uniform PPT machine can only
``look at'' polynomially many bits of its advice string, so without
loss of generality we can assume that $\len{w_{n}} = O(n^{c})$ for
some constant $c \geq 0$.

\subsection{Asymptotics}
\label{sec:asymptotics}

We have defined running time as a function of input length $n$, and we
will also frequently do so for probabilities.  The length $n$ is
frequently called the \emph{security parameter} of a cryptographic
scheme, because it (informally) lets us specify ``how much'' security
we want.

We say that a function $T(n)$ is \emph{polynomial}, written $T(n) =
\poly(n)$, if $T(n) = O(n^{c})$ for some fixed constant $c$.  A
nonnegative function $\nu(n)$ is \emph{negligible}, written $\nu(n) =
\negl(n)$, if it vanishes faster than the inverse of any polynomial:
$\nu(n) = o(n^{-c})$ for every constant $c > 0$, or equivalently,
$\lim_{n \to \infty} \nu(n) \cdot n^{c} = 0$.  We usually apply
the concept of a negligible function to quantify the probability of an
``extremely rare'' event, i.e., one that will effectively ``never
occur'' in a polynomial-time universe.  As with polynomial time, the
concept of a negligible function is a \emph{useful} and
\emph{well-behaved} abstraction, though at times we will need to be
more precise in our manipulations of probabilities.

\begin{question}
    Is the product of a negligible function and a non-negligible function always
    negligible, always non-negligible, or neither?
\end{question}
\begin{answer}
    Neither, it depends on the functions. Consider \(f(n) = 2^{-n}\) (negligible) and
    \(g(n) = 2^n\) (non-negligible).  Then, \(f(n) \cdot g(n) = 1\) which is clearly
    non-negligible. On the other hand, suppose \(f(n) = 2^{-n}\) (negligible) and \(g(n)
    = 1\) (non-negligible). Then \(f(n) \cdot g(n) = 2^{-n}\) which is negligible.
\end{answer}

\section{One-Way Functions}
\label{sec:one-way-functions}

There are many different notions of computational ``hardness'' in
computer science, e.g.: undecidability (exemplified by the Halting
Problem), $\NP$-completeness (circuit satisfiability), time/space
hierarchies, and others.

In cryptography, we have a few notions of hardness, the most basic of
which is \emph{one-wayness}.  A \emph{one-way function} (OWF) is often
called the ``minimal'' cryptographic object, because a scheme
satisfying almost any interesting computational security notion will
imply the existence of a OWF.  (This is obviously an informal
statement, but a good rule of thumb.  We will see some examples later
on.)  In other words, we can't have (computational) cryptography
without one-way functions.

\begin{definition}
  \label{def:owf}
  A function $f : \bit^{*} \to \bit^{*}$ is \emph{one-way} if it
  satisfies the following conditions.
  \begin{itemize}
  \item \emph{Easy to compute.}  There is an efficient algorithm
    computing $f$.  Formally, there exists a (uniform, deterministic)
    poly-time algorithm $\algo{F}$ such that $\algo{F}(x) = f(x)$ for
    all $x \in \bit^{*}$.
  \item \emph{Hard to invert.}  An efficient algorithm inverts $f$ on
    a random input with only negligible probability.  Formally, for
    any non-uniform PPT algorithm $\Inv$, the \emph{advantage} of
    $\Inv$ \[ \advan_{f}(\Inv) := \Pr_{x \gets \bit^{n}} \bracks*{
      \Inv(1^{n}, f(x)) \in f^{-1}(f(x)) } = \negl(n) \] is a
    negligible function (in the security parameter $n$).
  \end{itemize}
\end{definition}

\noindent Some remarks on this definition:

\begin{enumerate}
\item On $f^{-1}(f(x))$ versus $x$: notice that the inverter $\Inv$
  ``wins'' if it outputs \emph{any} preimage of $f(x)$, i.e., any
  element in the set $f^{-1}(f(x))$.  (The preimage need not even have
  length $n$.)  This is mainly to rule out trivial function, such as
  the function $f$ that just maps every input to $0$.  If we required
  the inverter to output the \emph{original} $x \in \bit^{n}$, then
  the best any inverter could do against this $f$ would be to guess
  randomly, which succeeds with only negligible probability $2^{-n}$.
  But this $f$ does not satisfy our intuitive idea of a function that
  is ``hard to invert.''
  
\item The choice of $\bit^{*}$ as the domain and range of $f$ is
  mainly for syntactic convenience, and it may be replaced by any
  finite sets, one for each value of the security parameter $n$.
  There should also be an efficient algorithm for sampling from the
  domain, so that the entire inversion experiment can be executed
  efficiently.

\item In contrast to some other notions of hardness you may have seen,
  this definition is inherently \emph{average-case} rather than
  \emph{worst-case}.  A worst-case definition would say that every
  $\Inv$ fails to invert $f(x)$ for \emph{some} (possibly rare) value
  of $x$.  Our definition is potentially much stronger: it says that
  $\Inv$ fails to invert $f(x)$ for \emph{almost all} values of $x \in
  \bit^{n}$ (for large enough $n$).\label{item:worst-average}
  
\item Notice that in addition to the value $f(x)$ to be inverted, the
  input to $\Inv$ includes the string $1^{n}$.  This is simply a
  technicality that always allows the inverter to run in time
  $\poly(n)$, even if $f(x)$ has length much less than $n$.  Usually
  we will omit this extra argument, with the implicit understanding
  that all algorithms are given the security parameter (in unary), and
  may run in time polynomial in it.
\end{enumerate}

\begin{question}
    Notice that our model allows for \(\Inv\) to have some advantage, even if it is
    negligible. Would a stronger model that requires \(\Inv\) to have an advantage of 0
    (i.e. a model that requires that it be impossible to find a preimage in
    polynomial time) be possible? Why or why not?
\end{question}
\begin{answer}
    No. Consider an \(\Inv\) that ignores the \(f(x)\) it is given and simply guesses a
    uniformly random preimage from \(\bit^n\). This can be done in polynomial time and
    has a nonzero probability of \(2^{-n}\) of succeeding. Hence, there are no functions
    that could be considered one-way if we defined it this way. In cryptography, when we
    can't achieve the ``ideal,'' we can often do the next best thing by using a
    negligible function to mean ``effectively zero.''
\end{answer}

\begin{question}
    Let \(f\) be a one-way function. Is \(g(x) = f(x) \concat 0\) also one-way?
\end{question}
\begin{answer}
    Yes. Intuitively 
\end{answer}

\subsection{Candidates}
\label{sec:candidates}

We have defined the concept of a one-way function, but does such an
object exist?  First, it can be shown that if a one-way function
exists, then $\P \neq \NP$.  Since we have no idea how to show that
$\P \neq \NP$, \emph{proving} that a one-way function exists is far
beyond our reach.  But we have many \emph{candidate} functions that we
believe to be one-way.  Consider these two examples:
\begin{enumerate}
\item Subset-sum: define $f_{\text{ss}} : (\Z_{N})^{n} \times \bit^{n}
  \to (\Z_{N})^{n} \times \Z_{N}$, where $N = 2^{n}$, as \[
  f_{\text{ss}}(a_{1}, \ldots, a_{n}, b_{1}, \ldots, b_{n}) = (a_{1},
  \ldots, a_{n}, S = \sum_{i\; :\; b_{i} = 1} a_{i} \bmod N). \]
  Notice that because the $a_{i}$s are part of the output, inverting
  $f$ means finding a subset of $\set{1,\ldots,n}$ (corresponding to
  those $b_{i}$s equaling $1$) that induces the given sum $S$ modulo
  $N$.
\item Multiplication: define $f_{\text{mult}} : \N^{2} \to \N$
  as\footnote{For security parameter $n$, we use the domain $[1,2^{n}]
    \times [1,2^{n}]$.  The cases $x=1$ and $y=1$ are special to rule
    out the trivial preimages $(1,xy)$ and $(xy, 1)$ for the output
    $xy \neq 1$.}
  \[
  f_{\text{mult}}(x,y) =
  \begin{cases}
    1 & \text{if } x=1 \vee y=1 \\
    x \cdot y & \text{otherwise.}
  \end{cases}
  \]
\end{enumerate}

Are these one-way functions, according to our definition?  This is
something for you to think about until the next lecture.

\end{document}

%%% Local Variables: 
%%% mode: latex
%%% TeX-master: t
%%% End: 
