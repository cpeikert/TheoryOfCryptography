\documentclass[11pt]{article}

\usepackage{fullpage,microtype,pdfsync,hyperref}
\usepackage{newtxtext}
\usepackage[T1]{fontenc}

\usepackage{mathtools,amsfonts,amssymb}
\usepackage[amsmath,amsthm,thmmarks,hyperref]{ntheorem}
\usepackage[capitalize,nameinlink,noabbrev]{cleveref}

\RequirePackage{mathtools}

%%% BLACKBOARD SYMBOLS

% hyperref package defines C and G, undefined to avoid conflicts
\let\C\relax
\let\G\relax
\newcommand{\C}{\ensuremath{\mathbb{C}}}
\newcommand{\D}{\ensuremath{\mathbb{D}}}
\newcommand{\F}{\ensuremath{\mathbb{F}}}
\newcommand{\G}{\ensuremath{\mathbb{G}}}
\newcommand{\J}{\ensuremath{\mathbb{J}}}
\newcommand{\N}{\ensuremath{\mathbb{N}}}
\newcommand{\Q}{\ensuremath{\mathbb{Q}}}
\newcommand{\R}{\ensuremath{\mathbb{R}}}
\newcommand{\T}{\ensuremath{\mathbb{T}}}
\newcommand{\Z}{\ensuremath{\mathbb{Z}}}
\newcommand{\QR}{\ensuremath{\mathbb{QR}}}

\newcommand{\Zt}{\ensuremath{\Z_t}}
\newcommand{\Zp}{\ensuremath{\Z_p}}
\newcommand{\Zq}{\ensuremath{\Z_q}}
\newcommand{\ZN}{\ensuremath{\Z_N}}
\newcommand{\Zps}{\ensuremath{\Z_p^*}}
\newcommand{\ZNs}{\ensuremath{\Z_N^*}}
\newcommand{\JN}{\ensuremath{\J_N}}
\newcommand{\QRN}{\ensuremath{\QR_{N}}}
\newcommand{\QRp}{\ensuremath{\QR_{p}}}

%%% THEOREM COMMANDS

\RequirePackage{amsthm}

\theoremstyle{plain}            % following are "theorem" style
\newtheorem{theorem}{Theorem}[section]
\newtheorem{lemma}[theorem]{Lemma}
\newtheorem{corollary}[theorem]{Corollary}
\newtheorem{proposition}[theorem]{Proposition}
\newtheorem{claim}[theorem]{Claim}
\newtheorem{fact}[theorem]{Fact}

\theoremstyle{definition}       % following are def style
\newtheorem{definition}[theorem]{Definition}
\newtheorem{conjecture}[theorem]{Conjecture}
\newtheorem{example}[theorem]{Example}
\newtheorem{protocol}[theorem]{Protocol}

\theoremstyle{remark}           % following are remark style
\newtheorem{remark}[theorem]{Remark}
\newtheorem{note}[theorem]{Note}

% equation numbering style
\numberwithin{equation}{section}

%%% GENERAL COMPUTING

\newcommand{\bit}{\ensuremath{\set{0,1}}}
\newcommand{\pmone}{\ensuremath{\set{-1,1}}}

% asymptotics
\DeclareMathOperator{\poly}{poly}
\DeclareMathOperator{\polylog}{polylog}
\DeclareMathOperator{\negl}{negl}
\newcommand{\Otil}{\ensuremath{\tilde{O}}}

% probability/distribution stuff
\DeclareMathOperator*{\E}{E}
\DeclareMathOperator*{\Var}{Var}

% sets in calligraphic type
\newcommand{\calD}{\ensuremath{\mathcal{D}}}
\newcommand{\calF}{\ensuremath{\mathcal{F}}}
\newcommand{\calG}{\ensuremath{\mathcal{G}}}
\newcommand{\calH}{\ensuremath{\mathcal{H}}}
\newcommand{\calX}{\ensuremath{\mathcal{X}}}
\newcommand{\calY}{\ensuremath{\mathcal{Y}}}

% types of indistinguishability
\newcommand{\compind}{\ensuremath{\stackrel{c}{\approx}}}
\newcommand{\statind}{\ensuremath{\stackrel{s}{\approx}}}
\newcommand{\perfind}{\ensuremath{\equiv}}

% font for general-purpose algorithms
\newcommand{\algo}[1]{\ensuremath{\mathsf{#1}}}
% font for general-purpose computational problems
\newcommand{\problem}[1]{\ensuremath{\mathsf{#1}}}
% font for complexity classes
\newcommand{\class}[1]{\ensuremath{\mathsf{#1}}}

% complexity classes and languages
\renewcommand{\P}{\class{P}}
\newcommand{\BPP}{\class{BPP}}
\newcommand{\NP}{\class{NP}}
\newcommand{\coNP}{\class{coNP}}
\newcommand{\AM}{\class{AM}}
\newcommand{\coAM}{\class{coAM}}
\newcommand{\IP}{\class{IP}}

%%% "LEFT-RIGHT" PAIRS OF SYMBOLS


% inner product
\DeclarePairedDelimiter\inner{\langle}{\rangle}
% absolute value
\DeclarePairedDelimiter\abs{\lvert}{\rvert}
% a set
\DeclarePairedDelimiter\set{\{}{\}}
% parens
\DeclarePairedDelimiter\parens{(}{)}
% tuple, alias for parens
\DeclarePairedDelimiter\tuple{(}{)}
% square brackets
\DeclarePairedDelimiter\bracks{[}{]}
% rounding off
\DeclarePairedDelimiter\round{\lfloor}{\rceil}
% floor function
\DeclarePairedDelimiter\floor{\lfloor}{\rfloor}
% ceiling function
\DeclarePairedDelimiter\ceil{\lceil}{\rceil}
% length of some vector, element
\DeclarePairedDelimiter\length{\lVert}{\rVert}
% "lifting" of a residue class
\DeclarePairedDelimiter\lift{\llbracket}{\rrbracket}
\DeclarePairedDelimiter\len{\lvert}{\rvert}

%%% CRYPTO-RELATED NOTATION

% KEYS AND RELATED

\newcommand{\key}[1]{\ensuremath{#1}}

\newcommand{\pk}{\key{pk}}
\newcommand{\vk}{\key{vk}}
\newcommand{\sk}{\key{sk}}
\newcommand{\mpk}{\key{mpk}}
\newcommand{\msk}{\key{msk}}
\newcommand{\fk}{\key{fk}}
\newcommand{\id}{id}
\newcommand{\keyspace}{\ensuremath{\mathcal{K}}}
\newcommand{\msgspace}{\ensuremath{\mathcal{M}}}
\newcommand{\ctspace}{\ensuremath{\mathcal{C}}}
\newcommand{\tagspace}{\ensuremath{\mathcal{T}}}
\newcommand{\idspace}{\ensuremath{\mathcal{ID}}}

\newcommand{\concat}{\ensuremath{\|}}

% GAMES

% advantage
\newcommand{\advan}{\ensuremath{\mathbf{Adv}}}

% different attack models
\newcommand{\attack}[1]{\ensuremath{\text{#1}}}

\newcommand{\atk}{\attack{atk}} % dummy attack
\newcommand{\indcpa}{\attack{ind-cpa}}
\newcommand{\indcca}{\attack{ind-cca}}
\newcommand{\anocpa}{\attack{ano-cpa}} % anonymous
\newcommand{\anocca}{\attack{ano-cca}}
\newcommand{\euacma}{\attack{eu-acma}} % forgery: adaptive chosen-message
\newcommand{\euscma}{\attack{eu-scma}} % forgery: static chosen-message
\newcommand{\suacma}{\attack{su-acma}} % strongly unforgeable

% ADVERSARIES
\newcommand{\attacker}[1]{\ensuremath{\mathcal{#1}}}

\newcommand{\Adv}{\attacker{A}}
\newcommand{\AdvA}{\attacker{A}}
\newcommand{\AdvB}{\attacker{B}}
\newcommand{\Dist}{\attacker{D}}
\newcommand{\Sim}{\attacker{S}}
\newcommand{\Ora}{\attacker{O}}
\newcommand{\Inv}{\attacker{I}}
\newcommand{\For}{\attacker{F}}

% CRYPTO SCHEMES

\newcommand{\scheme}[1]{\ensuremath{\text{#1}}}

% pseudorandom stuff
\newcommand{\prg}{\algo{PRG}}
\newcommand{\prf}{\algo{PRF}}
\newcommand{\prp}{\algo{PRP}}

% symmetric-key cryptosystem
\newcommand{\skc}{\scheme{SKC}}
\newcommand{\skcgen}{\algo{Gen}}
\newcommand{\skcenc}{\algo{Enc}}
\newcommand{\skcdec}{\algo{Dec}}

% public-key cryptosystem
\newcommand{\pkc}{\scheme{PKC}}
\newcommand{\pkcgen}{\algo{Gen}}
\newcommand{\pkcenc}{\algo{Enc}} % can also use \kemenc and \kemdec
\newcommand{\pkcdec}{\algo{Dec}}

% digital signatures
\newcommand{\sig}{\scheme{SIG}}
\newcommand{\siggen}{\algo{Gen}}
\newcommand{\sigsign}{\algo{Sign}}
\newcommand{\sigver}{\algo{Ver}}

% message authentication code
\newcommand{\mac}{\scheme{MAC}}
\newcommand{\macgen}{\algo{Gen}}
\newcommand{\mactag}{\algo{Tag}}
\newcommand{\macver}{\algo{Ver}}

% key-encapsulation mechanism
\newcommand{\kem}{\scheme{KEM}}
\newcommand{\kemgen}{\algo{Gen}}
\newcommand{\kemenc}{\algo{Encaps}}
\newcommand{\kemdec}{\algo{Decaps}}

% identity-based encryption
\newcommand{\ibe}{\scheme{IBE}}
\newcommand{\ibesetup}{\algo{Setup}}
\newcommand{\ibeext}{\algo{Ext}}
\newcommand{\ibeenc}{\algo{Enc}}
\newcommand{\ibedec}{\algo{Dec}}

% hierarchical IBE (as key encapsulation)
\newcommand{\hibe}{\scheme{HIBE}}
\newcommand{\hibesetup}{\algo{Setup}}
\newcommand{\hibeext}{\algo{Extract}}
\newcommand{\hibeenc}{\algo{Encaps}}
\newcommand{\hibedec}{\algo{Decaps}}

% binary tree encryption (as key encapsulation)
\newcommand{\bte}{\scheme{BTE}}
\newcommand{\btesetup}{\algo{Setup}}
\newcommand{\bteext}{\algo{Extract}}
\newcommand{\bteenc}{\algo{Encaps}}
\newcommand{\btedec}{\algo{Decaps}}

% trapdoor functions
\newcommand{\tdf}{\scheme{TDF}}
\newcommand{\tdfgen}{\algo{Gen}}
\newcommand{\tdfeval}{\algo{Eval}}
\newcommand{\tdfinv}{\algo{Invert}}
\newcommand{\tdfver}{\algo{Ver}}

%%% PROTOCOLS

\newcommand{\out}{\text{out}}
\newcommand{\view}{\text{view}}

%%% COMMANDS FOR LECTURES/HOMEWORKS

\RequirePackage{fancyhdr}

\newcommand{\lecheader}{%
  \chead{\large \textbf{Lecture \lecturenum\\\lecturetopic}}

  \lhead{\small \textbf{Theory of Cryptography}\\}

  \rhead{\small \textbf{Instructor:
      \href{http://www.eecs.umich.edu/~cpeikert/}{Chris Peikert}\\Scribe:
      \scribename}}

  \setlength{\headheight}{20pt}
  \setlength{\headsep}{16pt}
}


%%% xsim settings

\RequirePackage{xsim}
\RequirePackage{needspace}

\DeclareExerciseEnvironmentTemplate{QRef}
{%
    \par\vspace{\baselineskip}
    \Needspace*{2\baselineskip}
    \noindent
    \hyperref[sol:\ExerciseID]{\textbf{\XSIMmixedcase{\GetExerciseName}~\GetExerciseProperty{counter}.}}
    \label{ques:\ExerciseID}
}{}

\DeclareExerciseEnvironmentTemplate{ARef}{%
        \Needspace*{2\baselineskip}
        \noindent
        \hyperref[ques:\ExerciseID]{\textbf{\XSIMmixedcase{\GetExerciseParameter{exercise-name}}~\GetExerciseProperty{counter}.}}
        \label{sol:\ExerciseID}
        \GetExerciseBody{exercise}
        \newline
        \textbf{\XSIMmixedcase{\GetExerciseName}.}
}{\par\vspace{\baselineskip}}

\DeclareExerciseHeadingTemplate{simple}{%
    \section*{\XSIMmixedcase{\GetExerciseParameter{solution-name}s}}
}

\DeclareExerciseType{question}{%
    exercise-env = question,
    solution-env = answer,
    exercise-name = Question,
    solution-name = Answer,
    exercise-template = QRef,
    solution-template = ARef,
}

\xsimsetup{%
  path = {questions},
  file-extension = {aux},
  print-solutions/headings-template=simple
}

\AtEndDocument{\pagebreak\printsolutions}


% VARIABLES

\newcommand{\lecturenum}{10}
\newcommand{\lecturetopic}{Asymmetric Encryption}
\newcommand{\scribename}{Joel Odom}

% END OF VARIABLES

\lecheader

\pagestyle{plain}               % default: no special header

\begin{document}

\thispagestyle{fancy}          % first page should have special header

% LECTURE MATERIAL STARTS HERE

\newcommand{\challenge}{\algo{C}}

\section{Symmetric Encryption}
\label{sec:symm-encrypt}

\subsection{Recap}
\label{sec:recap}

Recall that in the last lecture we discussed cryptographic security in
the symmetric setting.  The typical arrangement is that Alice and Bob
share a common key that they use for encryption and decryption.  This
shared secret creates an distinction between the communicating parties
(Alice and Bob) and the adversarial eavesdropper (Eve).  The goal in
this setting is to ensure that without the shared secret key, Eve will
not be able to learn anything about the messages being sent between
Alice and Bob (except, of course, the fact that Alice and Bob are
communicating via messages from a known space).

Perfect secrecy (equivalently, Shannon secrecy) is typically
impractical in the symmetric setting because it requires that the
shared key be at least as large as any message sent, and each key may
only be used once.  For practicality, we relaxed our notion of perfect
secrecy to allow for a computationally bounded adversary, and we
arrived at the notion of \emph{indistinguishability under (adaptive)
  chosen plaintext attack} (IND-CPA).  In this security notion, we
allow a computationally bounded adversary access to an encryption
oracle $\skcenc_{k}(\cdot)$ and a challenge oracle
$\challenge_{k}^{b}(\cdot, \cdot)$. The adversary may make (a
polynomially bounded number of) calls to the encryption oracle,
adapting its queries to the answers it has already received.  The
adversary may make a single call to the challenge oracle, passing two
messages of his choice.  The challenge oracle encrypts one of the two
messages (determined by the bit $b$) and returns the ciphertext to the
adversary.  The adversary examines the returned ciphertext and
attempts to determine which of the two messages was encrypted.  If
there does not exist \emph{any} efficient adversary that can tell
which message was encrypted (with better than negligible advantage),
then the scheme is considered IND-CPA secure.  Formally, the pairs of
oracle \[ \langle \skcenc_{k}(\cdot), \challenge_{k}^{b}(\cdot, \cdot)
:= \skcenc_{k}(m_{b}) \rangle \] are indistinguishable for $b = 0,1$
(over the random choice of $k \gets \skcgen$ and any randomness of the
oracles).

Last time, we constructed the following symmetric cryptosystem $\skc$
with $\msgspace = \keyspace = \bit^{n}$, based on a PRF family
$\set{f_{k} : \bit^{n} \to \bit^{n}}$.

\begin{itemize}
\item $\skcgen$: output $k \gets \bit^{n}$.
\item $\skcenc_{k}(m)$: choose $r \gets \bit^{n}$ and output $c = (r,
  f_{k}(r) \oplus m)$
\item $\skcdec_{k}(r, c = (r,c'))$: output $f_{k}(r) \oplus c$
\end{itemize}
Completeness is evident by inspection.

\begin{question}
  Let \(F \colon \bit^n \to \bit^n\) be a uniformly random
  function. Is an oracle which on input \(x\) returns
  \(r_x \concat F(r_x))\), where \(r_x\) is an independent and
  uniformly random $n$ bit string, identical to an oracle for a
  uniformly random function \(\bit^n \to \bit^{2n}\)?
\end{question}

\begin{answer}
  No. For the \(r_x \concat F(r_x)\) case, the second half of the
  bitstring is entirely determined by the first half. To show that
  they differ, let's consider the probability that there is a
  collision in each world. For there to be a collision in the first
  world, all we need are two distinct inputs \(x\) and \(x'\) such
  that \(r_x = r_{x'}\). In other words, for a collision to occur, of
  the \(2^n\) uniformly random and independent \(n\)-bit \(r_x\)s, at
  least two must be identical. On the other hand in order for there to
  be a collision in the random function case, of the \(2^n\) uniformly
  random and independent \(2n\)-bit outputs, we need at least two of
  them to be identical. This is less likely than in the former case
  because the probability of two uniformly random \(2n\)-bit strings
  being the same is less than the probability of two uniformly random
  \(n\)-bit strings being the same. We will see this more concretely
  in the next section.
\end{answer}

\subsection{Security}
\label{sec:security}

\begin{theorem}
  The $\skc$ scheme described above is IND-CPA-secure (assuming
  $\set{f_{k}}$ is a PRF family).
\end{theorem}

\begin{proof}
  Note that it suffices to show that for either $b \in \bit$, \[
  \langle \skcenc_{k}(\cdot), \challenge_{k}^{b}(\cdot, \cdot) \rangle
  \compind \langle \algo{U}(\cdot), \algo{U}(\cdot,\cdot) \rangle, \]
  where the oracles $\algo{U}$ answer each query with a fresh
  uniformly random $(r,c') \gets U_{2n}$.  This is because the hybrid
  lemma allows us to transition from $b=0$ to $b=1$ in the IND-CPA
  definition via $\langle \algo{U}(\cdot), \algo{U}(\cdot,\cdot)
  \rangle$.

  We construct three hybrid experiments, $H_{0}$, $H_{1}$, and
  $H_{2}$, which each define the two oracles with which a adversary
  (attacking the cryptosystem) expects to interact.  Experiment
  $H_{0}$ will correspond to the IND-CPA attack on our actual
  cryptosystem (for arbitrary $b \in \bit$); experiment $H_{1}$ will
  correspond to the IND-CPA attack on our cryptosystem \emph{as if it
    were implemented with a truly random function}; and experiment
  $H_{2}$ will correspond to $\langle \algo{U}(\cdot),
  \algo{U}(\cdot,\cdot) \rangle$.  Then we will show that $H_{0}
  \compind H_{1} \statind H_{2}$, proving the theorem.

  \begin{itemize}
  \item $H_{0}$: choose $k \gets \bit^{n}$.

    Answer each query $m$ to the first (encryption) oracle by choosing
    $r \gets \bit^{n}$ and answering $(r, f_{k}(r) \oplus m)$.

    Answer the query $(m_{0}, m_{1})$ to the second (challenge) oracle
    by choosing $r \gets \bit^{n}$ and answering $(r, f_{k}(r) \oplus
    m_{b})$.

  \item $H_{1}$: this experiment ``lazily'' constructs a truly random
    function $F : \bit^{n} \to \bit^{n}$, defining its output values
    as needed.  That is, whenever the value of $F(r)$ is desired, it
    checks whether $F(r)$ has already been defined, and if so, uses
    that value.  Otherwise, it defines $F(r)$ to be a fresh uniformly
    random value in $\bit^{n}$.

    Answer each query $m$ to the first (encryption) oracle by choosing
    $r \gets \bit^{n}$, and answering $(r, F(r) \oplus m)$.

    Answer the query $(m_{0}, m_{1})$ to the second (challenge) oracle
    by choosing $r \gets \bit^{n}$ and answering $(r, F(r) \oplus
    m_{b})$.

  \item $H_{2}$: answer each query (to either oracle) by a fresh
    uniformly random value in $\bit^{2n}$.
  \end{itemize}

  The difference between $H_{0}$ and $H_{1}$ is simply that we have
  replaced $f_{k}$ with a truly uniform $F$.  Using the fact that
  $\set{f_{k}}$ is a PRF family, we would like to show that $H_{0}
  \compind H_{1}$.  This is done by constructing an (nuppt) simulator
  $\Sim^{g}$ that emulates either $H_{0}$ or $H_{1}$, depending on
  whether $g$ is drawn from $\set{f_{k}}$ or is a uniformly random
  function, respectively.  $\Sim^{g}$ works in the obvious way: it
  answers encryption queries $m$ by choosing $r \gets \bit^{n}$ and
  returning $(r, g(r) \oplus m)$, and answers the challenge query
  $(m_{0}, m_{1})$ by choosing $r \gets \bit^{n}$ and returning $(r,
  g(r) \oplus m_{b})$.

  The difference from $H_{1}$ to $H_{2}$ is that we answer every query
  by a uniformly random reply.  A cursory inspection of $H_{1}$ and
  $H_{2}$ may lead one to believe that $H_{1}$ and $H_{2}$ are
  \emph{identical}, but this is not quite the case: we must consider
  the case where $H_{1}$ happens to choose the same $r$ when answering
  two queries to its oracles.  In $H_{1}$, the value $F(r)$ used in
  the second component of the ciphertext remains the same, whereas in
  $H_{2}$, the second component of the ciphertext is still chosen
  uniformly at random.

  Fortunately, it is the case that $H_{1}$ and $H_{2}$ are identical,
  \emph{conditioned on the event that distinct queries choose distinct
    values of $r$}.  This is because in this event, both $H_{1}$ and
  $H_{2}$ produce a uniformly random and independent second ciphertext
  component for each query.  Therefore, we just need to bound the
  probability of the ``bad'' event, that some two queries choose the
  same $r$.  Because the adversary makes a total of at most $q =
  \poly(n)$ queries to its oracles, this probability is at most
  $q^{2}/2^{n} = \negl(n)$, by taking a union bound over all pairs of
  queries.  This means that there is a negligible statistical distance
  between $H_{1}$ and $H_{2}$, and $H_{1} \statind H_{2}$ as desired.

  Notice that the above argument is not just a ``technicality'' of the
  proof --- it is crucial for the actual security of the scheme!
  Suppose that in a real attack, the challenge oracle happens to
  choose the same $r$ as in one of the encryption queries.  The
  adversary can easily compute the value of $f_{k}(r)$ from that
  query, and therefore can trivially determine which of the two
  messages was encrypted by the challenge oracle!
\end{proof}

\subsection{Summary of Symmetric Cryptography}
\label{sec:summ-symm-key}

In the lectures to date, we have seen the following implications
(among others):
\[ \text{weak-OWF} \Rightarrow \text{OWF} \Rightarrow \text{PRG}
\Rightarrow \text{PRF} \Rightarrow \text{SKC}. \] Thus, a ``world'' in
which one-way functions exists is very rich, allowing us to do a lot
of cryptography (and we will see even more applications later on).
This world is often called ``minicrypt,'' because it follows from the
minimal cryptographic assumption of a (weak) OWF.  Minicrypt is fairly
well-understood, though new applications are still being found today.

Next, we will move beyond minicrypt into the brave new world of \[
\textbf{CRYPTOMANIA}, \] in which we shall observe even more
mind-boggling, seemingly paradoxical notions.  However, as we will
see, it is less clear what is the ``minimal'' assumption for
Cryptomania to exist.


\section{Asymmetric Encryption}
\label{sec:asymm-encrypt}

In the 1970s, people started asking a dangerous question: must Alice
and Bob share the \emph{same} key to perform secure encryption?  We
present here a cryptographic model called asymmetric (or
``public-key'') encryption, in which Alice and Bob may communicate
securely without a shared secret.  Specifically, Bob (and the rest of
the world) will encrypt messages to Alice using her \emph{public}
encryption key, and Alice will decrypt those messages using a related
\emph{secret} decryption key that only she knows.  (If Bob wants to
receive encrypted messages, he will generate his own pair of
asymmetric keys.)

Our formal model for a public-key cryptosystem $\pkc$ with message
space $\msgspace$ is as follows:
\begin{itemize}
\item $\pkcgen(1^{n})$ outputs a public key $\pk$ and secret key
  $\sk$.
\item $\pkcenc_{\pk}(m) := \pkcenc(\pk, m)$ for a message $m \in
  \msgspace$ outputs a ciphertext $c$.
\item $\pkcdec_{\sk} := \pkcdec(\sk, c)$ outputs a message $m \in
  \msgspace$.
\end{itemize}

The notion of completeness is as expected: for $(\pk, \sk) \gets
\pkcgen$, we ask that $\pkcdec_{sk}(\pkcenc_{pk}(m)) = m$ for all $m
\in \msgspace$.

What about security?  If we just define IND-CPA security exactly as in
the symmetric setting (giving the adversary oracle access to
$\pkcenc_{\pk}(\cdot)$ and $\challenge_{\pk}^{b}$), we miss a crucial
point: the adversary knows the public key explicitly!  (After all, it
is public information.)  Notice, then, that there is no point in
giving \emph{oracle} access to the encryption algorithm
$\pkcenc_{\pk}$, because the adversary can run the algorithm itself
(possibly even on ``bad'' randomness or messages, if it likes).  We
are therefore left with providing $\pk$ and the challenge oracle,
$\challenge_{pk}^{b}$, and a notion of indistinguishability between $b
= 0$ and $b = 1$ that is the same as in the symmetric-key setting.

\begin{definition}[Indistinguishability under chosen-plaintext attack
  for asymmetric encryption]
  \label{def:ind-cpa-pkc}
  We say that $\pkc$ is IND-CPA secure if the following are
  computationally indistinguishable for $b = 0,1$: \[ \langle pk,
  \challenge^{b}_{\pk}(m_{0}, m_{1}) := \pkcenc_{\pk}(m_{b})
  \rangle. \]
\end{definition}

To simplify the definition even further, we can restrict the message
space to $\msgspace = \bit$, yielding the following ``oracle-free''
definition:
\[ \langle \pk, \pkcenc_{\pk}(0) \rangle \compind \langle \pk,
\pkcenc_{\pk}(1) \rangle. \] As an exercise, show that a secure scheme
for single-bit messages implies an IND-CPA-secure scheme for any
$\msgspace = \bit^{\poly(n)}$.

As in the symmetric setting, note that $\pkcenc$ cannot be
deterministic if it is to be IND-CPA-secure: an adversary could simply
compare the challenge ciphertext with its own encryption of a $0$ or
$1$.  Note also that it doesn't make a lot of sense to talk about a
\emph{stateful} deterministic encryption algorithm in the public-key
setting, because many different parties can encrypt with the same
public key, and may not be able to keep joint state.

\begin{question}
  Given an IND-CPA-secure asymmetric scheme for one-bit messages,
  construct an IND-CPA-secure asymmetric scheme for two-bit
  messages. Sketch out a proof of this scheme's security. How could
  you extend this proof to show that an IND-CPA-secure asymmetric
  scheme for one-bit messages implies an IND-CPA-secure asymmetric
  scheme for \(\poly(n)\)-bit messages?
\end{question}

\begin{answer}
  Let \(\Pi = (\skcgen, \skcenc, \skcdec)\) be our IND-CPA-secure
  scheme for one-bit messages. Now, let
  \(\Pi' := (\skcgen', \skcenc', \skcdec')\) where
  \(\skcgen' := \skcgen\) and
  \(\skcenc'(b_1 \concat b_2) := \skcenc(b_1) \concat \skcenc(b_2)\).
  
  For our security proof we need to show that
  \((pk, \skcenc'(b_1 \concat b_2)) \compind (pk, \skcenc'(b_1'
  \concat b_2')\) for any bits \(b_1, b_2, b_1', b_2'\). We can do
  this through a straightforward application of the hybrid lemma. Let
  \(H_0 = (pk, \skcenc_{pk}(b_1) \concat \skcenc_{pk}(b_2))\), let
  \(H_1 = (pk, \pkcenc_{pk}(b_1') \concat \pkcenc_{pk}(b_2))\), and
  finally \(H_2 = \pkcenc_{pk}(b_1') \pkcenc_{pk}(b_2'))\). In other
  words, each of our hybrid worlds successively replaces a bit from
  the encryption of our left-world message with a bit from the
  encryption of our right-world message. The rough idea is that since
  \((pk, \pkcenc_{pk}(0)) \compind (pk, \pkcenc_{pk}(1))\), each
  hybrid world will be indistinguishable from the next, which we can
  show formally using the composition
  lemma.

  For brevity, consider just \(H_0\) and \(H_1\). Consider an
  algorithm \(S\) which on input \((pk, c)\) returns
  \((pk, c \concat \pkcenc_{pk}(b_2))\). Clearly \(S\) runs in
  polynomial time. Moreover, if the input to \(S\) comes from
  \((pk, \pkcenc_{pk}(b_1))\), then \(S\)'s output is \(H_0\) and if
  it is \((pk, \pkcenc_{pk}(b_1'))\) then it is \(H_1\). By the
  security of \(\Pi\),
  \(\pkcenc_{pk}(b_1) \compind \pkcenc_{pk}(b_1')\), so
  \(H_0 \compind H_1\) by the composition lemma.

  To adapt this proof, the exact same scheme is used -- the
  \(\poly(n)\) bit message is encrypted bitwise using our one-bit
  encryption scheme. The security proof is identical but with more
  hybrid worlds. Just as in this proof, each hybrid world would
  replace the first \(i\) bits of an encryption of \(m\) with the
  first \(i\) bits of an encryption of \(m'\). Since the message is of
  length \(\poly(n)\), we have polynomially many hybrid worlds, so we
  can apply the hybrid lemma.
\end{answer}

\subsection{Constructions}
\label{sec:constructions}

Is public-key cryptography possible?  Attempts to construct public-key
cryptosystems from one-way-functions generally fail, and there is some
theoretical explanation why (which is beyond the scope of this
course).  But we \emph{can} construct public-key cryptosystems from
seemingly stronger assumptions.

\subsubsection{From Trapdoor OWPs}
\label{sec:from-trapdoor-owps}

Informally, a \emph{trapdoor} OWP family is a family of one-way
permutations where there is additionally some ``trapdoor'' that allows
for efficient inversion.  Formally, $\set{f_{s}: D_{s} \to D_{s}}$ is a
family of trapdoor OWPs if:
\begin{itemize}
\item there exists a PPT function sampler $\algo{S}(1^{n})$ that
  outputs a function index $s$ and trapdoor $t$, and a poly-time
  inversion algorithm $\algo{F}^{-1}$ such that $\algo{F}^{-1}(t, y) =
  f_{s}^{-1}(y)$ for such $s,t$ and every $y \in D_{s}$.
\item it is a OWP family with respect to $\algo{S}$, where as usual
  the inverter is given $1^{n}$, $s$, and $f_{s}(x)$ (but $t$ is
  withheld).
\end{itemize}
Note that the family still has a hard-core predicate, because it is
one-way; the trapdoor is just an additional \emph{functional} property
that is not exposed to the adversary.

Using a trapdoor OWP family with hard-core predicate $h$, we can
construct a public-key cryptosystem $\pkc$ for $\msgspace = \bit$ as
follows:
\begin{itemize}
\item $\pkcgen$: $(s, t) \gets \algo{S}$.  Output $\pk = s$ and $\sk =
  t$.
\item $\pkcenc_{s}(m)$: choose $r \gets D_{s}$ and output $c =
  (f_{s}(r), h(r) \oplus m)$.
\item $\pkcdec_{t}(c = (y, c'))$: output $m' = h(f_{s}^{-1}(y)) \oplus
  c'$ 
\end{itemize}
Completeness of the scheme is evident by inspection (note that
$\pkcdec_{t}$ can compute $f_{s}^{-1}$ because it knows the trapdoor
$t$).  It is important that we are using a \emph{hard-core} predicate
$h(r)$ to conceal the message bit $m$, because in general $f_{s}$
might reveal certain bits of $r$ while still being one-way.

\begin{theorem}
  \label{thm:td-owp-pkc}
  The $\pkc$ scheme described above is IND-CPA-secure (assuming
  $\set{f_{s}}$ is a trapdoor OWP family with hard-core predicate
  $h$).
\end{theorem}

\begin{proof}
  We need to show that \[ \langle s, f_{s}(r), h(r) \oplus 0 \rangle
  \compind \langle s, f_{s}(r), h(r) \oplus 1 \rangle. \] By the
  hypothesis that $h$ is hard-core, we have \[ \langle s, f_{s}(r),
  h(r) \oplus 0 \rangle \compind \langle s, f_{s}(r), \algo{U}_{1}
  \rangle \equiv \langle s, f_{s}(r), \algo{U}_{1} \oplus 1 \rangle
  \compind \langle s, f_{s}(r), h(r) \oplus 1 \rangle. \]
\end{proof}

Do trapdoor OWP families exist?  Here are some plausible candidates:
\begin{enumerate}

\item Rabin's function: $f_{N} \colon \QR_{N}^{*} \to \QR_{N}^{*}$,
  defined as $f_{N}(x) = x^{2} \bmod N$, for $N$ a product of two
  primes.

\item RSA: $f_{N,e} \colon \ZN^{*} \to \ZN^{*}$, defined as $f_{N,
    e}(x) = x^{e} \bmod N$, for $N = pq$ a product of two primes and
  $e$ coprime with $\varphi(N) = (p-1)(q-1)$.  The trapdoor is $d =
  e^{-1} \bmod \varphi(N)$, which can be used to calculate $f_{N,
    e}^{-1}(y) = y^{d} \bmod N$.  This works because $x^{ed} = x^{ed
    \bmod \varphi(N)} = x \in \ZN^{*}$, by Euler's theorem and the
  fact that $\ZN^{*}$ has order $\varphi(N)$.

\item What about modular exponentiation $f_{p,g}(x) = g^{x} \bmod p$?
  It is unknown if there is any trapdoor for the discrete log problem.

\item Assorted others: Paillier's function (works mod $N^{2}$),
  ``lossy'' trapdoor functions (which are not permutations, but are
  injective, which suffices for encryption), and injective TDFs based
  on lattices.
\end{enumerate}

\subsubsection{From Diffie-Hellman}
\label{sec:from-diffie-hellman}

Even though we lack a \emph{trapdoor} for the discrete logarithm
problem, it turns out that we can still construct a public-key
cryptosystem that uses modular exponentiation.  The starting point is
the Diffie-Hellman key-exchange protocol, which can be used by Alice
and Bob to agree on a shared secret value over a public channel.

Let $G = \langle g \rangle$ be a (multiplicative) cyclic group of
known order $q$, with public generator $g$.  Alice chooses some $a
\gets \Zq$ and Bob chooses some $b \gets \Zq$.  Alice sends $\alpha =
g^{a}$ and Bob sends $\beta = g^{b}$.  They then can agree on a shared
value $g^{ab}$: Alice by computing $\beta^{a}$, and Bob by computing
$\alpha^{b}$.  It is conjectured that in certain groups, it is
computationally infeasible for any algorithm to compute $g^{ab}$ given
only $g$, $\alpha = g^{a}$, and $\beta = g^{b}$.  Moreover, it is
believed that the shared value $g^{ab}$ ``looks random,'' even given
those other values.

\begin{conjecture}[DDH Assumption]
  The Decision Diffie-Hellman assumption for a group $G = \langle g
  \rangle$ of order $q$ says that
  \[ (g, g^{a}, g^{b}, g^{ab}) \compind (g, g^{a}, g^{b}, g^{c}), \]
  where $a, b, c \gets \Zq$ are uniformly random and
  independent.\footnote{To be completely formal, to give a meaningful
    asymptotic definition using computational indistinguishability, we
    should define the DDH assumption over an \emph{infinite family} of
    groups $G$.}
\end{conjecture}

For a prime $p$, is DDH true or false for $\Zp^{*}$?  It is actually
\emph{false}!  Think about the probability that $a \cdot b$ is even,
versus the probability that $c$ is even, and recall that in $\Zp^{*}$
we can test whether $z$ is even, given $g^{z}$.  We can avoid this
annoying property by working in the group of \emph{quadratic residues}
$G = \QRp^{*}$, where $p=2q+1$ and $q$ itself is prime.  This group
has order $(p-1)/2 = q$, and DDH is believed to hold in it (and some
others, such as those defined over elliptic curves).

By simply ``packaging'' the Diffie-Hellman protocol in the appropriate
way, we obtain the \emph{ElGamal} public-key cryptosystem over the
group $G$, with message space $\msgspace = G$.
\begin{itemize}
\item $\pkcgen$: choose secret key $\sk = a \gets \Zq$, and let $\pk =
  \alpha = g^{a}$.
\item $\pkcenc_{\alpha}(m)$: choose $b \gets \Zq$, and output $c =
  (g^{b}, \alpha^{b} \cdot m)$.
\item $\pkcdec_{a}(c = (\beta, c'))$: output $c' / \beta^{a}$.
\end{itemize}
Correctness is by inspection.  Security follows directly from the DDH
assumption: it is a simple exercise to show that the view of the
adversary is computationally indistinguishable from $(g, g^{a}, g^{b},
g^{c})$ (where $a, b, c \gets \Zq$), no matter which message is
encrypted by the challenger.

\end{document}

%%% Local Variables:
%%% mode: latex
%%% TeX-master: t
%%% End:
