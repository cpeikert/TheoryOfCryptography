\documentclass[11pt]{article}

\usepackage{fullpage,microtype,pdfsync,hyperref}
\usepackage{newtxtext}
\usepackage[T1]{fontenc}

\usepackage{mathtools,amsfonts,amssymb}
\usepackage[amsmath,amsthm,thmmarks,hyperref]{ntheorem}
\usepackage[capitalize,nameinlink,noabbrev]{cleveref}

\RequirePackage{mathtools}

%%% BLACKBOARD SYMBOLS

% hyperref package defines C and G, undefined to avoid conflicts
\let\C\relax
\let\G\relax
\newcommand{\C}{\ensuremath{\mathbb{C}}}
\newcommand{\D}{\ensuremath{\mathbb{D}}}
\newcommand{\F}{\ensuremath{\mathbb{F}}}
\newcommand{\G}{\ensuremath{\mathbb{G}}}
\newcommand{\J}{\ensuremath{\mathbb{J}}}
\newcommand{\N}{\ensuremath{\mathbb{N}}}
\newcommand{\Q}{\ensuremath{\mathbb{Q}}}
\newcommand{\R}{\ensuremath{\mathbb{R}}}
\newcommand{\T}{\ensuremath{\mathbb{T}}}
\newcommand{\Z}{\ensuremath{\mathbb{Z}}}
\newcommand{\QR}{\ensuremath{\mathbb{QR}}}

\newcommand{\Zt}{\ensuremath{\Z_t}}
\newcommand{\Zp}{\ensuremath{\Z_p}}
\newcommand{\Zq}{\ensuremath{\Z_q}}
\newcommand{\ZN}{\ensuremath{\Z_N}}
\newcommand{\Zps}{\ensuremath{\Z_p^*}}
\newcommand{\ZNs}{\ensuremath{\Z_N^*}}
\newcommand{\JN}{\ensuremath{\J_N}}
\newcommand{\QRN}{\ensuremath{\QR_{N}}}
\newcommand{\QRp}{\ensuremath{\QR_{p}}}

%%% THEOREM COMMANDS

\RequirePackage{amsthm}

\theoremstyle{plain}            % following are "theorem" style
\newtheorem{theorem}{Theorem}[section]
\newtheorem{lemma}[theorem]{Lemma}
\newtheorem{corollary}[theorem]{Corollary}
\newtheorem{proposition}[theorem]{Proposition}
\newtheorem{claim}[theorem]{Claim}
\newtheorem{fact}[theorem]{Fact}

\theoremstyle{definition}       % following are def style
\newtheorem{definition}[theorem]{Definition}
\newtheorem{conjecture}[theorem]{Conjecture}
\newtheorem{example}[theorem]{Example}
\newtheorem{protocol}[theorem]{Protocol}

\theoremstyle{remark}           % following are remark style
\newtheorem{remark}[theorem]{Remark}
\newtheorem{note}[theorem]{Note}

% equation numbering style
\numberwithin{equation}{section}

%%% GENERAL COMPUTING

\newcommand{\bit}{\ensuremath{\set{0,1}}}
\newcommand{\pmone}{\ensuremath{\set{-1,1}}}

% asymptotics
\DeclareMathOperator{\poly}{poly}
\DeclareMathOperator{\polylog}{polylog}
\DeclareMathOperator{\negl}{negl}
\newcommand{\Otil}{\ensuremath{\tilde{O}}}

% probability/distribution stuff
\DeclareMathOperator*{\E}{E}
\DeclareMathOperator*{\Var}{Var}

% sets in calligraphic type
\newcommand{\calD}{\ensuremath{\mathcal{D}}}
\newcommand{\calF}{\ensuremath{\mathcal{F}}}
\newcommand{\calG}{\ensuremath{\mathcal{G}}}
\newcommand{\calH}{\ensuremath{\mathcal{H}}}
\newcommand{\calX}{\ensuremath{\mathcal{X}}}
\newcommand{\calY}{\ensuremath{\mathcal{Y}}}

% types of indistinguishability
\newcommand{\compind}{\ensuremath{\stackrel{c}{\approx}}}
\newcommand{\statind}{\ensuremath{\stackrel{s}{\approx}}}
\newcommand{\perfind}{\ensuremath{\equiv}}

% font for general-purpose algorithms
\newcommand{\algo}[1]{\ensuremath{\mathsf{#1}}}
% font for general-purpose computational problems
\newcommand{\problem}[1]{\ensuremath{\mathsf{#1}}}
% font for complexity classes
\newcommand{\class}[1]{\ensuremath{\mathsf{#1}}}

% complexity classes and languages
\renewcommand{\P}{\class{P}}
\newcommand{\BPP}{\class{BPP}}
\newcommand{\NP}{\class{NP}}
\newcommand{\coNP}{\class{coNP}}
\newcommand{\AM}{\class{AM}}
\newcommand{\coAM}{\class{coAM}}
\newcommand{\IP}{\class{IP}}

%%% "LEFT-RIGHT" PAIRS OF SYMBOLS


% inner product
\DeclarePairedDelimiter\inner{\langle}{\rangle}
% absolute value
\DeclarePairedDelimiter\abs{\lvert}{\rvert}
% a set
\DeclarePairedDelimiter\set{\{}{\}}
% parens
\DeclarePairedDelimiter\parens{(}{)}
% tuple, alias for parens
\DeclarePairedDelimiter\tuple{(}{)}
% square brackets
\DeclarePairedDelimiter\bracks{[}{]}
% rounding off
\DeclarePairedDelimiter\round{\lfloor}{\rceil}
% floor function
\DeclarePairedDelimiter\floor{\lfloor}{\rfloor}
% ceiling function
\DeclarePairedDelimiter\ceil{\lceil}{\rceil}
% length of some vector, element
\DeclarePairedDelimiter\length{\lVert}{\rVert}
% "lifting" of a residue class
\DeclarePairedDelimiter\lift{\llbracket}{\rrbracket}
\DeclarePairedDelimiter\len{\lvert}{\rvert}

%%% CRYPTO-RELATED NOTATION

% KEYS AND RELATED

\newcommand{\key}[1]{\ensuremath{#1}}

\newcommand{\pk}{\key{pk}}
\newcommand{\vk}{\key{vk}}
\newcommand{\sk}{\key{sk}}
\newcommand{\mpk}{\key{mpk}}
\newcommand{\msk}{\key{msk}}
\newcommand{\fk}{\key{fk}}
\newcommand{\id}{id}
\newcommand{\keyspace}{\ensuremath{\mathcal{K}}}
\newcommand{\msgspace}{\ensuremath{\mathcal{M}}}
\newcommand{\ctspace}{\ensuremath{\mathcal{C}}}
\newcommand{\tagspace}{\ensuremath{\mathcal{T}}}
\newcommand{\idspace}{\ensuremath{\mathcal{ID}}}

\newcommand{\concat}{\ensuremath{\|}}

% GAMES

% advantage
\newcommand{\advan}{\ensuremath{\mathbf{Adv}}}

% different attack models
\newcommand{\attack}[1]{\ensuremath{\text{#1}}}

\newcommand{\atk}{\attack{atk}} % dummy attack
\newcommand{\indcpa}{\attack{ind-cpa}}
\newcommand{\indcca}{\attack{ind-cca}}
\newcommand{\anocpa}{\attack{ano-cpa}} % anonymous
\newcommand{\anocca}{\attack{ano-cca}}
\newcommand{\euacma}{\attack{eu-acma}} % forgery: adaptive chosen-message
\newcommand{\euscma}{\attack{eu-scma}} % forgery: static chosen-message
\newcommand{\suacma}{\attack{su-acma}} % strongly unforgeable

% ADVERSARIES
\newcommand{\attacker}[1]{\ensuremath{\mathcal{#1}}}

\newcommand{\Adv}{\attacker{A}}
\newcommand{\AdvA}{\attacker{A}}
\newcommand{\AdvB}{\attacker{B}}
\newcommand{\Dist}{\attacker{D}}
\newcommand{\Sim}{\attacker{S}}
\newcommand{\Ora}{\attacker{O}}
\newcommand{\Inv}{\attacker{I}}
\newcommand{\For}{\attacker{F}}

% CRYPTO SCHEMES

\newcommand{\scheme}[1]{\ensuremath{\text{#1}}}

% pseudorandom stuff
\newcommand{\prg}{\algo{PRG}}
\newcommand{\prf}{\algo{PRF}}
\newcommand{\prp}{\algo{PRP}}

% symmetric-key cryptosystem
\newcommand{\skc}{\scheme{SKC}}
\newcommand{\skcgen}{\algo{Gen}}
\newcommand{\skcenc}{\algo{Enc}}
\newcommand{\skcdec}{\algo{Dec}}

% public-key cryptosystem
\newcommand{\pkc}{\scheme{PKC}}
\newcommand{\pkcgen}{\algo{Gen}}
\newcommand{\pkcenc}{\algo{Enc}} % can also use \kemenc and \kemdec
\newcommand{\pkcdec}{\algo{Dec}}

% digital signatures
\newcommand{\sig}{\scheme{SIG}}
\newcommand{\siggen}{\algo{Gen}}
\newcommand{\sigsign}{\algo{Sign}}
\newcommand{\sigver}{\algo{Ver}}

% message authentication code
\newcommand{\mac}{\scheme{MAC}}
\newcommand{\macgen}{\algo{Gen}}
\newcommand{\mactag}{\algo{Tag}}
\newcommand{\macver}{\algo{Ver}}

% key-encapsulation mechanism
\newcommand{\kem}{\scheme{KEM}}
\newcommand{\kemgen}{\algo{Gen}}
\newcommand{\kemenc}{\algo{Encaps}}
\newcommand{\kemdec}{\algo{Decaps}}

% identity-based encryption
\newcommand{\ibe}{\scheme{IBE}}
\newcommand{\ibesetup}{\algo{Setup}}
\newcommand{\ibeext}{\algo{Ext}}
\newcommand{\ibeenc}{\algo{Enc}}
\newcommand{\ibedec}{\algo{Dec}}

% hierarchical IBE (as key encapsulation)
\newcommand{\hibe}{\scheme{HIBE}}
\newcommand{\hibesetup}{\algo{Setup}}
\newcommand{\hibeext}{\algo{Extract}}
\newcommand{\hibeenc}{\algo{Encaps}}
\newcommand{\hibedec}{\algo{Decaps}}

% binary tree encryption (as key encapsulation)
\newcommand{\bte}{\scheme{BTE}}
\newcommand{\btesetup}{\algo{Setup}}
\newcommand{\bteext}{\algo{Extract}}
\newcommand{\bteenc}{\algo{Encaps}}
\newcommand{\btedec}{\algo{Decaps}}

% trapdoor functions
\newcommand{\tdf}{\scheme{TDF}}
\newcommand{\tdfgen}{\algo{Gen}}
\newcommand{\tdfeval}{\algo{Eval}}
\newcommand{\tdfinv}{\algo{Invert}}
\newcommand{\tdfver}{\algo{Ver}}

%%% PROTOCOLS

\newcommand{\out}{\text{out}}
\newcommand{\view}{\text{view}}

%%% COMMANDS FOR LECTURES/HOMEWORKS

\RequirePackage{fancyhdr}

\newcommand{\lecheader}{%
  \chead{\large \textbf{Lecture \lecturenum\\\lecturetopic}}

  \lhead{\small \textbf{Theory of Cryptography}\\}

  \rhead{\small \textbf{Instructor:
      \href{http://www.eecs.umich.edu/~cpeikert/}{Chris Peikert}\\Scribe:
      \scribename}}

  \setlength{\headheight}{20pt}
  \setlength{\headsep}{16pt}
}


%%% xsim settings

\RequirePackage{xsim}
\RequirePackage{needspace}

\DeclareExerciseEnvironmentTemplate{QRef}
{%
    \par\vspace{\baselineskip}
    \Needspace*{2\baselineskip}
    \noindent
    \hyperref[sol:\ExerciseID]{\textbf{\XSIMmixedcase{\GetExerciseName}~\GetExerciseProperty{counter}.}}
    \label{ques:\ExerciseID}
}{}

\DeclareExerciseEnvironmentTemplate{ARef}{%
        \Needspace*{2\baselineskip}
        \noindent
        \hyperref[ques:\ExerciseID]{\textbf{\XSIMmixedcase{\GetExerciseParameter{exercise-name}}~\GetExerciseProperty{counter}.}}
        \label{sol:\ExerciseID}
        \GetExerciseBody{exercise}
        \newline
        \textbf{\XSIMmixedcase{\GetExerciseName}.}
}{\par\vspace{\baselineskip}}

\DeclareExerciseHeadingTemplate{simple}{%
    \section*{\XSIMmixedcase{\GetExerciseParameter{solution-name}s}}
}

\DeclareExerciseType{question}{%
    exercise-env = question,
    solution-env = answer,
    exercise-name = Question,
    solution-name = Answer,
    exercise-template = QRef,
    solution-template = ARef,
}

\xsimsetup{%
  path = {questions},
  file-extension = {aux},
  print-solutions/headings-template=simple
}

\AtEndDocument{\pagebreak\printsolutions}


% VARIABLES

\newcommand{\lecturenum}{11}
\newcommand{\lecturetopic}{Message Authentication}
\newcommand{\scribename}{Daniel Dadush}

\def\tg{\mathrm{tag}}
\def\msg{\mathrm{msg}}
\newcommand{\newl}
{\vspace{1em}

\noindent}

% END OF VARIABLES

\lecheader

\pagestyle{plain}               % default: no special header

\begin{document}

\thispagestyle{fancy}           % first page should have special header

% LECTURE MATERIAL STARTS HERE

\section{Message Authentication}
\label{sec:mess-auth}

We start with some simple questions from real life:
\begin{itemize}
\item How do you know an email is from who it claims to be from?  (Do
  you?)
\item How does a soldier know his orders come from his commander?
\item How does a bartender know you're 21?
\end{itemize}
All of these are questions of \emph{message authentication}:
determining whether the contents of a message (an email, a soldier's
orders, an identification card) came from their supposed source,
without being modified (or generated outright) by a malicious
interloper.

A first natural question is: does encryption solve the problem of
message authentication?

Here is a proposal for the setting where Alice and Bob share a secret
key: to authenticate a message, Alice simply encrypts the message $m$
under a secure shared-key scheme.  For example, using the one-time
pad, Alice sends: \[ c \gets \underbrace{m}_{\text{message}} \oplus
\underbrace{k}_{\text{key}}. \] Bob decrypts the message and decides
whether it ``looks good'' (e.g., whether it is properly formatted and
readable).

A moment's thought reveals the following property: \emph{An adversary
  can flip any desired bit(s) of the message, without even knowing
  what the message is!}  If the adversary knows anything about the
format of the message, or its likely contents in certain places (which
is the case in most applications), then it can change the message
without causing any suspicion on the part of Bob.

Hence, although the one-time pad is \emph{perfectly secret}, it is in
a sense also \emph{perfectly inauthentic}.  The moral of this exercise
is: Encryption is not \emph{sufficient} for authentication.  Later we
will see that neither is it \emph{necessary}.  We need an entirely new
model and security definition.

\subsection{Model}
\label{sec:model}

We want the sender to be able to attach a ``tag'' or ``signature'' to
every message, which can be checked for validity by the receiver.  In
the shared-key setting, a \emph{message authentication code} $\mac$
for message space $\msgspace$ is made up of algorithms $\macgen$,
$\mactag$, $\macver$:
\begin{itemize}
\item $\macgen$ outputs a key $k$.
\item $\mactag_k(m) := \mactag(k,m)$ outputs a tag $t \in \tagspace$
  (the ``tag space'').
\item $\macver_k(m,t) := \macver(k,m,t)$ either accepts or rejects
  (i.e., outputs $1$ or $0$, respectively).
\end{itemize}

For completeness, we require that for any $m \in \msgspace$, for $k
\gets \macgen$ and $t \gets \mactag_{k}(m)$, we have
$\macver_{k}(m,t)=1$.

\subsection{Information-Theoretic Treatment}
\label{sec:inform-theor}

The following definition is the analogue of perfect secrecy for
authenticity, which captures the intuition that given a valid
message-tag pair, no forger should be able to find a convincing tag
for a \emph{different} message.

\begin{definition}[Perfect unforgeability]
  \label{def:perfect-unforg}
  A $\mac$ scheme is \emph{perfectly unforgeable} if, for all
  (possibly unbounded) $\For$ and all $m \in \msgspace$,
  \[ \advan_{\mac}(\For) := \Pr_{k \gets \macgen, t \gets
    \mactag(m)}[\For(m, t) \text{ succeeds}] \leq
  \frac{1}{\abs{\tagspace}}, \] where ``succeeds'' means that $\For$
  outputs some $(m',t') \in \msgspace \times \tagspace$ such that
  $\macver_{k}(m',t') = 1$ and $m \neq m'$.

  If the above holds for the relaxed success condition requiring only
  that $(m',t') \neq (m,t)$ (i.e., $\For$ even succeeds if it outputs
  a \emph{different} tag for the \emph{same} message), then the scheme
  is said to be \emph{strongly} (and perfectly) unforgeable.
\end{definition}

\noindent Some remarks and observations about the definition:
\begin{itemize}
\item For any message $m'$, the forger $\For$ can always just guess a
  tag $t'$ uniformly at random.  A valid tag (that makes
  $\macver_{k}(m',t')$ accept) must exist by correctness, so
  $1/\abs{\tagspace}$ is the strongest bound on the forger's advantage
  that we can hope for.
\item Unlike with encryption, the $\mactag_{k}$ algorithm can be
  \emph{deterministic} (and all the examples we see today will be).
\item If $\macver_{k}$ accepts a \emph{unique} tag for any given
  message, then the $\mac$ is perfectly unforgeable if and only if it
  is \emph{strongly} perfectly unforgeable.
\end{itemize}

\subsubsection{Construction}
\label{sec:construction-info-mac}

We'll construct a (strongly) perfectly unforgeable $\mac$ using a
special kind of hash function family.

\begin{definition}
  \label{def:pairwise-ind-hash}
  A family of functions $\calH = \set{h_{k} \colon \msgspace \to
    \tagspace}$ is \emph{pairwise independent} if for any
  \emph{distinct} $m,m' \in \msgspace$, the random variable $(h(m),
  h(m'))$ is uniform over $\tagspace^{2}$, for a uniformly random $h
  \gets \calH$.
\end{definition}

\begin{example} 
  \label{ex:pairwise-ind}
  Let $p$ be a prime.  Define $h_{a,b}(x) = ax + b \bmod p$ for $(a,b)
  \in \Zp^2$.  We let $\msgspace = \tagspace = \Zp$.  We check that
  the family $\set{h_{(a,b)}}$ is pairwise independent: for any
  \emph{distinct} $m,m' \in \Zp$, and any $t,t' \in \Zp$,
  \begin{align*}
    \Pr_{(a,b) \gets \Zp^2}\left[h_{a,b}(m) = t \wedge h_{a,b}(m') =
      t'\right] =
    \Pr_{(a,b) \gets \Zp^2}\left[am+b = t \wedge am'+b = t'\right] = \\
    \Pr_{(a,b) \gets \Zp^2}\left[
      \begin{pmatrix} m & 1 \\ m ' & 1 \end{pmatrix}
      \begin{pmatrix} a \\ b \end{pmatrix} =
      \begin{pmatrix} t \\ t' \end{pmatrix} \right] = \Pr_{(a,b) \gets
      \Zp^2}\left[ \begin{pmatrix} a \\ b \end{pmatrix} =
      \begin{pmatrix} m & 1 \\ m ' & 1 \end{pmatrix}^{-1}
      \begin{pmatrix} t \\ t' \end{pmatrix} \right] = \frac{1}{p^2}.
  \end{align*}
  Here we note that the matrix $\left( \begin{smallmatrix} m & 1 \\ m
      ' & 1 \end{smallmatrix} \right)$ is invertible modulo $p$ by our
  assumption that $m \neq m'$.  Our family is therefore pairwise
  independent, as claimed.
\end{example}

From \cref{def:pairwise-ind-hash}, it is immediate to see
that for any distinct $m,m' \in \msgspace$ and any $t,t' \in
\tagspace$, \[ \Pr_{h \gets \calH}\left[h(m')=t' \mid h(m)=t\right] =
\frac{1}{\abs{\tagspace}}. \]

\noindent
Our $\mac$ construction is as follows: let $\calH = \set{h_k \colon
  \msgspace \to \tagspace}$ be a pairwise independent hash family.
\begin{itemize}
\item $\macgen$ outputs $h_{k} \gets \calH$.
\item $\mactag_k(m)$ outputs $h_{k}(m) \in \tagspace$.
\item $\macver_k(m, t)$ accepts if $t = h_k(m)$, and rejects otherwise.
\end{itemize}

\begin{theorem}
  \label{thm:pwise-unforgeable}
  The $\mac$ described above is strongly, perfectly unforgeable.
\end{theorem}

\begin{proof}
  Since $\macver_{k}$ accepts at most one tag for a given message, it
  suffices to prove that $\mac$ is perfectly unforgeable.  Now for any
  $m \in \msgspace$, we see that
  \[
  \advan_{\mac}(\For) := \Pr_{k \gets \macgen, t \gets
    \mactag(m)}[\For(m, t) \text{ succeeds}] = \Pr_{k \gets
    \macgen}[\For(m, h_k(m)) = (m', h_k(m')) \wedge m' \neq m].
  \]
  Now breaking up the probability space over the forger's choice of
  $m'$ and $t'$, the above expression becomes
  \begin{align*}
    \sum_{\substack{(m',t') \in \msgspace \times \tagspace \\ m' \neq
        m}} &\Pr_{k \gets \macgen}[\For(m, t) = (m',t') \wedge
    h_{k}(m')=t' \mid h_{k}(m) = t] \\
    & = \sum_{\substack{(m',t') \in \msgspace \times \tagspace \\
        m' \neq m}} \Pr[\For(m, t) = (m',t')] \cdot \Pr_{k \gets
      \macgen}[h_k(m') = t' \mid h_{k}(m) = t] \\
    & = \sum_{\substack{(m',t') \in \msgspace \times \tagspace \\
        m' \neq m}} \Pr[\For(m, t) = (m',t')] \cdot
    \frac{1}{|\tagspace|},
  \end{align*}
  where the first equality is because $\For$'s random coins are
  independent of the choice of $k \gets \macgen$, and the second
  inequality is because $\calH$ is pairwise independent.  We can
  immediately upper bound the final expression above by
  $1/\abs{\tagspace}$, as desired.
\end{proof}

\begin{question}[ID=one-time]
  Although perfectly unforgeable, the above scheme is only
  one-time. Suppose \(\mathcal{H}\) is the concrete hash family
  described in \cref{ex:pairwise-ind} and show how an attacker
  with access to two message-tag pairs generated from the same key can
  forge a new message-tag pair.
\end{question}

\begin{answer}
  Let \((m_1, t_1)\) and \((m_2, t_2)\) be our two pairs. As it turns
  out, these two pairs are enough to recover the secret key \((a,b)\),
  and from there we can forge any message we want. By the construction
  of our scheme we have that \(t_i = h_{a,b}(m_i) = am_i + b \pmod p\)
  for \(i = 1,2\). Here we have a system of two equations with two
  unknowns and as such we can solve with basic linear algebra.
  \begin{align*}
    \begin{pmatrix} m_1 & 1 \\ m_2 & 1 \end{pmatrix} \begin{pmatrix} a \\ b
    \end{pmatrix} &= \begin{pmatrix} t_1 \\ t_2 \end{pmatrix} \pmod {p} \\
    \begin{pmatrix} a \\ b \end{pmatrix} &= (m_1 - m_2)^{-1} \begin{pmatrix} 1 &
      -1 \\ -m_2 & m_1 \end{pmatrix} \begin{pmatrix} t_1 \\ t_2 \end{pmatrix} \pmod{p} \\ 
    a &= (m_1 - m_2)^{-1} (t_1 - t_2) \pmod {p} \\
    b &= (m_1 - m_2)^{-1} (m_1t_2 - t_1m_2) \pmod{p}
  \end{align*}
\end{answer}
  
\begin{question}
  Just as a pairwise-independent hash function allowed us to construct
  a one-time perfectly unforgeable MAC, a \(d\)-wise independent hash
  family can be used to form a \((d-1)\)-time perfectly unforgeable
  MAC in precisely the same way (the security proof is essentially the
  same). Using \cref{ex:pairwise-ind} as a template, describe
  how one could construct such a family.
\end{question}

\begin{answer}
  The key point to recognize is that the pairwise-independence of
  \(h_{a,b} = ax + b \pmod{p}\) comes from the fact that a linear
  function has two degrees of freedom.  Supplying one pair \((m,t)\)
  such that \(h_{a,b}(m) = t\) fixes a relationship between \(a\) and
  \(b\) but still leaves \(p\) possible combinations of what \(a\) and
  \(b\) could be. To go from pairwise independence to \(d\)-wise
  independence, we simply increase the degrees of freedom by using
  polynomials of degree \(d-1\).  Concretely, we define our family
  \(\mathcal{H} = \set{h_k}\) such that
  \[h_{c_0, \ldots, c_{d-1}}(x) = c_0 + c_1x + \cdots + c_{d-1}x^{d-1}
    = \sum_{i=0}^{d-1} c_i x^i.\]
\end{answer}

\subsubsection{Critiques}
\label{sec:critiques-info-mac}

As noted in \cref{ques:one-time}, the scheme is only one-time!
This problem is a deficiency in our \emph{definition} of
unforgeability, because it does not comport with the fact that we
would like to use a $\mac$ scheme to tag multiple messages.  However,
it can be shown that perfect (information-theoretic) unforgeability is
impossible to achieve, if the forger gets to see enough message-tag
pairs relative to the (fixed) secret key size.  So to have any hope of
achieving security, we need to consider a computational definition
instead.

\subsection{Computational Treatment}
\label{sec:comp-mac}

We want to be conservative and capture the fact that the adversary
(forger) can choose to see tags on \emph{arbitrary} messages of its
choice, but still cannot forge a tag for another message.  As usual,
we do so by giving the adversary oracle access to the scheme.

\begin{definition}[Unforgeability under Chosen-Message Attack]
  \label{def:uf-mca}
  We say that $\mac$ is UF-CMA if for all nuppt $\For$,
  \[
  \advan_{\mac}(\For) := \Pr_{k \gets \macgen}\left[\For^{\mactag_k}
  \text{ succeeds} \right] \leq \negl(n),
  \]
  where here ``succeeds'' means that $\For$ outputs $(m',t')$ such that
  \begin{enumerate}
  \item $\macver_k(m',t') = 1$, and
  \item 
    \begin{enumerate}
    \item \emph{Standard unforgeability:} $m'$ was not a query to the
      $\mactag_{k}$ oracle, or
    \item \emph{Strong unforgeability:} $(m',t')$ was not a query /
      response pair from the $\mactag_{k}$ oracle.
    \end{enumerate}
  \end{enumerate}
\end{definition}

\paragraph{Construction for fixed-length messages.}

Here we use a PRF in place of pairwise independent hash family.
Intuitively, this should work since PRFs ``appear'' uniform and
independent on any polynomial number of (adaptive) queries.  Letting
$\msgspace = \tagspace = \bit^n$, and $\set{f_k \colon \msgspace \to
  \tagspace}$ be a PRF family, we construct the $\mac = (\macgen,
\mactag, \macver)$ identically to the previous construction.

\begin{theorem} 
  The above $\mac$ is strongly unforgeable under chosen message
  attack.
\end{theorem}

\begin{proof}
  We shall show that attacking $\mac$ is at least as hard as breaking
  the PRF family (i.e., distinguishing a function chosen at random
  from the family from a truly uniform, random function).  Let $\For$
  be a candidate nuppt forger against $\mac$.  We use $\For$ to
  construct an (oracle) distinguisher $\Dist^{\cdot}$ for the PRF game
  as follows.  Notice that $\Dist$ needs to ``simulate'' the
  chosen-message attack for $\For$, by providing a $\mactag_{k}$
  oracle; it will do so using its own oracle.
  
  $\Dist$ is given oracle access to a function $g$, where $g \gets
  \set{f_k}$ or $g \gets U(\bit^n \to \bit^n)$, as in the PRF
  definition.  $\Dist^{g}$ runs $\For$, and whenever $\For$ queries a
  message $m$ to be tagged, $\Dist$ queries $t = g(m)$ and returns $t$
  to $\For$, also storing $m$ in an internal list of queries it
  maintains.  Finally, $\For$ outputs a candidate forgery $(m',t')$.
  If $m'$ is different from all of the queries so far and $t' =
  g(m')$, then $\Dist$ returns $1$, otherwise it returns $0$.

  We observe that $D$ is clearly nuppt.  We also remark that $\Dist$
  only outputs $1$ when $\For$ outputs a tag for a message $m$ that
  was not previously queried.  Now we want to know the advantage of
  $\Dist$ against $\set{f_k}$:
  \[
  \advan_{\prf}(\Dist) = \left|\Pr_{g \gets \set{f_k}}[D^g = 1] -
    \Pr_{g \gets U(\bit^n \to \bit^n)}[D^g = 1]\right|.
  \]
  \begin{enumerate}
  \item When $g \gets \set{f_k}$, we see that $D^g$ emulates the
    chosen-message attack to $\For$, and accepts exactly when $\For$
    succeeds according to the definition.  Hence $\Pr[D^g = 1] =
    \advan_{\mac}(\For)$.
  \item When $g \gets U(\bit^n \to \bit^n)$, we claim that $\Pr[D^g =
    1] \leq 2^{-n}$.  We note that when $\For$ returns a message $m'$
    different from its queries $m_1,\dots,m_q$, the value $g(m)$ is
    still uniform on $\bit^n$ conditioned on the values of
    $g(m_1),\dots,g(m_k)$, because $g$ is a truly random function.
    Therefore, $\For$ can succeed at guessing $g(m')$ with probability
    at most $2^{-n}$, as claimed.
  \end{enumerate}
  We conclude that $\Dist$ has advantage at least $\advan_{\mac}(\For)
  - 2^{-n}$ against $\set{f_k}$.  Since $\set{f_k}$ is a PRF family by
  assumption, we must have that $\advan_{\mac}(\For) - 2^{-n} \in
  \negl(n) \Rightarrow \advan_{\mac}(\For) \in \negl(n)$ as needed.
\end{proof}

\begin{question}
  Unfortunately, the above scheme only works for fixed-length
  \(n\)-bit messages where \(n\) is the security parameter. The
  following is an attempt to use such a MAC and construct a new MAC
  (\(\mac'\)) that works for arbitrary length messages. Show that it
  is not UF-CMA secure.
  \begin{itemize}
  \item \(\macgen' := \macgen\)
  \item \(\mactag'\) takes an arbitrary-length message \(m\) (whose
    length we denote \(N\)), pads its length to be a multiple of
    \(n\), divides \(m\) into \(n\)-bit blocks
    \((m_1, \ldots, m_{N/n})\) and outputs the tag
    \(t := \mactag(m_1) \concat \mactag(m_2) \concat \ldots \concat
    \mactag(m_{N/n})\).
  \item \(\macver'\) is defined in the natural way.
  \end{itemize}
  \leavevmode
\end{question}

\begin{answer}
  This new MAC is subject to several different attacks. One attack
  would be for an attacker to simply permute the blocks of a
  particular message. That is, they could query \(m_1 \concat m_2\),
  receiving the tag \(t_1 \concat t_2\) and then return the
  new pair \((m_2 \concat m_1, t_2 \concat t_1)\). \\

  \noindent Another options would be a length-extension or reduction
  attack whereby the attacker queries \(m_1 \concat m_2\), receives
  the tag \(t_1 \concat t_2\), and either shortens the message --
  returning the pair \((m_1, t_1)\) -- or lengthens it -- returning
  the pair \((m_1 \concat m_1 \concat m_2, t_1
  \concat t_1 \concat t_2)\). \\

  \noindent One final attack would be to combine the blocks from two
  messages. That is, the attacker could query \(m_1 \concat m_2\),
  receive the tag \(t_1 \concat t_2\), query \(m_1' \concat m_2'\),
  recieve the tag \(t_1' \concat t_2'\) and then return the pair
  \((m_1 \concat m_1', t_1 \concat t_1')\)
\end{answer}

\section{Authenticated Encryption}
\label{sec:auth-encrypt}

Often Alice and Bob would want both secrecy and authenticity together
in one package; this is more than just the sum of the two properties.

Our model is as follows: an authenticated encryption scheme
$\scheme{AE}$ is made up of $\skcgen$, $\skcenc$, $\skcdec$ as usual.
But $\skcdec$ outputs an element of $\msgspace \cup \set{\bot}$, where
$\bot$ is a distinguished symbol indicating ``inauthentic
ciphertext,'' and any other output message implicitly means that the
ciphertext was judged to be authentic.

\begin{definition}
  $\scheme{AE}$ is a secure authenticated encryption scheme if it is:
  \begin{enumerate}
  \item IND-CPA-secure as an encryption scheme, and
  \item Strongly unforgeable as a $\mac$.  That is, for all nuppt
    $\For$,
    \[
    \Pr_{k \gets \skcgen}[\For^{\skcenc_k(\cdot)} \text{ forges}] \leq \negl(n),
    \]
    where ``forges'' means that $\For$ outputs some $c'$ where
    $\skcdec_k(c') \neq \bot$, and $c'$ is different from every
    response of the $\skcenc_{k}$ oracle.  (That is, the only way to
    obtain a valid ciphertext is to get it from the encryption
    oracle.)
  \end{enumerate}
\end{definition}

\paragraph{Candidate constructions.}

Here we give some attempts to combine an IND-CPA secure $\skc$ with a
strongly unforgeable $\mac$ to construct $\scheme{AE}$.  In all cases,
we make sure to use \emph{independent} keys for both $\skc$ and
$\mac$, which will be important in any potential security proof.  Take
$m \in \msgspace$, and define $\scheme{AE}.\skcgen$ to choose $k_a
\gets \mac.\macgen$ and $k_e \gets \scheme{SKC}.\skcgen$, and output
key $(k_{a}, k_{e})$.  Now consider the following encryption
algorithms $\scheme{AE}.\skcenc_{(k_{a},k_{e})}(m)$:

\begin{enumerate}
\item ``Encrypt and tag:'' output $(c \gets \skc.\skcenc_{k_e}(m), t
  \gets \mac.\mactag_{k_a}(m))$.  This scheme has a problem: since
  $\mactag$ need not have any secrecy properties, its output might
  leak some of the plaintext $m$.  (For example, it is easy to
  construct a ``pathological'' secure $\mactag$ that includes the
  first few bits of $m$ in its output tag.)  Therefore, this scheme
  will not necessarily be IND-CPA secure.
  
\item ``Tag then encrypt:'' output $c \gets \skc.\skcenc_{k_e}(m
  \concat \mac.\mactag_{k_a}(m))$.  How should $\skcdec$ operate?  Can
  you prove the scheme secure?
  
\item ``Encrypt then tag:'' output $(c \gets \skc.\skcenc_{k_e}(m), t
  \gets \mac.\mactag_{k_a}(c))$.  How should $\skcdec$ operate?  Is
  the scheme secure?
\end{enumerate}

\end{document}

%%% Local Variables: 
%%% mode: latex
%%% TeX-master: t
%%% End: 
