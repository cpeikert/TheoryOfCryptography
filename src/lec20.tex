\documentclass[11pt]{article}

\usepackage{fullpage,microtype,pdfsync,hyperref}
\usepackage{newtxtext}
\usepackage[T1]{fontenc}

\usepackage{mathtools,amsfonts,amssymb}
\usepackage[amsmath,amsthm,thmmarks,hyperref]{ntheorem}
\usepackage[capitalize,nameinlink,noabbrev]{cleveref}

\RequirePackage{mathtools}

%%% BLACKBOARD SYMBOLS

% hyperref package defines C and G, undefined to avoid conflicts
\let\C\relax
\let\G\relax
\newcommand{\C}{\ensuremath{\mathbb{C}}}
\newcommand{\D}{\ensuremath{\mathbb{D}}}
\newcommand{\F}{\ensuremath{\mathbb{F}}}
\newcommand{\G}{\ensuremath{\mathbb{G}}}
\newcommand{\J}{\ensuremath{\mathbb{J}}}
\newcommand{\N}{\ensuremath{\mathbb{N}}}
\newcommand{\Q}{\ensuremath{\mathbb{Q}}}
\newcommand{\R}{\ensuremath{\mathbb{R}}}
\newcommand{\T}{\ensuremath{\mathbb{T}}}
\newcommand{\Z}{\ensuremath{\mathbb{Z}}}
\newcommand{\QR}{\ensuremath{\mathbb{QR}}}

\newcommand{\Zt}{\ensuremath{\Z_t}}
\newcommand{\Zp}{\ensuremath{\Z_p}}
\newcommand{\Zq}{\ensuremath{\Z_q}}
\newcommand{\ZN}{\ensuremath{\Z_N}}
\newcommand{\Zps}{\ensuremath{\Z_p^*}}
\newcommand{\ZNs}{\ensuremath{\Z_N^*}}
\newcommand{\JN}{\ensuremath{\J_N}}
\newcommand{\QRN}{\ensuremath{\QR_{N}}}
\newcommand{\QRp}{\ensuremath{\QR_{p}}}

%%% THEOREM COMMANDS

\RequirePackage{amsthm}

\theoremstyle{plain}            % following are "theorem" style
\newtheorem{theorem}{Theorem}[section]
\newtheorem{lemma}[theorem]{Lemma}
\newtheorem{corollary}[theorem]{Corollary}
\newtheorem{proposition}[theorem]{Proposition}
\newtheorem{claim}[theorem]{Claim}
\newtheorem{fact}[theorem]{Fact}

\theoremstyle{definition}       % following are def style
\newtheorem{definition}[theorem]{Definition}
\newtheorem{conjecture}[theorem]{Conjecture}
\newtheorem{example}[theorem]{Example}
\newtheorem{protocol}[theorem]{Protocol}

\theoremstyle{remark}           % following are remark style
\newtheorem{remark}[theorem]{Remark}
\newtheorem{note}[theorem]{Note}

% equation numbering style
\numberwithin{equation}{section}

%%% GENERAL COMPUTING

\newcommand{\bit}{\ensuremath{\set{0,1}}}
\newcommand{\pmone}{\ensuremath{\set{-1,1}}}

% asymptotics
\DeclareMathOperator{\poly}{poly}
\DeclareMathOperator{\polylog}{polylog}
\DeclareMathOperator{\negl}{negl}
\newcommand{\Otil}{\ensuremath{\tilde{O}}}

% probability/distribution stuff
\DeclareMathOperator*{\E}{E}
\DeclareMathOperator*{\Var}{Var}

% sets in calligraphic type
\newcommand{\calD}{\ensuremath{\mathcal{D}}}
\newcommand{\calF}{\ensuremath{\mathcal{F}}}
\newcommand{\calG}{\ensuremath{\mathcal{G}}}
\newcommand{\calH}{\ensuremath{\mathcal{H}}}
\newcommand{\calX}{\ensuremath{\mathcal{X}}}
\newcommand{\calY}{\ensuremath{\mathcal{Y}}}

% types of indistinguishability
\newcommand{\compind}{\ensuremath{\stackrel{c}{\approx}}}
\newcommand{\statind}{\ensuremath{\stackrel{s}{\approx}}}
\newcommand{\perfind}{\ensuremath{\equiv}}

% font for general-purpose algorithms
\newcommand{\algo}[1]{\ensuremath{\mathsf{#1}}}
% font for general-purpose computational problems
\newcommand{\problem}[1]{\ensuremath{\mathsf{#1}}}
% font for complexity classes
\newcommand{\class}[1]{\ensuremath{\mathsf{#1}}}

% complexity classes and languages
\renewcommand{\P}{\class{P}}
\newcommand{\BPP}{\class{BPP}}
\newcommand{\NP}{\class{NP}}
\newcommand{\coNP}{\class{coNP}}
\newcommand{\AM}{\class{AM}}
\newcommand{\coAM}{\class{coAM}}
\newcommand{\IP}{\class{IP}}

%%% "LEFT-RIGHT" PAIRS OF SYMBOLS


% inner product
\DeclarePairedDelimiter\inner{\langle}{\rangle}
% absolute value
\DeclarePairedDelimiter\abs{\lvert}{\rvert}
% a set
\DeclarePairedDelimiter\set{\{}{\}}
% parens
\DeclarePairedDelimiter\parens{(}{)}
% tuple, alias for parens
\DeclarePairedDelimiter\tuple{(}{)}
% square brackets
\DeclarePairedDelimiter\bracks{[}{]}
% rounding off
\DeclarePairedDelimiter\round{\lfloor}{\rceil}
% floor function
\DeclarePairedDelimiter\floor{\lfloor}{\rfloor}
% ceiling function
\DeclarePairedDelimiter\ceil{\lceil}{\rceil}
% length of some vector, element
\DeclarePairedDelimiter\length{\lVert}{\rVert}
% "lifting" of a residue class
\DeclarePairedDelimiter\lift{\llbracket}{\rrbracket}
\DeclarePairedDelimiter\len{\lvert}{\rvert}

%%% CRYPTO-RELATED NOTATION

% KEYS AND RELATED

\newcommand{\key}[1]{\ensuremath{#1}}

\newcommand{\pk}{\key{pk}}
\newcommand{\vk}{\key{vk}}
\newcommand{\sk}{\key{sk}}
\newcommand{\mpk}{\key{mpk}}
\newcommand{\msk}{\key{msk}}
\newcommand{\fk}{\key{fk}}
\newcommand{\id}{id}
\newcommand{\keyspace}{\ensuremath{\mathcal{K}}}
\newcommand{\msgspace}{\ensuremath{\mathcal{M}}}
\newcommand{\ctspace}{\ensuremath{\mathcal{C}}}
\newcommand{\tagspace}{\ensuremath{\mathcal{T}}}
\newcommand{\idspace}{\ensuremath{\mathcal{ID}}}

\newcommand{\concat}{\ensuremath{\|}}

% GAMES

% advantage
\newcommand{\advan}{\ensuremath{\mathbf{Adv}}}

% different attack models
\newcommand{\attack}[1]{\ensuremath{\text{#1}}}

\newcommand{\atk}{\attack{atk}} % dummy attack
\newcommand{\indcpa}{\attack{ind-cpa}}
\newcommand{\indcca}{\attack{ind-cca}}
\newcommand{\anocpa}{\attack{ano-cpa}} % anonymous
\newcommand{\anocca}{\attack{ano-cca}}
\newcommand{\euacma}{\attack{eu-acma}} % forgery: adaptive chosen-message
\newcommand{\euscma}{\attack{eu-scma}} % forgery: static chosen-message
\newcommand{\suacma}{\attack{su-acma}} % strongly unforgeable

% ADVERSARIES
\newcommand{\attacker}[1]{\ensuremath{\mathcal{#1}}}

\newcommand{\Adv}{\attacker{A}}
\newcommand{\AdvA}{\attacker{A}}
\newcommand{\AdvB}{\attacker{B}}
\newcommand{\Dist}{\attacker{D}}
\newcommand{\Sim}{\attacker{S}}
\newcommand{\Ora}{\attacker{O}}
\newcommand{\Inv}{\attacker{I}}
\newcommand{\For}{\attacker{F}}

% CRYPTO SCHEMES

\newcommand{\scheme}[1]{\ensuremath{\text{#1}}}

% pseudorandom stuff
\newcommand{\prg}{\algo{PRG}}
\newcommand{\prf}{\algo{PRF}}
\newcommand{\prp}{\algo{PRP}}

% symmetric-key cryptosystem
\newcommand{\skc}{\scheme{SKC}}
\newcommand{\skcgen}{\algo{Gen}}
\newcommand{\skcenc}{\algo{Enc}}
\newcommand{\skcdec}{\algo{Dec}}

% public-key cryptosystem
\newcommand{\pkc}{\scheme{PKC}}
\newcommand{\pkcgen}{\algo{Gen}}
\newcommand{\pkcenc}{\algo{Enc}} % can also use \kemenc and \kemdec
\newcommand{\pkcdec}{\algo{Dec}}

% digital signatures
\newcommand{\sig}{\scheme{SIG}}
\newcommand{\siggen}{\algo{Gen}}
\newcommand{\sigsign}{\algo{Sign}}
\newcommand{\sigver}{\algo{Ver}}

% message authentication code
\newcommand{\mac}{\scheme{MAC}}
\newcommand{\macgen}{\algo{Gen}}
\newcommand{\mactag}{\algo{Tag}}
\newcommand{\macver}{\algo{Ver}}

% key-encapsulation mechanism
\newcommand{\kem}{\scheme{KEM}}
\newcommand{\kemgen}{\algo{Gen}}
\newcommand{\kemenc}{\algo{Encaps}}
\newcommand{\kemdec}{\algo{Decaps}}

% identity-based encryption
\newcommand{\ibe}{\scheme{IBE}}
\newcommand{\ibesetup}{\algo{Setup}}
\newcommand{\ibeext}{\algo{Ext}}
\newcommand{\ibeenc}{\algo{Enc}}
\newcommand{\ibedec}{\algo{Dec}}

% hierarchical IBE (as key encapsulation)
\newcommand{\hibe}{\scheme{HIBE}}
\newcommand{\hibesetup}{\algo{Setup}}
\newcommand{\hibeext}{\algo{Extract}}
\newcommand{\hibeenc}{\algo{Encaps}}
\newcommand{\hibedec}{\algo{Decaps}}

% binary tree encryption (as key encapsulation)
\newcommand{\bte}{\scheme{BTE}}
\newcommand{\btesetup}{\algo{Setup}}
\newcommand{\bteext}{\algo{Extract}}
\newcommand{\bteenc}{\algo{Encaps}}
\newcommand{\btedec}{\algo{Decaps}}

% trapdoor functions
\newcommand{\tdf}{\scheme{TDF}}
\newcommand{\tdfgen}{\algo{Gen}}
\newcommand{\tdfeval}{\algo{Eval}}
\newcommand{\tdfinv}{\algo{Invert}}
\newcommand{\tdfver}{\algo{Ver}}

%%% PROTOCOLS

\newcommand{\out}{\text{out}}
\newcommand{\view}{\text{view}}

%%% COMMANDS FOR LECTURES/HOMEWORKS

\RequirePackage{fancyhdr}

\newcommand{\lecheader}{%
  \chead{\large \textbf{Lecture \lecturenum\\\lecturetopic}}

  \lhead{\small \textbf{Theory of Cryptography}\\}

  \rhead{\small \textbf{Instructor:
      \href{http://www.eecs.umich.edu/~cpeikert/}{Chris Peikert}\\Scribe:
      \scribename}}

  \setlength{\headheight}{20pt}
  \setlength{\headsep}{16pt}
}


%%% xsim settings

\RequirePackage{xsim}
\RequirePackage{needspace}

\DeclareExerciseEnvironmentTemplate{QRef}
{%
    \par\vspace{\baselineskip}
    \Needspace*{2\baselineskip}
    \noindent
    \hyperref[sol:\ExerciseID]{\textbf{\XSIMmixedcase{\GetExerciseName}~\GetExerciseProperty{counter}.}}
    \label{ques:\ExerciseID}
}{}

\DeclareExerciseEnvironmentTemplate{ARef}{%
        \Needspace*{2\baselineskip}
        \noindent
        \hyperref[ques:\ExerciseID]{\textbf{\XSIMmixedcase{\GetExerciseParameter{exercise-name}}~\GetExerciseProperty{counter}.}}
        \label{sol:\ExerciseID}
        \GetExerciseBody{exercise}
        \newline
        \textbf{\XSIMmixedcase{\GetExerciseName}.}
}{\par\vspace{\baselineskip}}

\DeclareExerciseHeadingTemplate{simple}{%
    \section*{\XSIMmixedcase{\GetExerciseParameter{solution-name}s}}
}

\DeclareExerciseType{question}{%
    exercise-env = question,
    solution-env = answer,
    exercise-name = Question,
    solution-name = Answer,
    exercise-template = QRef,
    solution-template = ARef,
}

\xsimsetup{%
  path = {questions},
  file-extension = {aux},
  print-solutions/headings-template=simple
}

\AtEndDocument{\pagebreak\printsolutions}


% VARIABLES

\newcommand{\lecturenum}{20}
\newcommand{\lecturetopic}{Secure 2-Party Computation}
\newcommand{\scribename}{Anand Louis}

\newcommand{\defeq}{\stackrel{\textup{def}}{=}}

% END OF VARIABLES

\lecheader

\pagestyle{plain}               % default: no special header

\begin{document}

\thispagestyle{fancy} % first page should have special header

% LECTURE MATERIAL STARTS HERE

\section{Introduction}

Consider the {\em Billionaire's Problem}: {\em Larry} and {\em Sergey}
are both wealthy men.  They want to design a protocol to find out
whose net worth is higher, without having to reveal their net worths
to each other.  Since they are business partners, they trust each
other to follow the protocol truthfully (i.e., to use their true worth
as inputs and to follow the protocol instructions faithfully), but
they still do not want the protocol to reveal more about their net
worths than is absolutely necessary.  Can this be achieved?  More
fundamentally, how do we define the notion of security for this goal?

Note that ``total'' privacy cannot be achieved, as they will each
learn something new about the other's wealth, namely, whether it is
greater or less than his own.  In some special cases, Larry can infer
Sergey's net worth \emph{entirely} from just this single piece of
knowledge.  For example, if Larry's net worth is \$1 and Sergey's net
worth is \$0, then when Larry learns that Sergey's worth is less than
his own, he can infer that Sergey's net worth is exactly \$0 (we
ignore the possibility of Sergey being in debt).  However, this
leakage of knowledge is \emph{inherent in the task they are carrying
  out for these values}, and therefore should not should not be
considered a deficiency of any particular protocol they might use.

Alternatively, Larry might start with some prior knowledge about
Sergey's wealth that might enable him to infer Sergey's exact net
worth from just knowing whose net worth is higher.  For example, if
Larry knows that Sergey's net worth is either \$4 billion or \$5
billion, and his own worth is \$4.5 billion, then learning who is
wealthier immediately reveals Sergey's net worth to Larry.  This also
should not be considered a violation of our security goal.

In both of these examples, we must be content to let Larry learn
whatever he can infer from the final result (i.e., who is wealthier)
--- \emph{but we want him to learn nothing more than that}!  So in
defining security, we will aim to restrict the ``\emph{relative
  knowledge}'' revealed by a protocol, versus what is learned from the
outcome alone.  Similarly to the setting of zero knowledge, we will do
so using the notion of an efficient simulator that is given only the
input and output of the party, and must simulate the entire view of
that party in the protocol.

\section{Secure Two-Party Computation}
\label{sec:2pc}

Here we formalize a model and security definition for the informal
goals described above.

\subsection{Model}
\label{sec:model}

We will consider a very simplified model that does not capture many
real-world concerns, but is still rich enough to make the problem
interesting and non-trivial.

\begin{enumerate}
\item There are two parties (more formally, two ppt algorithms) $P_1$
  and $P_2$, who have inputs $x_1$ and $x_2$ respectively, and wish to
  evaluate a public polynomial time-computable function
  $f(\cdot,\cdot)$ on those inputs.  For example, in the billionaire's
  problem, $f(x_{1}, x_{2}) = [x_{1} > x_{2}]$.

  Without loss of generality, we may assume that $f$ is a
  \emph{deterministic} function that outputs a \emph{single} value
  that is given to both parties.  If we wish for $P_{1}$ and $P_{2}$
  to receive the outputs of two possibly different deterministic
  functions $f_{1}(\cdot, \cdot)$, $f_{2}(\cdot,\cdot)$
  (respectively), this can be emulated using a single function
  $f'((x_1,r_1),(x_2,r_2)) = (f_1(x_1,x_2) \oplus r_1 \| f_2(x_1,x_2)
  \oplus r_2 )$, where each $P_{i}$ augments its own input $x_{i}$ by
  a uniformly random string $r_{i}$ of appropriate length.  Since
  $r_1$ and $r_2$ are chosen uniformly at random and are independent
  of everything else, the output $f_{1}(x_{1}, x_{2})$ is perfectly
  hidden from $P_{2}$, as is $f_{2}(x_{1}, x_{2})$ from $P_{1}$.

  We can also evaluate a \emph{randomized} function $f$ by emulating
  it with a deterministic function (showing how to do this is one of
  your homework problems).  However, security becomes quite a bit
  subtler to define in this case; see below.

\item We assume that the parties are {\em semi-honest}, often called
  ``honest but curious.''  That is, they run the protocol exactly as
  specified (no deviations, malicious or otherwise), but may try to
  learn as much as possible about the input of the other party from
  their views of the protocol.  Hence, we want the view of each party
  not to leak more knowledge than necessary.

\item As usual, the view of a party $P_i$ in an interaction with the
  other party on their inputs $x_{1}, x_{2}$, denoted
  $\view_{P_{i}}[P_{1}(x_{1}) \leftrightarrow P_{2}(x_{2})]$, consists
  of its input $x_i$, the random coins $r_{P_i}$ used by $P_{i}$, and
  all the messages received from the other party.  The final output of
  $P_{i}$ is denoted $\out_{P_{i}}[P_{1}(x_{1}) \leftrightarrow
  P_{2}(x_{2})]$.
\end{enumerate}

\subsection{Security Definition}
\label{sec:security-definition}

\begin{definition}
  \label{def:two-party}
  A pair of ppt machines $(P_1,P_2)$ is a secure 2-party protocol (for
  static, semi-honest adversaries) for a deterministic polynomial
  time-computable function $f(\cdot,\cdot)$ if the following
  properties hold:
  \begin{enumerate}
  \item \emph{Completeness}: for all $i \in \set{1,2}$ and all $x_1,
    x_2 \in \bit^{*}$, we have (with probability $1$): \[
    \out_{P_{i}}[P_{1}(x_{1}) \leftrightarrow P_{2}(x_{2})] =
    f(x_1,x_2). \]

  \item \emph{Privacy}: there exist nuppt simulators $\Sim_1, \Sim_2$
    such that for all $x_{1}, x_{2} \in \bit^{*}$ and all $i \in
    \set{1,2}$, \[ \view_{P_{i}}[P_{1}(x_{1}) \leftrightarrow
    P_{2}(x_{2})] \compind \Sim_{i}(x_{i}, f(x_{1}, x_{2})). \]
  \end{enumerate}
\end{definition}

A few remarks are in order.  First, privacy is \emph{per-instance}:
the only knowledge leaked to a party by the protocol on inputs $x_{1},
x_{2}$ is whatever can be inferred (efficiently) from the party's own
input and the value of $f(x_{1}, x_{2})$.  For example, if the inputs
are such that $f(x_{1}, x_{2})$ reveals nothing at all, then the
execution of the protocol on those inputs should also reveal nothing;
conversely, if the output reveals everything about both parties'
inputs, then the protocol is allowed to leak everything as well.
Second, any ``prior knowledge'' that the parties have about each
others' inputs is captured by the non-uniformity of the simulator (and
implicit distinguisher) in the definition of privacy.

\subsection{Definition for Randomized Functions}
\label{sec:defin-rand-funct}

\Cref{def:two-party} is relatively straightforward due to
the simplicities of our model, in particular, the deterministic nature
of $f$.  We briefly discuss some of the issues that arise in defining
security for randomized functions.  First, how should completeness be
defined?  It no longer makes sense to demand that $\out_{P_{1}} =
f(x_{1}, x_{2})$, since we now have a random variable $f(x_{1}, x_{2};
r)$ over the choice of $r$ (which neither party should be able to
influence).  Instead, we want that both $\out_{P_{1}}$ and
$\out_{P_{2}}$ in a \emph{single} execution of the protocol are
\emph{simultaneously} distributed as $f(x_{1}, x_{2}; r)$ for
\emph{the same} random $r$.  This is so that the protocol between
$P_{1}$ and $P_{2}$ has the effect of emulating a single, consistent
randomized evaluation of $f$.  Formally, we want that for all $x_{1},
x_{2}$, \[ (\out_{P_{1}}, \out_{P_{2}})[P_{1}(x_{1}) \leftrightarrow
P_{2}(x_{2})] \compind (f(x_{1}, x_{2}; r), f(x_{1}, x_{2}; r)), \]
where $r$ is uniformly random and the same in both appearances of $f$.

The next natural question is how to define \emph{privacy} against a
semi-honest party.  Again, simultaneity of the respective views of
$P_{1}$ and $P_{2}$ is an important issue, and is even more subtle to
get right.  It turns out that the proper way of addressing all these
concerns is to define correctness and privacy \emph{all together} by
comparing two \emph{joint} distributions: the ``real world''
distribution of the parties' outputs and the semi-honest party's view,
versus the ``ideal world'' distribution of the function output and
simulated view (again, for a \emph{single} randomized evaluation of
$f$).  Formally, we require that there exist nuppt simulators
$\Sim_{1}$, $\Sim_{2}$ such that for all $x_{1}, x_{2}$ and all $i \in
\set{1,2}$, \[ (\out_{P_{1}}, \out_{P_{2}},\view_{P_i})[P_1(x_1)
\leftrightarrow P_2(x_2)] \compind (f(x_{1}, x_{2}; r),
f(x_1,x_2;r),S_i(x_i,f(x_1,x_2;r))). \] Note that the above condition
automatically implies the correctness condition above, so it is the
only one needed to prove security.

\subsection{Secure Protocol for Addition}
\label{sec:secure-prot-addit}

As a brief test case, we consider a contrived protocol for evaluating
the addition function $f(x_{1}, x_{2}) = x_{1} + x_{2}$.

\begin{center}
  \begin{tabular}{ccc}
    $P_1(x_1)$ & & $P_2(x_2)$ \\
    & $\underrightarrow{\quad x_1 \quad}$ & \\
    & $\underleftarrow{\quad x_2 \quad}$ & \\
    \text{output} $x_1 + x_2$ &  &  \text{output} $x_1 + x_2$
  \end{tabular}
\end{center}

Clearly the protocol is complete.  For privacy, since $P_1$ is
entitled to know the value of $x_1 + x_2$, and also already knows
$x_1$, he can trivially infer the value of $x_2$.  Formally, we can
give a simulator $\Sim_{1}(x_{1}, s = f(x_{1}, x_{2}) = x_{1}+x_{2})$
that just outputs the view consisting of input $x_{1}$, empty
randomness, and a single message $s-x_{1}$ coming from $P_{2}$.
Clearly, this view is identical to $P_{1}$'s view in the real
protocol.

\section{Secure Protocol for Arbitrary Circuits}
\label{sec:secure-prot-circuits}

We now describe a protocol, originally described by Yao, for
evaluating an \emph{arbitrary} function $f$ represented as a boolean
(logical) circuit.  We describe the basic idea for just a single logic
gate, and then outline how it generalizes to arbitrary circuits.

Let $g : \bit \times \bit \to \bit$ be an arbitrary logic gate on two
input bits (e.g., the NAND function).  Party $P_1$ holds the first
input bit $x_1 \in \bit$ and party $P_2$ holds the second input bit
$x_2 \in \bit$.  Together they wish to compute $g(x_1,x_2)$ securely,
in the sense of \cref{def:two-party}.

At a high level, the protocol works like this:
\begin{center}
  \begin{tabular}{ccc}
    $P_1(x_1)$ & & $P_2(x_2)$ \\
    & $\underrightarrow{\quad \text{``garbled'' gate $g$} \quad}$ & \\
    & $\underrightarrow{\quad \text{``garbled'' input $x_{1}$} \quad}$ & \\
    & \fbox{$\underrightarrow{\quad \text{oblivious transfer of
          ``garbled'' input $x_{2}$} \quad}$} & \\
    & & compute garbled output \\
    & $\underrightarrow{\quad \text{``dictionary'' for garbled outputs}
    \quad}$ &
    \\
    & & look up actual value $g(x_{1}, x_{2})$ \\
    & $\underleftarrow{\quad g(x_{1}, x_{2}) \quad}$ & \\
    output $g(x_1,x_2)$ & & output $g(x_1,x_2)$
  \end{tabular}
\end{center}

For intuition, the crucial points for security are:
\begin{itemize}
\item $P_{2}$ never sees $x_{1}$ in an ``ungarbled'' form, so
  $P_{2}$ learning nothing about $x_{1}$.
\item By the security of the ``oblivious transfer'' sub-protocol
  (described below), $P_{1}$ learns nothing about $x_{2}$.
\item Using the garbled inputs $x_{1}$, $x_{2}$ with the garbled gate,
  $P_{2}$ can ``obliviously'' compute the garbled output for
  \emph{only} the correct value of $g(x_{1}, x_{2})$.
\end{itemize}

Concretely, these ideas are implemented using basic symmetric-key
encryption.  The idea is the following: each ``wire'' of the gate (the
two inputs and one output) is associated with a pair of random
symmetric encryption keys, chosen by $P_{1}$; the two keys correspond
to the two possible ``values'' (0 or 1) that the wire can take.  For
$i= 1,2$, let $k^i_0, k^i_1$ be the keys corresponding to the input
wire $x_{i}$, and let $k^{o}_0, k^o_1$ be the keys corresponding to
the output wire.  The ``garbled circuit'' that $P_1$ sends to $P_2$ is
a table of four doubly encrypted values, presented in a \emph{random}
order:

\begin{center}
  \begin{tabular}{|c|}
    \hline
    $\skcenc_{k^1_0}(\skcenc_{k^2_0}(k^o_{g(0,0)}))$ \\
    $\skcenc_{k^1_0}(\skcenc_{k^2_1}(k^o_{g(0,1)}))$ \\
    $\skcenc_{k^1_1}(\skcenc_{k^2_0}(k^o_{g(1,0)}))$ \\
    $\skcenc_{k^1_1}(\skcenc_{k^2_1}(k^o_{g(1,1)}))$ \\ \hline
  \end{tabular}
\end{center}

Observe that if $P_{2}$ knows (say) $k^{1}_{0}$ and $k^{2}_{1}$, i.e.,
the keys corresponding to inputs $x_{1}=0$ and $x_{2} = 1$, then
$P_{1}$ can decrypt $k^{o}_{g(0,1)}$, the key corresponding to the
output value of the gate, \emph{but none of the other entries!}  (Note
that this requires the encryption scheme to satisfy some simple
properties, such as the ability to detect when a ciphertext has
decrypted successfully.  These properties are easy to obtain.)  The
random order of the table prevents $P_{1}$ from learning the
``meaning'' of the keys that it knows, otherwise this information
would be leaked by which of the table entries decrypt properly.  In
conclusion, knowing exactly one key for each input wire allows $P_{2}$
to learn exactly one key (the correct one) for the output wire,
without learning the meanings of any of the keys.

The only remaining question is how $P_{2}$ obtains the right keys for
the input wires.  For $x_{1}$, this is easy: $P_1$ just sends $P_2$
the key $k^{1}_{x_{1}}$ corresponding to its input bit $x_{1}$.  Note
that $P_2$ learns nothing about $x_1$ from this.  Next, $P_1$ and
$P_2$ run an ``oblivious transfer'' protocol (described in the next
subsection) which allows $P_2$ to learn $k^{2}_{x_{2}}$, and
\emph{only} $k^{2}_{x_{2}}$, without revealing anything about the
value of $x_{2}$ to $P_{1}$.

Finally, $P_{1}$ tells $P_{2}$ the ``meanings'' of the two possible
output keys $k^{o}_{0}, k^{o}_{1}$, which reveals to $P_{2}$ the value
of $g(x_{1}, x_{2})$.  Then $P_{2}$ sends this value to $P_{1}$ as
well (recall that both parties are semi-honest, so neither will lie).

For more complex circuits $f$, the protocol generalizes in a
straightforward manner: $P_{1}$ chooses two keys for every wire in the
circuit, and constructs a garbled table for each gate, using the
appropriate keys for the input and output wires.  $P_{2}$ can compute
the garbled gates iteratively, while remaining oblivious to the
meanings of the intermediate wires.  Then $P_{1}$ finally reveals the
meanings of just the output wires.

A proof of security for this construction is beyond the scope of this
lecture, but contains no surprises; see the paper by Lindell and
Pinkas for a full rigorous proof.  The key point is that a simulator
can construct garbled gates that \emph{always} result in the same key
being decrypted (irrespective of which inputs keys were used), thus
allowing the simulator to ``force'' $P_{2}$ to output the correct
value $f(x_{1}, x_{2})$.  Security of the symmetric encryption scheme
prevents $P_{2}$ from detecting these malformed garbled gates, since
$P_{2}$ can only decrypt one entry from each gate.

\section{Oblivious Transfer}
\label{sec:oblivious-transfer}

We conclude by describing how to perform an oblivious transfer between
the two parties.  We will consider the specific form of oblivious
transfer that is required to complete our protocol: $P_1$ is holding
two bits $b_0,b_1$ and wants to transfer \emph{exactly one} of them to
$P_2$, according to $P_{2}$'s choice bit $\sigma = x_{2}$, while
learning nothing about which one was received.  (To transfer entire
keys $k_{0}, k_{1} \in \bit^{n}$, the parties can just run the
protocol $n$ times, using the same choice bit $\sigma$ and the
corresponding pairs of bits from $k_{0}, k_{1}$).

Our protocol relies on a family of trapdoor permutations $\set{f_{s}
  \colon \bit^{n} \to \bit^{n}}$ with hard-core predicate $h \colon
\bit^{n} \to \bit$.

\begin{center}
  \begin{tabular}{ccc}
    $P_1(b_0,b_1)$ & & $P_2(\sigma)$ \\ 
    choose $(f_s, f_{s}^{-1})$ & & \\
    & $\underrightarrow{\quad f_s \quad}$ & \\
    & & $v_{\sigma} \gets \bit^{n}$ \\
    & & $w_{\sigma} = f_{s}(v_{\sigma})$ \\
    & & $w_{1-\sigma} \gets \bit^{n}$ \\
    & $\underleftarrow{\quad w_{0}, w_{1} \quad}$ & \\
    let $v_{0} = f_{s}^{-1}(w_{0})$, $v_{1} = f_{s}^{-1}(w_{1})$ &
    & \\
    & $\underrightarrow{\quad c_{0} = h(v_{0}) \oplus b_{0}, c_{1} =
      h(v_{1}) \oplus b_{1} \quad}$ & \\
    & & output $h(v_{\sigma}) \oplus c_{\sigma}$
  \end{tabular}
\end{center}

In words, $P_1$ picks a random function $f_s$ (with trapdoor) from the
family and sends it to $P_2$.  Then $P_2$ chooses uniformly random
$w_{0}, w_{1} \in \bit^{n}$ so that it knows the preimage of
\emph{only} $w_{\sigma}$, and sends these to $P_{1}$.  (Observe that
this reveals no information about $\sigma$ to $P_{1}$ because $w_{0},
w_{1}$ are uniform and independent.)  Next, $P_{1}$ encrypts its two
bits $b_{0}, b_{1}$ using the hard-core predicate $h$ on the preimages
of $w_{0}, w_{1}$, respectively.  Finally $P_{2}$, knowing the
preimage $w_{\sigma}$, can decrypt $b_{\sigma}$, but it learns nothing
about $b_{1-\sigma}$ due to the hardness of~$h$.  Note that the
protocol crucially relies on the fact that $P_{2}$ is semi-honest,
otherwise it could choose $w_{0}, w_{1}$ so that it knew both
preimages.  (A full, formal proof of security for this protocol is not
too hard, and is a worthwhile exercise.)
 
\end{document}

%%% Local Variables: 
%%% mode: latex
%%% TeX-master: t
%%% End: 
