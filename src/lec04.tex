\documentclass[11pt]{article}

\usepackage{fullpage}
\usepackage{newtxtext}
\usepackage[T1]{fontenc}
\usepackage{hyperref,microtype,pdfsync}
\usepackage{amssymb}
\usepackage{algorithm,algorithmic}

\RequirePackage{mathtools}

%%% BLACKBOARD SYMBOLS

% hyperref package defines C and G, undefined to avoid conflicts
\let\C\relax
\let\G\relax
\newcommand{\C}{\ensuremath{\mathbb{C}}}
\newcommand{\D}{\ensuremath{\mathbb{D}}}
\newcommand{\F}{\ensuremath{\mathbb{F}}}
\newcommand{\G}{\ensuremath{\mathbb{G}}}
\newcommand{\J}{\ensuremath{\mathbb{J}}}
\newcommand{\N}{\ensuremath{\mathbb{N}}}
\newcommand{\Q}{\ensuremath{\mathbb{Q}}}
\newcommand{\R}{\ensuremath{\mathbb{R}}}
\newcommand{\T}{\ensuremath{\mathbb{T}}}
\newcommand{\Z}{\ensuremath{\mathbb{Z}}}
\newcommand{\QR}{\ensuremath{\mathbb{QR}}}

\newcommand{\Zt}{\ensuremath{\Z_t}}
\newcommand{\Zp}{\ensuremath{\Z_p}}
\newcommand{\Zq}{\ensuremath{\Z_q}}
\newcommand{\ZN}{\ensuremath{\Z_N}}
\newcommand{\Zps}{\ensuremath{\Z_p^*}}
\newcommand{\ZNs}{\ensuremath{\Z_N^*}}
\newcommand{\JN}{\ensuremath{\J_N}}
\newcommand{\QRN}{\ensuremath{\QR_{N}}}
\newcommand{\QRp}{\ensuremath{\QR_{p}}}

%%% THEOREM COMMANDS

\RequirePackage{amsthm}

\theoremstyle{plain}            % following are "theorem" style
\newtheorem{theorem}{Theorem}[section]
\newtheorem{lemma}[theorem]{Lemma}
\newtheorem{corollary}[theorem]{Corollary}
\newtheorem{proposition}[theorem]{Proposition}
\newtheorem{claim}[theorem]{Claim}
\newtheorem{fact}[theorem]{Fact}

\theoremstyle{definition}       % following are def style
\newtheorem{definition}[theorem]{Definition}
\newtheorem{conjecture}[theorem]{Conjecture}
\newtheorem{example}[theorem]{Example}
\newtheorem{protocol}[theorem]{Protocol}

\theoremstyle{remark}           % following are remark style
\newtheorem{remark}[theorem]{Remark}
\newtheorem{note}[theorem]{Note}

% equation numbering style
\numberwithin{equation}{section}

%%% GENERAL COMPUTING

\newcommand{\bit}{\ensuremath{\set{0,1}}}
\newcommand{\pmone}{\ensuremath{\set{-1,1}}}

% asymptotics
\DeclareMathOperator{\poly}{poly}
\DeclareMathOperator{\polylog}{polylog}
\DeclareMathOperator{\negl}{negl}
\newcommand{\Otil}{\ensuremath{\tilde{O}}}

% probability/distribution stuff
\DeclareMathOperator*{\E}{E}
\DeclareMathOperator*{\Var}{Var}

% sets in calligraphic type
\newcommand{\calD}{\ensuremath{\mathcal{D}}}
\newcommand{\calF}{\ensuremath{\mathcal{F}}}
\newcommand{\calG}{\ensuremath{\mathcal{G}}}
\newcommand{\calH}{\ensuremath{\mathcal{H}}}
\newcommand{\calX}{\ensuremath{\mathcal{X}}}
\newcommand{\calY}{\ensuremath{\mathcal{Y}}}

% types of indistinguishability
\newcommand{\compind}{\ensuremath{\stackrel{c}{\approx}}}
\newcommand{\statind}{\ensuremath{\stackrel{s}{\approx}}}
\newcommand{\perfind}{\ensuremath{\equiv}}

% font for general-purpose algorithms
\newcommand{\algo}[1]{\ensuremath{\mathsf{#1}}}
% font for general-purpose computational problems
\newcommand{\problem}[1]{\ensuremath{\mathsf{#1}}}
% font for complexity classes
\newcommand{\class}[1]{\ensuremath{\mathsf{#1}}}

% complexity classes and languages
\renewcommand{\P}{\class{P}}
\newcommand{\BPP}{\class{BPP}}
\newcommand{\NP}{\class{NP}}
\newcommand{\coNP}{\class{coNP}}
\newcommand{\AM}{\class{AM}}
\newcommand{\coAM}{\class{coAM}}
\newcommand{\IP}{\class{IP}}

%%% "LEFT-RIGHT" PAIRS OF SYMBOLS


% inner product
\DeclarePairedDelimiter\inner{\langle}{\rangle}
% absolute value
\DeclarePairedDelimiter\abs{\lvert}{\rvert}
% a set
\DeclarePairedDelimiter\set{\{}{\}}
% parens
\DeclarePairedDelimiter\parens{(}{)}
% tuple, alias for parens
\DeclarePairedDelimiter\tuple{(}{)}
% square brackets
\DeclarePairedDelimiter\bracks{[}{]}
% rounding off
\DeclarePairedDelimiter\round{\lfloor}{\rceil}
% floor function
\DeclarePairedDelimiter\floor{\lfloor}{\rfloor}
% ceiling function
\DeclarePairedDelimiter\ceil{\lceil}{\rceil}
% length of some vector, element
\DeclarePairedDelimiter\length{\lVert}{\rVert}
% "lifting" of a residue class
\DeclarePairedDelimiter\lift{\llbracket}{\rrbracket}
\DeclarePairedDelimiter\len{\lvert}{\rvert}

%%% CRYPTO-RELATED NOTATION

% KEYS AND RELATED

\newcommand{\key}[1]{\ensuremath{#1}}

\newcommand{\pk}{\key{pk}}
\newcommand{\vk}{\key{vk}}
\newcommand{\sk}{\key{sk}}
\newcommand{\mpk}{\key{mpk}}
\newcommand{\msk}{\key{msk}}
\newcommand{\fk}{\key{fk}}
\newcommand{\id}{id}
\newcommand{\keyspace}{\ensuremath{\mathcal{K}}}
\newcommand{\msgspace}{\ensuremath{\mathcal{M}}}
\newcommand{\ctspace}{\ensuremath{\mathcal{C}}}
\newcommand{\tagspace}{\ensuremath{\mathcal{T}}}
\newcommand{\idspace}{\ensuremath{\mathcal{ID}}}

\newcommand{\concat}{\ensuremath{\|}}

% GAMES

% advantage
\newcommand{\advan}{\ensuremath{\mathbf{Adv}}}

% different attack models
\newcommand{\attack}[1]{\ensuremath{\text{#1}}}

\newcommand{\atk}{\attack{atk}} % dummy attack
\newcommand{\indcpa}{\attack{ind-cpa}}
\newcommand{\indcca}{\attack{ind-cca}}
\newcommand{\anocpa}{\attack{ano-cpa}} % anonymous
\newcommand{\anocca}{\attack{ano-cca}}
\newcommand{\euacma}{\attack{eu-acma}} % forgery: adaptive chosen-message
\newcommand{\euscma}{\attack{eu-scma}} % forgery: static chosen-message
\newcommand{\suacma}{\attack{su-acma}} % strongly unforgeable

% ADVERSARIES
\newcommand{\attacker}[1]{\ensuremath{\mathcal{#1}}}

\newcommand{\Adv}{\attacker{A}}
\newcommand{\AdvA}{\attacker{A}}
\newcommand{\AdvB}{\attacker{B}}
\newcommand{\Dist}{\attacker{D}}
\newcommand{\Sim}{\attacker{S}}
\newcommand{\Ora}{\attacker{O}}
\newcommand{\Inv}{\attacker{I}}
\newcommand{\For}{\attacker{F}}

% CRYPTO SCHEMES

\newcommand{\scheme}[1]{\ensuremath{\text{#1}}}

% pseudorandom stuff
\newcommand{\prg}{\algo{PRG}}
\newcommand{\prf}{\algo{PRF}}
\newcommand{\prp}{\algo{PRP}}

% symmetric-key cryptosystem
\newcommand{\skc}{\scheme{SKC}}
\newcommand{\skcgen}{\algo{Gen}}
\newcommand{\skcenc}{\algo{Enc}}
\newcommand{\skcdec}{\algo{Dec}}

% public-key cryptosystem
\newcommand{\pkc}{\scheme{PKC}}
\newcommand{\pkcgen}{\algo{Gen}}
\newcommand{\pkcenc}{\algo{Enc}} % can also use \kemenc and \kemdec
\newcommand{\pkcdec}{\algo{Dec}}

% digital signatures
\newcommand{\sig}{\scheme{SIG}}
\newcommand{\siggen}{\algo{Gen}}
\newcommand{\sigsign}{\algo{Sign}}
\newcommand{\sigver}{\algo{Ver}}

% message authentication code
\newcommand{\mac}{\scheme{MAC}}
\newcommand{\macgen}{\algo{Gen}}
\newcommand{\mactag}{\algo{Tag}}
\newcommand{\macver}{\algo{Ver}}

% key-encapsulation mechanism
\newcommand{\kem}{\scheme{KEM}}
\newcommand{\kemgen}{\algo{Gen}}
\newcommand{\kemenc}{\algo{Encaps}}
\newcommand{\kemdec}{\algo{Decaps}}

% identity-based encryption
\newcommand{\ibe}{\scheme{IBE}}
\newcommand{\ibesetup}{\algo{Setup}}
\newcommand{\ibeext}{\algo{Ext}}
\newcommand{\ibeenc}{\algo{Enc}}
\newcommand{\ibedec}{\algo{Dec}}

% hierarchical IBE (as key encapsulation)
\newcommand{\hibe}{\scheme{HIBE}}
\newcommand{\hibesetup}{\algo{Setup}}
\newcommand{\hibeext}{\algo{Extract}}
\newcommand{\hibeenc}{\algo{Encaps}}
\newcommand{\hibedec}{\algo{Decaps}}

% binary tree encryption (as key encapsulation)
\newcommand{\bte}{\scheme{BTE}}
\newcommand{\btesetup}{\algo{Setup}}
\newcommand{\bteext}{\algo{Extract}}
\newcommand{\bteenc}{\algo{Encaps}}
\newcommand{\btedec}{\algo{Decaps}}

% trapdoor functions
\newcommand{\tdf}{\scheme{TDF}}
\newcommand{\tdfgen}{\algo{Gen}}
\newcommand{\tdfeval}{\algo{Eval}}
\newcommand{\tdfinv}{\algo{Invert}}
\newcommand{\tdfver}{\algo{Ver}}

%%% PROTOCOLS

\newcommand{\out}{\text{out}}
\newcommand{\view}{\text{view}}

%%% COMMANDS FOR LECTURES/HOMEWORKS

\RequirePackage{fancyhdr}

\newcommand{\lecheader}{%
  \chead{\large \textbf{Lecture \lecturenum\\\lecturetopic}}

  \lhead{\small \textbf{Theory of Cryptography}\\}

  \rhead{\small \textbf{Instructor:
      \href{http://www.eecs.umich.edu/~cpeikert/}{Chris Peikert}\\Scribe:
      \scribename}}

  \setlength{\headheight}{20pt}
  \setlength{\headsep}{16pt}
}


%%% xsim settings

\RequirePackage{xsim}
\RequirePackage{needspace}

\DeclareExerciseEnvironmentTemplate{QRef}
{%
    \par\vspace{\baselineskip}
    \Needspace*{2\baselineskip}
    \noindent
    \hyperref[sol:\ExerciseID]{\textbf{\XSIMmixedcase{\GetExerciseName}~\GetExerciseProperty{counter}.}}
    \label{ques:\ExerciseID}
}{}

\DeclareExerciseEnvironmentTemplate{ARef}{%
        \Needspace*{2\baselineskip}
        \noindent
        \hyperref[ques:\ExerciseID]{\textbf{\XSIMmixedcase{\GetExerciseParameter{exercise-name}}~\GetExerciseProperty{counter}.}}
        \label{sol:\ExerciseID}
        \GetExerciseBody{exercise}
        \newline
        \textbf{\XSIMmixedcase{\GetExerciseName}.}
}{\par\vspace{\baselineskip}}

\DeclareExerciseHeadingTemplate{simple}{%
    \section*{\XSIMmixedcase{\GetExerciseParameter{solution-name}s}}
}

\DeclareExerciseType{question}{%
    exercise-env = question,
    solution-env = answer,
    exercise-name = Question,
    solution-name = Answer,
    exercise-template = QRef,
    solution-template = ARef,
}

\xsimsetup{%
  path = {questions},
  file-extension = {aux},
  print-solutions/headings-template=simple
}

\AtEndDocument{\pagebreak\printsolutions}


% VARIABLES

\newcommand{\lecturenum}{4}
\newcommand{\lecturetopic}{Number Theory, OWF Variants}
\newcommand{\scribename}{Shiva Kintali}

% END OF VARIABLES

\lecheader

\pagestyle{plain}               % default: no special header

\begin{document}

\thispagestyle{fancy}           % first page should have special header

% LECTURE MATERIAL STARTS HERE

Today we will see some concrete one-way function candidates that arise
from number theory, and abstract out some of their other special
properties that will be useful when we proceed to investigate
pseudorandomness.

\section{Collections of OWFs}
\label{sec:coll-owfs-vari}

Our generic definition of a one-way function is concise, and very
useful for complexity-theoretic crypto.  However, it tends not to be
as appropriate for the kinds of hard functions that we use in
``real-life'' crypto; below we give a more flexible definition.  (In
your homework, you will show that the generic OWF definition is
equivalent to this one.)

\begin{definition}
  \label{def:collection-owfs}
  A \emph{collection of one-way functions} is a family $F = \set{
    f_{s} \colon D_{s} \to R_{s} }_{s \in S}$ satisfying the following
  conditions:
  \begin{enumerate}
  \item \emph{Easy to sample a function:} there is a PPT algorithm
    $\algo{S}$ such that $\algo{S}()$ outputs some $s \in S$
    (according to some arbitrary distribution).
  \item \emph{Easy to sample from domain:} there is a PPT algorithm
    $\algo{D}$ such that $\algo{D}(s)$ outputs some $x \in D_{s}$
    (according to some arbitrary distribution).
  \item \emph{Easy to evaluate function:} there is a PPT algorithm
    $\algo{F}$ such that $\algo{F}(s,x)=f_{s}(x)$ for all $s \in S$,
    $x \in D_{s}$.
  \item \emph{Hard to invert:} for any non-uniform PPT algorithm
    $\Inv$, \[ \Pr_{s \gets \algo{S}(1^n), x \gets \algo{D}(s)}
    \bracks*{\Inv(s,f_{s}(x)) \in f_{s}^{-1}(f_{s}(x))} = \negl(n). \]
  \end{enumerate}
\end{definition}

For example, the subset-sum function $f_{\text{ss}}$ is more naturally
defined as a collection, as follows.  Let $S_{n} = (\Z_{N})^{n}$ where
$N=2^{n}$, and let the full index set $S = \cup_{n=1}^{\infty} S_{n}$.
Define the domain $D_{\vec{a}} = \bit^{n}$ and the range $R_{\vec{a}}
= \Z_{N}$, for all $\vec{a} = (a_{1}, \ldots, a_{n}) \in S_{n}$. The
corresponding function is defined as \[ f_{\vec{a}}(x) =
\sum_{i=1}^{n} a_{i} \cdot x_{i} \bmod N. \] The algorithms $\algo{S}$
(function sampler), $\algo{D}$ (domain sampler), and $\algo{F}$
(function evaluator) are all straightforward.

In the remainder of the lecture, we will see other examples of OWF
collections (some with other special properties) that arise from
number theory.

\section{Number Theory Background}
\label{sec:numb-theory-backgr}

\begin{definition}
   \label{def:gcd}
   For positive integers $a,b \in \N$, their \emph{greatest common
     divisor} $d = \gcd(a,b)$ is the largest integer $d$ such that $d
   \mid a$ and $d \mid b$.
\end{definition}

As a consequence of Algorithm~\ref{alg:exteuclid} below, there always
exist integers $x,y \in \Z$ such that $ax + by = \gcd(a,b)$.  We say
that $a$ and $b$ are \emph{co-prime} (or \emph{relatively prime}) if
$\gcd(a,b) = 1$, i.e., $ax = 1 \bmod b$.  From this, $x$ is the
multiplicative inverse of $a$ modulo $b$, and likewise $y$ is the
multiplicative inverse of $b$ modulo $a$.  The following deterministic
algorithm shows that $\gcd(a,b)$ (and additionally, the integers $x$
and $y$) can be computed efficiently.

\newcommand{\exteuclid}{\algo{ExtendedEuclid}}
\renewcommand{\algorithmicrequire}{\textbf{Input:}}
\renewcommand{\algorithmicensure}{\textbf{Output:}}

\begin{algorithm}
  \caption{Algorithm $\exteuclid(a,b)$ for computing the greatest
    common divisor of $a$ and $b$.}
  \label{alg:exteuclid}

  \begin{algorithmic}[1]
    \REQUIRE Positive integers $a \geq b > 0$.

    \ENSURE $(x,y) \in \Z^{2}$ such that $ax + by = \gcd(a,b)$.

    \IF {$b \mid a$}
    \RETURN $(0,1)$
    \ELSE
    \STATE Let $a = b \cdot q + r$ for $r \in \set{1, \ldots, b-1}$
    \STATE $(x',y') \gets \exteuclid(b,r)$
    \RETURN $(y', x' - q \cdot y')$
    \ENDIF
  \end{algorithmic}
\end{algorithm}

\begin{theorem}
  \label{thm:exteuclid}
  $\exteuclid$ is correct and runs in polynomial time in the
  \emph{lengths} of $a$ and $b$, i.e., in $poly(\log a + \log b)$
  time.
\end{theorem}

\begin{proof}
  For correctness, we argue by induction on the second argument $b$.
  Clearly the algorithm is correct when $b=1$.  If $b \mid a$, then
  $\gcd(a,b) = b$, hence $\exteuclid$ correctly returns $(0,1)$.  If
  $b \nmid a$ then by the inductive hypothesis (using $b > r$), the
  recursive call correctly returns $(x',y')$ such that $bx' + ry' =
  \gcd(b,r)$.  It can be checked that $\gcd(a,b) = \gcd(b,r)$, because
  any common divisor of $a$ and $b$ is also a divisor of $r$.
  Finally, observe that \[ \gcd(b,r) = bx' + ry' = bx' + (a - b \cdot
  q)y' = ay' + (x' - q \cdot y')b. \] Hence $\exteuclid$ correctly
  returns $(y', x' - q \cdot y')$.
  
  For the running time, observe that all the basic operations (not
  including the recursive call) can be implemented in polynomial
  time.  The following claim establishes the overall efficiency.
  
  \begin{claim}
    For $2^{n} > a \geq b > 0$, $\exteuclid$ makes at most $2n$
    recursive calls.
  \end{claim}

  We use induction.  The claim is true when $a < 2^{1}$.  Suppose the
  claim is true for all $a < 2^n$, and suppose $a < 2^{n+1}$.  Two
  cases arise:
  \begin{itemize}
  \item If $b < 2^n$, the first recursive call is on $(b, r)$.
    Since $b < 2^n$, by the inductive hypothesis we make at most
    $2n$ more recursive calls.  Hence the total number of recursive
    calls is at most $1 + 2n < 2(n+1)$.
  \item If $b \geq 2^n$, i.e., $2^{n+1} > a \geq b \geq 2^n$, we have
    $a = b \cdot 1 + r$ for $r = a-b < 2^n < b$.  The first recursive
    call is on $(b \geq 2^n, r < 2^n)$.  In turn, its recursive call
    uses $r < 2^n$ as its first parameter.  By the inductive
    hypothesis, the number of recursive calls following the second one
    is at most $2n$.  Hence the total number of recursive calls is at
    most $2 + 2n \leq 2(n+1)$. \qedhere
  \end{itemize}
\end{proof}

\noindent We frequently work with the ring $(\ZN, +, \cdot)$ of
integers modulo a positive integer $N$.

\begin{lemma}[Chinese remainder theorem, special case]
  Let $N = p \cdot q$ for distinct primes $p,q$.  The ring $\Z_N$ is
  isomorphic to the product ring $\Zp \times \Zq$, via the isomorphism
  $h(x) = (x \bmod p, x \bmod q)$.
\end{lemma}

\noindent A few remarks about the above lemma:
\begin{itemize}
\item In the product ring $\Zp \times \Zq$, addition and
  multiplication are coordinate-wise.
\item Clearly the isomorphism $h$ is efficiently computable.  Less
  obvious is that it is also efficiently \emph{invertible}.  Suppose
  we know some elements $c_{p}, c_{q} \in \ZN$ such that
  $h(c_{p}) = (1,0)$ and $h(c_{q}) = (0,1)$; such a pair is sometimes
  called a \emph{CRT basis}.  Then given $(x,y) \in \Zp \times \Zq$,
  it is easy to see that
  $h^{-1}(x,y) = \bar{x} \cdot c_{p} + \bar{y} \cdot c_{q}$, where
  $\bar{x}, \bar{y} \in \ZN$ are any elements such that
  $\bar{x} = x \pmod{p}$ and $\bar{y} = y \pmod{q}$. (For example, we
  could take the ``smallest'' elements in~$\ZN$ that have the required
  residues modulo~$p$ and~$q$, respectively.)
\end{itemize}

\begin{question}
  For the CRT isomorphism on $\Z_{15}$, verify that the following
  example equations hold: $h(7 \cdot 9) = h(7) \cdot h(9)$ and
  $h(6 + 11) = h(6) + h(11)$.
\end{question}

\begin{answer}
  Note that $15$ is the product of two primes, $3$ and $5$, so
  \(\Z_{15} \cong \Z_3 \times \Z_5\).

  First, we consider \(h(7 \cdot 9)\). We have
  \(7 \cdot 9 = 3 \pmod{15}\) and
  \(h(3) = (3 \bmod 3, 3 \bmod 5) = (0, 3)\). Next,
  \(h(7) = (7 \bmod{3}, 7 \bmod{5}) = (1,2)\) and
  \(h(9) = (9 \bmod{3}, 9 \bmod 5) = (0, 4)\). Finally, we multiply
  the pairs elementwise, recalling that the first elements of each
  pair are from~\(\Z_3\) and the second elements are from~\(\Z_5\). We
  see that \((1,2) \cdot (0,4) = (0 \bmod 3, 8 \bmod 5) = (0,3)\), as
  expected.

  Now we consider \(h(6 + 11)\). We have \(6 + 11 = 2 \pmod{15}\) and
  \(h(2) = (2 \bmod 3, 2 \bmod 5) = (2,2)\). Next,
  \(h(6) = (6 \bmod{3}, 6 \bmod{5}) = (0,1)\) and
  \(h(11) = (11 \bmod{3}, 11 \bmod{5}) = (2,1)\). Finally, adding the
  pairs elementwise (and reducing modulo the appropriate moduli)
  yields \((0,1) + (2,1) = (2,2)\), as expected.
\end{answer}

\begin{question}
  Show that \(h^{-1}(x,y) = \bar{x} \cdot c_p + \bar{y} \cdot c_q\),
  as claimed above.
\end{question}

\begin{answer}
  We use the CRT isomorphism several times:
  \begin{align*}
    (x,y) &= (x, 0) + (0, y) \\
          &= h(\bar{x}) \cdot (1,0) + h(\bar{y}) \cdot (0,1) \\
          &= h(\bar{x}) \cdot h(c_p) + h(\bar{y}) \cdot h(c_q) \\
          &= h(\bar{x} \cdot c_p + \bar{y} \cdot c_q),
  \end{align*}
  and the result follows by applying the bijection~$h^{-1}$ to both
  sides.
\end{answer}

\begin{question}
  Show how to compute $c_{p}, c_{q}$ efficiently (hint: use
  $\exteuclid$ on $p,q$).
\end{question}

\begin{answer}
  Because $p,q$ are distinct primes, their gcd is 1, so
  $\exteuclid(p,q)$ returns integers $a,b \in \Z$ such that
  $a \cdot p + b \cdot q = 1$. We claim that
  $c_{p} = b \cdot q \bmod{p}$ and $c_{q} = a \cdot p \bmod{q}$. To
  see this, observe that $b \cdot q = 1 - a \cdot p = 1 \pmod{p}$
  (because~$a$ is an integer) and $b \cdot q = 0 \pmod{q}$
  (because~$b$ is an integer), so $h(b \cdot q) = (1,0)$, as needed. A
  similar calculation shows that $h(a \cdot p) = (0,1)$.
\end{answer}

\begin{definition}
  The multiplicative group $\ZN^{*} := \set{ x \in \Z_N : x \text{
      is invertible mod $N$, i.e., } \gcd(x,N) = 1 }$.
\end{definition}

\noindent Here are some useful facts about the multiplicative group
$\ZN^{*}$:
\begin{itemize}
\item For a prime $p$, $\Zp^{*} = \set{1, \ldots, p-1}$.
\item When $N=pq$ for distinct primes $p,q$, we have $\Z^{*}_N \cong
  \Zp^{*} \times \Zq^{*}$.
\end{itemize}

\begin{definition}
  For $N \in \Z^+$, Euler's \emph{totient function} $\varphi(n)$ is
  defined to be $\abs{\ZN^{*}}$, i.e., the number of positive integers
  $a \leq n$ relatively prime to $n$.
\end{definition}

\noindent Here are some useful facts about the totient function:
\begin{itemize}
\item For a prime $p$, we have $\varphi(p) = p - 1$.
\item For a prime $p$ and positive integer $a$, we have $\varphi(p^a)
  = (p-1) p^{a-1} = p^a - p^{a-1}$.
\item If $gcd(a,b) = 1$, then $\varphi(a \cdot b) = \varphi(a) \cdot
  \varphi(b)$.
\end{itemize}

\begin{definition}
  The subgroup of \emph{quadratic residues} is defined as \[ \QRN^{*}
  = \set{y \in \ZN^{*} : \exists\; x \in \ZN^{*} \text{ s.t. } y = x^2
    \bmod N} \subseteq \ZN^{*}. \]
\end{definition}

\noindent Here are some useful facts about $\QRN^{*}$:
\begin{itemize}
\item For an odd prime $p$, $\abs{\QRp^{*}} = \frac{p-1}{2}$, because
  $x \mapsto x^{2}$ is $2$-to-$1$ over $\Zp^{*}$.  (Exercise: prove
  this.)

\item When $N = pq$ for distinct odd primes $p,q$, we have $\QRN^{*}
  \cong \QRp^{*} \times \QR_{q}^{*}$, hence $\abs{\QRN^{*}} =
  \frac{p-1}{2} \cdot \frac{q-1}{2}$.

\item For an odd prime $p$, we have $-1 \in \QRp^{*}$ if and only if
  $p = 1 \bmod 4$.
\end{itemize}

\section{Factoring-Related Functions}
\label{sec:fact-related}

We can abstract out a modulus generation algorithm $\algo{S}$, which
given the security parameter $1^{n}$ outputs the product $N$ of two
primes $p,q$.  For example, $\algo{S}$ might choose $p$ and $q$ to be
uniformly random and independent $n$-bit primes.

\emph{Rabin's function} $f_{N} \colon \ZN^{*} \to \QRN^{*}$ is defined
as follows: \[ f_{N}(x) = x^{2} \bmod N. \] Precisely defining the
collection according to Definition~\ref{def:collection-owfs} is a
simple exercise.  Note that $f_{N}$ is $4$-to-$1$, because each $y \in
\QRN^{*}$ has two square roots modulo $p$, and two modulo $q$.

\begin{theorem}
  \label{thm:rabin-owf-factor}
  If factoring is hard (with respect to $\algo{S}$), then the Rabin
  collection (with function generator $\algo{S}$) is one-way.
\end{theorem}

\begin{proof}
  First, as already discussed it is easy to generate a function,
  sample its domain, and evaluate the function.  The main fact we use
  to prove one-wayness is the following.

  \begin{claim}
    \label{claim:roots-factor}
    Let $N = pq$ be the product of distinct odd primes.  Given any
    $x_{1}, x_{2} \in \ZN^{*}$ such that $x_{1}^{2} = x_{2}^{2} \bmod
    N$ but $x_{1} \neq \pm x_{2} \bmod N$, the factors of $N$ can be
    computed efficiently.
  \end{claim}

  \begin{proof}[Proof of Claim]
    We have $x_{1}^{2} = x_{2}^{2} \bmod p$ and $x_{1}^{2} = x_{2}^{2}
    \bmod q$, which implies $x_{1} = \pm x_{2} \bmod p$ and $x_{1} =
    \pm x_{2} \bmod q$.  But we cannot have both $+$ or both $-$, by
    assumption.  Wlog, we have $x_{1} = + x_{2} \bmod p$ and $x_{1} =
    - x_{2} \bmod q$.  Then $p \mid (x_{1}-x_{2})$ but $q \nmid
    (x_{1}-x_{2})$, otherwise we'd have $q \mid (2x_{2}) \Rightarrow q
    \mid x_{2} \Rightarrow x_{2} \not\in \ZN^{*}$.  Then
    $\gcd(x_{1}-x_{2},N) = p$, which we can compute efficiently.
  \end{proof}

  Continuing with the proof of Theorem~\ref{thm:rabin-owf-factor}, we
  prove one-wayness by contrapositive, via a reduction.  Assuming we
  have an inverter for the Rabin function, the idea is to choose our
  own $x_{1} \in \ZN^{*}$ and invoke the inverter on $y = f_{N}(x_{1})
  = x_{1}^{2} \bmod N$.  The square root $x_{2}$ it returns will be
  $\neq \pm x_{1}$, with probability $1/2$.  In such a case, we get
  the prime factorization of $N$ by Claim~\ref{claim:roots-factor}.
  We now proceed more formally.

  Assume a non-uniform PPT inverter $\Inv$ violating the one-wayness
  of the Rabin collection, i.e., \[ \Pr_{N \gets \algo{S}(1^{n}), x
    \gets \ZN^{*}}[\Inv(N, y = x^{2} \bmod N) \in \sqrt{y} \bmod N] =
  \delta(n) \] is non-negligible.

  Our factoring algorithm $\Adv(N)$ works as follows: first, generate
  a uniform $x_{1} \gets \ZN^{*}$.  Let $y = x_{1}^{2} \bmod N$ and
  let $x_{2} \gets \Inv(N, y)$.  If $x_{2}^{2} = y \bmod N$ but $x_{1}
  \neq \pm x_{2} \bmod N$, then compute the factorization of $N$ by
  Claim~\ref{claim:roots-factor}.

  We now analyze the reduction.  First, $N$ and $y$ are distributed as
  expected, so $\Inv$ outputs $x_{2}$ such that $x_{2}^{2} = y \bmod
  N$ with probability $\delta$.  Conditioned on the fixed value of
  $y$, there are four possible values for $x_{1}$, each equally likely
  by construction.  So we have $x_{2}^{2} = y \bmod N$ and $x_{2} \neq
  \pm x_{1} \bmod N$ with prob $\delta/2$, which is non-negligible by
  assumption.
\end{proof}

Suppose $p,q = 3 \bmod 4$.  Then $-1$ is not a square modulo $p$
(respectively, $q$).  So for any $x \in \Zp^{*}$ (resp., $\Zq^{*}$),
exactly one of $\pm x$ is a square modulo $p$ (resp., $q$).  From this
it can be seen that if we restrict the Rabin function to have domain
$\QRN^{*}$, i.e., $f_{N} \colon \QRN^{*} \to \QRN^{*}$, it becomes a
\emph{permutation} (bijection).

\medskip
\noindent {\bf{Question}}: Our proof that $f_{N}$ is one-way used
(quite essentially) the fact that $f_{N}$ is $4$-to-$1$.  Now that we
have changed its domain to make $f_{N}$ a permutation, is the proof
still valid?

\begin{definition}[One-Way Permutation]
  A collection $F = \set{f_{s} \colon D_{s} \to D_{s}}_{s \in S}$ is a
  collection of \emph{one-way permutations} if it is a collection of
  one-way functions $f_{s}$ under the \emph{uniform} distribution over
  $D_{s}$, and each $f_{s}$ is a \emph{permutation} of $D_{s}$ (i.e.,
  a bijection).
\end{definition}

\end{document}

%%% Local Variables:
%%% mode: latex
%%% TeX-master: t
%%% End:
