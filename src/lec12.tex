\documentclass[11pt]{article}

\usepackage{fullpage,microtype,pdfsync,hyperref}
\usepackage{newtxtext}
\usepackage[T1]{fontenc}

\usepackage{mathtools,amsfonts,amssymb}
\usepackage[amsmath,amsthm,thmmarks,hyperref]{ntheorem}
\usepackage[capitalize,nameinlink,noabbrev]{cleveref}

\RequirePackage{mathtools}

%%% BLACKBOARD SYMBOLS

% hyperref package defines C and G, undefined to avoid conflicts
\let\C\relax
\let\G\relax
\newcommand{\C}{\ensuremath{\mathbb{C}}}
\newcommand{\D}{\ensuremath{\mathbb{D}}}
\newcommand{\F}{\ensuremath{\mathbb{F}}}
\newcommand{\G}{\ensuremath{\mathbb{G}}}
\newcommand{\J}{\ensuremath{\mathbb{J}}}
\newcommand{\N}{\ensuremath{\mathbb{N}}}
\newcommand{\Q}{\ensuremath{\mathbb{Q}}}
\newcommand{\R}{\ensuremath{\mathbb{R}}}
\newcommand{\T}{\ensuremath{\mathbb{T}}}
\newcommand{\Z}{\ensuremath{\mathbb{Z}}}
\newcommand{\QR}{\ensuremath{\mathbb{QR}}}

\newcommand{\Zt}{\ensuremath{\Z_t}}
\newcommand{\Zp}{\ensuremath{\Z_p}}
\newcommand{\Zq}{\ensuremath{\Z_q}}
\newcommand{\ZN}{\ensuremath{\Z_N}}
\newcommand{\Zps}{\ensuremath{\Z_p^*}}
\newcommand{\ZNs}{\ensuremath{\Z_N^*}}
\newcommand{\JN}{\ensuremath{\J_N}}
\newcommand{\QRN}{\ensuremath{\QR_{N}}}
\newcommand{\QRp}{\ensuremath{\QR_{p}}}

%%% THEOREM COMMANDS

\RequirePackage{amsthm}

\theoremstyle{plain}            % following are "theorem" style
\newtheorem{theorem}{Theorem}[section]
\newtheorem{lemma}[theorem]{Lemma}
\newtheorem{corollary}[theorem]{Corollary}
\newtheorem{proposition}[theorem]{Proposition}
\newtheorem{claim}[theorem]{Claim}
\newtheorem{fact}[theorem]{Fact}

\theoremstyle{definition}       % following are def style
\newtheorem{definition}[theorem]{Definition}
\newtheorem{conjecture}[theorem]{Conjecture}
\newtheorem{example}[theorem]{Example}
\newtheorem{protocol}[theorem]{Protocol}

\theoremstyle{remark}           % following are remark style
\newtheorem{remark}[theorem]{Remark}
\newtheorem{note}[theorem]{Note}

% equation numbering style
\numberwithin{equation}{section}

%%% GENERAL COMPUTING

\newcommand{\bit}{\ensuremath{\set{0,1}}}
\newcommand{\pmone}{\ensuremath{\set{-1,1}}}

% asymptotics
\DeclareMathOperator{\poly}{poly}
\DeclareMathOperator{\polylog}{polylog}
\DeclareMathOperator{\negl}{negl}
\newcommand{\Otil}{\ensuremath{\tilde{O}}}

% probability/distribution stuff
\DeclareMathOperator*{\E}{E}
\DeclareMathOperator*{\Var}{Var}

% sets in calligraphic type
\newcommand{\calD}{\ensuremath{\mathcal{D}}}
\newcommand{\calF}{\ensuremath{\mathcal{F}}}
\newcommand{\calG}{\ensuremath{\mathcal{G}}}
\newcommand{\calH}{\ensuremath{\mathcal{H}}}
\newcommand{\calX}{\ensuremath{\mathcal{X}}}
\newcommand{\calY}{\ensuremath{\mathcal{Y}}}

% types of indistinguishability
\newcommand{\compind}{\ensuremath{\stackrel{c}{\approx}}}
\newcommand{\statind}{\ensuremath{\stackrel{s}{\approx}}}
\newcommand{\perfind}{\ensuremath{\equiv}}

% font for general-purpose algorithms
\newcommand{\algo}[1]{\ensuremath{\mathsf{#1}}}
% font for general-purpose computational problems
\newcommand{\problem}[1]{\ensuremath{\mathsf{#1}}}
% font for complexity classes
\newcommand{\class}[1]{\ensuremath{\mathsf{#1}}}

% complexity classes and languages
\renewcommand{\P}{\class{P}}
\newcommand{\BPP}{\class{BPP}}
\newcommand{\NP}{\class{NP}}
\newcommand{\coNP}{\class{coNP}}
\newcommand{\AM}{\class{AM}}
\newcommand{\coAM}{\class{coAM}}
\newcommand{\IP}{\class{IP}}

%%% "LEFT-RIGHT" PAIRS OF SYMBOLS


% inner product
\DeclarePairedDelimiter\inner{\langle}{\rangle}
% absolute value
\DeclarePairedDelimiter\abs{\lvert}{\rvert}
% a set
\DeclarePairedDelimiter\set{\{}{\}}
% parens
\DeclarePairedDelimiter\parens{(}{)}
% tuple, alias for parens
\DeclarePairedDelimiter\tuple{(}{)}
% square brackets
\DeclarePairedDelimiter\bracks{[}{]}
% rounding off
\DeclarePairedDelimiter\round{\lfloor}{\rceil}
% floor function
\DeclarePairedDelimiter\floor{\lfloor}{\rfloor}
% ceiling function
\DeclarePairedDelimiter\ceil{\lceil}{\rceil}
% length of some vector, element
\DeclarePairedDelimiter\length{\lVert}{\rVert}
% "lifting" of a residue class
\DeclarePairedDelimiter\lift{\llbracket}{\rrbracket}
\DeclarePairedDelimiter\len{\lvert}{\rvert}

%%% CRYPTO-RELATED NOTATION

% KEYS AND RELATED

\newcommand{\key}[1]{\ensuremath{#1}}

\newcommand{\pk}{\key{pk}}
\newcommand{\vk}{\key{vk}}
\newcommand{\sk}{\key{sk}}
\newcommand{\mpk}{\key{mpk}}
\newcommand{\msk}{\key{msk}}
\newcommand{\fk}{\key{fk}}
\newcommand{\id}{id}
\newcommand{\keyspace}{\ensuremath{\mathcal{K}}}
\newcommand{\msgspace}{\ensuremath{\mathcal{M}}}
\newcommand{\ctspace}{\ensuremath{\mathcal{C}}}
\newcommand{\tagspace}{\ensuremath{\mathcal{T}}}
\newcommand{\idspace}{\ensuremath{\mathcal{ID}}}

\newcommand{\concat}{\ensuremath{\|}}

% GAMES

% advantage
\newcommand{\advan}{\ensuremath{\mathbf{Adv}}}

% different attack models
\newcommand{\attack}[1]{\ensuremath{\text{#1}}}

\newcommand{\atk}{\attack{atk}} % dummy attack
\newcommand{\indcpa}{\attack{ind-cpa}}
\newcommand{\indcca}{\attack{ind-cca}}
\newcommand{\anocpa}{\attack{ano-cpa}} % anonymous
\newcommand{\anocca}{\attack{ano-cca}}
\newcommand{\euacma}{\attack{eu-acma}} % forgery: adaptive chosen-message
\newcommand{\euscma}{\attack{eu-scma}} % forgery: static chosen-message
\newcommand{\suacma}{\attack{su-acma}} % strongly unforgeable

% ADVERSARIES
\newcommand{\attacker}[1]{\ensuremath{\mathcal{#1}}}

\newcommand{\Adv}{\attacker{A}}
\newcommand{\AdvA}{\attacker{A}}
\newcommand{\AdvB}{\attacker{B}}
\newcommand{\Dist}{\attacker{D}}
\newcommand{\Sim}{\attacker{S}}
\newcommand{\Ora}{\attacker{O}}
\newcommand{\Inv}{\attacker{I}}
\newcommand{\For}{\attacker{F}}

% CRYPTO SCHEMES

\newcommand{\scheme}[1]{\ensuremath{\text{#1}}}

% pseudorandom stuff
\newcommand{\prg}{\algo{PRG}}
\newcommand{\prf}{\algo{PRF}}
\newcommand{\prp}{\algo{PRP}}

% symmetric-key cryptosystem
\newcommand{\skc}{\scheme{SKC}}
\newcommand{\skcgen}{\algo{Gen}}
\newcommand{\skcenc}{\algo{Enc}}
\newcommand{\skcdec}{\algo{Dec}}

% public-key cryptosystem
\newcommand{\pkc}{\scheme{PKC}}
\newcommand{\pkcgen}{\algo{Gen}}
\newcommand{\pkcenc}{\algo{Enc}} % can also use \kemenc and \kemdec
\newcommand{\pkcdec}{\algo{Dec}}

% digital signatures
\newcommand{\sig}{\scheme{SIG}}
\newcommand{\siggen}{\algo{Gen}}
\newcommand{\sigsign}{\algo{Sign}}
\newcommand{\sigver}{\algo{Ver}}

% message authentication code
\newcommand{\mac}{\scheme{MAC}}
\newcommand{\macgen}{\algo{Gen}}
\newcommand{\mactag}{\algo{Tag}}
\newcommand{\macver}{\algo{Ver}}

% key-encapsulation mechanism
\newcommand{\kem}{\scheme{KEM}}
\newcommand{\kemgen}{\algo{Gen}}
\newcommand{\kemenc}{\algo{Encaps}}
\newcommand{\kemdec}{\algo{Decaps}}

% identity-based encryption
\newcommand{\ibe}{\scheme{IBE}}
\newcommand{\ibesetup}{\algo{Setup}}
\newcommand{\ibeext}{\algo{Ext}}
\newcommand{\ibeenc}{\algo{Enc}}
\newcommand{\ibedec}{\algo{Dec}}

% hierarchical IBE (as key encapsulation)
\newcommand{\hibe}{\scheme{HIBE}}
\newcommand{\hibesetup}{\algo{Setup}}
\newcommand{\hibeext}{\algo{Extract}}
\newcommand{\hibeenc}{\algo{Encaps}}
\newcommand{\hibedec}{\algo{Decaps}}

% binary tree encryption (as key encapsulation)
\newcommand{\bte}{\scheme{BTE}}
\newcommand{\btesetup}{\algo{Setup}}
\newcommand{\bteext}{\algo{Extract}}
\newcommand{\bteenc}{\algo{Encaps}}
\newcommand{\btedec}{\algo{Decaps}}

% trapdoor functions
\newcommand{\tdf}{\scheme{TDF}}
\newcommand{\tdfgen}{\algo{Gen}}
\newcommand{\tdfeval}{\algo{Eval}}
\newcommand{\tdfinv}{\algo{Invert}}
\newcommand{\tdfver}{\algo{Ver}}

%%% PROTOCOLS

\newcommand{\out}{\text{out}}
\newcommand{\view}{\text{view}}

%%% COMMANDS FOR LECTURES/HOMEWORKS

\RequirePackage{fancyhdr}

\newcommand{\lecheader}{%
  \chead{\large \textbf{Lecture \lecturenum\\\lecturetopic}}

  \lhead{\small \textbf{Theory of Cryptography}\\}

  \rhead{\small \textbf{Instructor:
      \href{http://www.eecs.umich.edu/~cpeikert/}{Chris Peikert}\\Scribe:
      \scribename}}

  \setlength{\headheight}{20pt}
  \setlength{\headsep}{16pt}
}


%%% xsim settings

\RequirePackage{xsim}
\RequirePackage{needspace}

\DeclareExerciseEnvironmentTemplate{QRef}
{%
    \par\vspace{\baselineskip}
    \Needspace*{2\baselineskip}
    \noindent
    \hyperref[sol:\ExerciseID]{\textbf{\XSIMmixedcase{\GetExerciseName}~\GetExerciseProperty{counter}.}}
    \label{ques:\ExerciseID}
}{}

\DeclareExerciseEnvironmentTemplate{ARef}{%
        \Needspace*{2\baselineskip}
        \noindent
        \hyperref[ques:\ExerciseID]{\textbf{\XSIMmixedcase{\GetExerciseParameter{exercise-name}}~\GetExerciseProperty{counter}.}}
        \label{sol:\ExerciseID}
        \GetExerciseBody{exercise}
        \newline
        \textbf{\XSIMmixedcase{\GetExerciseName}.}
}{\par\vspace{\baselineskip}}

\DeclareExerciseHeadingTemplate{simple}{%
    \section*{\XSIMmixedcase{\GetExerciseParameter{solution-name}s}}
}

\DeclareExerciseType{question}{%
    exercise-env = question,
    solution-env = answer,
    exercise-name = Question,
    solution-name = Answer,
    exercise-template = QRef,
    solution-template = ARef,
}

\xsimsetup{%
  path = {questions},
  file-extension = {aux},
  print-solutions/headings-template=simple
}

\AtEndDocument{\pagebreak\printsolutions}


% VARIABLES

\newcommand{\lecturenum}{12}
\newcommand{\lecturetopic}{CCA, Digital Signatures}
\newcommand{\scribename}{Rajiv Iyer}

% END OF VARIABLES

\lecheader

\pagestyle{plain}               % default: no special header

\begin{document}

\thispagestyle{fancy}           % first page should have special header

% LECTURE MATERIAL STARTS HERE

\section{Chosen-Ciphertext Attacks}
\label{sec:chos-ciph-attacks}

\subsection{Motivation}
\label{sec:cca-motivation}

In the preceding lectures, our notion of security for encryption was
constrained to work against so-called \emph{passive} adversaries.
Such adversaries are permitted to see the ciphertexts for messages of
their choice, and (in the public-key setting) to generate ciphertexts
on their own.  However, the adversary never gets to see the
\emph{decryptions} of any messages.  For some applications, we will
need a stronger definition in which the adversary gets (limited)
access to the decryption machinery as well.

To establish the motivation for this stricter definition, imagine an
auction scenario, where Alice and Eve are bidders for some item that
Bob has for sale.  Alice is the honest participant, while Eve is the
malicious adversary who tries to outbid Alice by the smallest amount
possible.  Consider how it might work:
\begin{itemize}
\item Alice encrypts her bid $b_{A}$ under a key $k$, and sends the
  resulting ciphertext $c$ to Bob.
\item Eve intercepts the ciphertext $c$ and modifies some bit(s),
  dispatching the modified ciphertext $c'$ to Bob.  To Bob, $c'$ looks
  like a ciphertext and is treated as such.  Bob decrypts Alice's and
  Eve's ciphertexts, yielding bids $b_{A}$ and $b_{E}$, and decides who
  has bid the highest.
\item To the highest bidder, Bob sends a message to the effect of
  ``Congratulations, you won with a bid of $b$!'' where $b = b_A$ or
  $b=b_E$, whichever is larger.
\end{itemize}

Suppose that the encryption scheme used in this scenario allows Eve to
modify the ciphertext $c$, so that the result $c'$ encrypts a message
that is related (in a predictable way) to $b_{A}$, \emph{even though
  Eve does not know what $b_{A}$ is}?  For example, what if Eve can
increase the underlying bid by $1$?  Then she will always win the
auction with the lowest possible bid, \emph{and} she will learn the
value of Alice's bid!

It turns out that many encryption schemes are ``malleable'' in the
above-described way.  For example, consider our PRF-based scheme,
where $c = (r , f_{k}(r) \oplus m)$.  Flipping any bit of the second
component causes the same bit in the underlying message to flip.  So
if Alice's bid happened to be even, Eve can increment it by $1$ by
flipping the least significant bit.  (If Even suspects that Alice's
bid is odd, she can flip other bits to increase it by $1$ or more.)
%In the public-key setting, textbook RSA is similarly vulnerable: given
%Alice's ciphertext $c = b_{A}^{e} \bmod N$, Eve can (for example)
%double Alice's bid with $2^{e} \cdot c = (2b_{A})^{e} \bmod N$.

The example illustrates some fine points:
\begin{itemize}
\item Eve is \emph{generating} ciphertexts of her own, not just seeing
  ciphertexts that come from an encryption oracle.  She is mounting a
  so-called ``chosen-ciphertext attack'' on the scheme.
\item Bob is taking some action based on the result of
  $\skcdec_{k}(c)$.  In this case, when Bob sends his congratulations
  message, he is even acting as a \emph{decryption oracle}!
\end{itemize}

These issues are not just philosophical curiosities.  PKCS\#1 is a
family of public-key cryptographic standards implemented by web
servers and browsers, smart cards, etc.  In 1998, Bleichenbacher
demonstrated a ``chosen-ciphertext'' attack against the PKCS\#1
standard, which queried the decryption device (e.g., an SSL-equipped
web server) on a large number of specially crafted ciphertexts, which
gradually revealed information to the attacker.  In practical terms,
this meant that an SSL session key could be exposed in a reasonable
amount of time, perhaps a day or less.  (PKCS\#1 has since been
updated with a ``chosen ciphertext-secure'' protocol that withstands
such attacks.)

\subsection{Definition}
\label{sec:cca-definition}

We seek a very conservative definition that ensures security
\emph{even if the attacker is given access to a decryption oracle}.
Of course, the attacker can learn the underlying message of any
ciphertext simply by querying its decryption oracle, so how can we
hope to achieve any meaningful notion of secrecy?  The answer is that
we slightly weaken the decryption oracle (in one of a couple of
possible ways) to prevent the adversary from querying it on the
challenge ciphertext.

\begin{definition}
  \label{def:skc-cca}
  We say that $\skc$ is secure under \emph{chosen-ciphertext attack}
  (IND-CCA-secure) if the following tuple of oracles is
  indistinguishable for $b=0,1$:
  \[ \langle k \gets \skcgen : \skcenc_{k}(\cdot),
  \algo{D}_{k}(\cdot), \algo{C}_{k}^{b}(\cdot, \cdot) \rangle, \]
  where $\algo{C}_{k}(m_{0},m_{1})$ is the usual challenge oracle that
  outputs $c^{*} = \skcenc_{k}(m_{b})$ and terminates, and
  $\algo{D}_{k}(c)$ is an oracle that outputs $\bot$ if $c = c^{*}$
  (the challenge ciphertext), else it outputs $\skcdec_{k}(c)$.

  We say that $\skc$ is IND-CCA1-secure (or secure under ``lunchtime''
  attacks) if the above holds where $\algo{D}_{k}$ is a decryption
  oracle that answers queries \emph{before} $\algo{C}_{k}$ is invoked,
  then terminates (i.e., it does not answer queries after the
  challenge ciphertext $c^{*}$ has been produced).
\end{definition}

(``Lunchtime'' security is weaker than full IND-CCA security.  The
name refers to the following whimsical scenario: Eve breaks into Bob's
office while Bob is out to lunch, thus temporarily gaining access to
his decryption machinery.  She wants to formulate a sequence of
queries so that later on, when Bob comes back from lunch and she no
longer has a decryption oracle, she can violate the secrecy of
encrypted messages sent to/from Bob.)

\subsection{Achieving CCA Security}
\label{sec:achi-cca-secur}

Our PRF-based scheme is not CCA-secure, as it is vulnerable to the
bit-flipping attack described above.  

To design a CCA-secure scheme, we wish to use the notion of
\emph{invalid ciphertexts}, i.e., we allow the decryption algorithm
$\skcdec_{k}(\cdot)$ to output $\bot$ on ciphertexts that don't look
``well-formed.''  The intuition is to design the scheme so that the
only way an adversary can obtain a well-formed ciphertext is by
directly querying the encryption oracle.  This way, it already
``knows'' the underlying message, hence the decryption oracle is
``useless!''  (This is just the intuition for how we think about
designing a CCA-secure scheme.)

It can be shown that in the shared-key setting, combining a message
authentication code ($\mac$) with an IND-CPA-secure in the right way
yields Authenticated Encryption (AE), a concept from last lecture.
Moreover, it can be proved that an Authenticated Encryption (AE)
scheme is, in fact, IND-CCA-secure.  (You will prove these facts in
your homework.)

\begin{question}
  Can textbook RSA ciphertexts be manipulated with predictable
  consequences to the underlying plaintexts (similar to the PRF-based
  scheme)? Recall that RSA encryption takes a plaintext \(m \in \Z_n\)
  (where \(n = pq\) for primes \(p\), \(q\)) and returns
  \(m^e \pmod {n}\).
\end{question}

\begin{answer}
  Yes. Multiplying a ciphertext that encrypts a message \(m\) by
  \(x^e\) yields a ciphertext that encrypts \(m \cdot x\):
  \[ \pkcenc(m) \cdot x^e \equiv m^e \cdot x^e \equiv (mx)^e \equiv
    \pkcenc(mx) .
  \]
\end{answer}

\begin{question}
  How does the ``bit-flipping'' trick for our PRF-based encryption
  scheme violate our precise definition of CCA security?
\end{question}

\begin{answer}
  Let \(m\) be any \(n-1\) bitstring. On input \(k\) an attacker
  simply queries \(c^* \gets C_k^b(1\concat m, 0 \concat m)\), then
  queries the decryption oracle on \(\overline{c^*}\) and outputs the
  first bit of the resulting plaintext. Note that
  \(\overline{c^*} \neq c^*\) so the decryption oracle does not output
  \(\bot\).  Since \(D_k(\overline{c^*} = \overline{D_k(c^*)})\) by
  the ``bit-flipping'' trick, the first bit of the resulting plaintext
  will be 0 if \(c^*\) is an encryption of \(1 \concat m\) and 1
  otherwise.
\end{answer}

\section{Digital Signatures}
\label{sec:digital-signatures}

We now switch back to message authentication itself.  Just as with
encryption, a natural question is whether the tagging algorithm
$\mactag$ and verification algorithm $\macver$ must use the
\emph{same} key, or could they be \emph{asymmetric}?  That is, is it
possible for verification to be performed by anyone (using a public
verification key), while signing can be performed by only one party?

Digital signatures are the asymmetric analogue of $\mac$s. For
messages sent over an insecure channel, a properly implemented digital
signature gives the receiver reason to believe the message was sent by
the claimed sender.

\subsection{Model}
\label{sec:model}

Very similarly to a $\mac$, a digital signature scheme $\sig$ is a
triple of algorithms:

\begin{itemize}
\item $\siggen$: outputs a (public) verification key $\vk$ and a
  (secret) signing key $\sk$.
\item $\sigsign_{\sk}(m) = \sigsign(\sk,m)$: given a message $m \in
  \msgspace$, outputs a signature $\sigma$.
\item $\sigver_{\vk}(m,\sigma) = \sigver(\vk,m,\sigma)$: given a
  message $m$ and signature $\sigma$, either accepts or rejects.
\end{itemize}

\subsection{Definition}
\label{sec:uf-cma-definition}

We seek a security definition in which the adversary (forger) may see
signatures for messages that it selects arbitrarily (hence
\textit{chosen-message-attack} or CMA), but still cannot forge a
signature for another message.  Therefore, we give the adversary
oracle access to $\sigsign_{\sk}(\cdot)$.  Moreover, the forger should
explicitly be given the signature verification key $\vk$, because it
is public information.

\begin{definition}
  \label{def:uf-cma-sig}
  We say that $\sig$ is \emph{(strongly) unforgeable under
    chosen-message-attack} (UF-CMA-secure) if for every nuppt $\For$,
  \[ \advan_{\sig}(\For) := \Pr_{(\vk,\sk) \gets \siggen}
  \bracks*{\For^{\sigsign_{\sk}(\cdot)}(\vk) \text{ succeeds}} \leq
  \negl(n), \] where ``succeeds'' means that $\For$ outputs $(m' ,
  \sigma')$ such that $\sigver_{v_{k}}(m' , \sigma') = 1$, and either
  \begin{itemize}
  \item \emph{Standard} unforgeability: $m'$ differs from all queries
    $m_{i}$ to the $\sigsign$ oracle;
  \item \emph{Strong} unforgeability: $(m', \sigma')$ differs from all
    query/response pairs $(m_{i}, \sigma_{i})$ of the $\sigsign$
    oracle.
  \end{itemize}
\end{definition}

As with a $\mac$, the definition of \emph{strong} unforgeability
corresponds to a looser notion of success, implying a stricter
definition of security.

\subsection{A (Failed) Proposal}
\label{sec:failed-proposal}

We sketch out a scheme with the view of making it UF-CMA-secure, but
which we will show fails to live up the definition.  The scheme is
based on a family of trapdoor one-way permutations.  Recall that a TDP
family is a family $\calF = \set{f_{s} : D_{s} \to D_{s}}$ of one-way
permutations, where the function sampler $\algo{S}$ outputs, along
with $f_{s} \gets \calF$, a \emph{trapdoor} $t$ that makes it easy to
compute $f_{s}^{-1}$.  The intuition is that the signer can use the
trapdoor to compute a signature by inverting $f_{s}$ on the message,
which the verifier can check by computing $f_{s}$ in the ``forward''
direction.

Formally, we define the following triplet of algorithms:
\begin{itemize}
\item $\siggen$: choose $(s,t) \gets \algo{S}$.  Output $\vk = s$ and
  $\sk = t$. 
\item $\sigsign_{t} (m \in D_{s})$: output $\sigma = f_{s}^{-1}(m)$.
\item $\sigver_{s}(m , \sigma)$: accept if $f_{s}(\sigma) = m$,
  otherwise reject.
\end{itemize}

The scheme is correct by inspection.  And $f_{s}$ is ``hard to
invert'' by assumption, so why isn't the scheme secure according to
\cref{def:uf-cma-sig}?  The key point is that $f_{s}$ is
hard to invert on a \emph{random} value $y \in D_{s}$, but a forger
gets to \emph{choose} the message on which it inverts $f_{s}$!  In
fact, observe that there is a trivial attack on the above scheme
(which doesn't even make use of a signing oracle): given $\vk = s$,
choose some $\sigma \in D_{s}$, let $m = f_{s}(\sigma)$, and output
$(m,\sigma)$ as a forgery.  Clearly, $\sigver_{s}$ accepts this pair!

Designing a truly UF-CMA-secure signature scheme is quite tricky, and
involves several steps.  We will see how to do so in the next lecture.

\end{document}

\subsection{A One-Time Signature}
\label{sec:one-time-signature}

We suggest a "one time" secure signature scheme based on one way
functions. We constrain the $\sigsign_{s_{k}}(\cdot)$ oracle to be
able to answer \emph{only one} query. We describe the construction
based on an OWF $f$ over a message space $M = \bit^{n}$ as the
following triplet of algorithms:

\begin{itemize}
\item $\siggen$ works as follows:
  \begin{itemize}
  \item for $i \in [n]$ and $b \in \bit$, we choose $X_{i}^{b}
    \leftarrow \bit^{n}$ and let $Y_{i}^{b} = f(X_{i}^{b})$.
  \item We select $s_{k} = \set{X_{i}^{b}}$ and $v_{k} =
    \set{Y_{i}^{b}}$
  \end{itemize}
\item $\sigsign (s_{k} , m \in \bit^{n})$ ouputs $\sigma =
  (X_{i}^{m_{i}})$ where $\sigma$ is a tuple in $\set{\bit^{n}}^{n}$
\item $\sigver (v_{k} , m , \sigma)$ checks $f(\sigma_{i}) =
  Y_{i}^{m_{i}}$ , $\forall i \in [n]$.
\end{itemize}

The completeness is evident on inspection. Also, for any forgery $(m'
, \sigma')$ there must exist an $i$ such that $m_{i}' \neq m_{i}$
(where $m$ was the query). If $\sigma'$ is verified, then $\sigma_{i}'
\in f^{-1}(Y_{i}^{m_{i}})$.

\end{document}

%%% Local Variables: 
%%% mode: latex
%%% TeX-master: t
%%% End: 
