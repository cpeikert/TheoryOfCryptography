\documentclass[11pt]{article}

\usepackage{fullpage,microtype,pdfsync,hyperref}
\usepackage{newtxtext}
\usepackage[T1]{fontenc}

\usepackage{mathtools,amsfonts,amssymb}
\usepackage[amsmath,amsthm,thmmarks,hyperref]{ntheorem}
\usepackage[capitalize,nameinlink,noabbrev]{cleveref}

\usepackage{algorithm,algorithmic}

\RequirePackage{mathtools}

%%% BLACKBOARD SYMBOLS

% hyperref package defines C and G, undefined to avoid conflicts
\let\C\relax
\let\G\relax
\newcommand{\C}{\ensuremath{\mathbb{C}}}
\newcommand{\D}{\ensuremath{\mathbb{D}}}
\newcommand{\F}{\ensuremath{\mathbb{F}}}
\newcommand{\G}{\ensuremath{\mathbb{G}}}
\newcommand{\J}{\ensuremath{\mathbb{J}}}
\newcommand{\N}{\ensuremath{\mathbb{N}}}
\newcommand{\Q}{\ensuremath{\mathbb{Q}}}
\newcommand{\R}{\ensuremath{\mathbb{R}}}
\newcommand{\T}{\ensuremath{\mathbb{T}}}
\newcommand{\Z}{\ensuremath{\mathbb{Z}}}
\newcommand{\QR}{\ensuremath{\mathbb{QR}}}

\newcommand{\Zt}{\ensuremath{\Z_t}}
\newcommand{\Zp}{\ensuremath{\Z_p}}
\newcommand{\Zq}{\ensuremath{\Z_q}}
\newcommand{\ZN}{\ensuremath{\Z_N}}
\newcommand{\Zps}{\ensuremath{\Z_p^*}}
\newcommand{\ZNs}{\ensuremath{\Z_N^*}}
\newcommand{\JN}{\ensuremath{\J_N}}
\newcommand{\QRN}{\ensuremath{\QR_{N}}}
\newcommand{\QRp}{\ensuremath{\QR_{p}}}

%%% THEOREM COMMANDS

\RequirePackage{amsthm}

\theoremstyle{plain}            % following are "theorem" style
\newtheorem{theorem}{Theorem}[section]
\newtheorem{lemma}[theorem]{Lemma}
\newtheorem{corollary}[theorem]{Corollary}
\newtheorem{proposition}[theorem]{Proposition}
\newtheorem{claim}[theorem]{Claim}
\newtheorem{fact}[theorem]{Fact}

\theoremstyle{definition}       % following are def style
\newtheorem{definition}[theorem]{Definition}
\newtheorem{conjecture}[theorem]{Conjecture}
\newtheorem{example}[theorem]{Example}
\newtheorem{protocol}[theorem]{Protocol}

\theoremstyle{remark}           % following are remark style
\newtheorem{remark}[theorem]{Remark}
\newtheorem{note}[theorem]{Note}

% equation numbering style
\numberwithin{equation}{section}

%%% GENERAL COMPUTING

\newcommand{\bit}{\ensuremath{\set{0,1}}}
\newcommand{\pmone}{\ensuremath{\set{-1,1}}}

% asymptotics
\DeclareMathOperator{\poly}{poly}
\DeclareMathOperator{\polylog}{polylog}
\DeclareMathOperator{\negl}{negl}
\newcommand{\Otil}{\ensuremath{\tilde{O}}}

% probability/distribution stuff
\DeclareMathOperator*{\E}{E}
\DeclareMathOperator*{\Var}{Var}

% sets in calligraphic type
\newcommand{\calD}{\ensuremath{\mathcal{D}}}
\newcommand{\calF}{\ensuremath{\mathcal{F}}}
\newcommand{\calG}{\ensuremath{\mathcal{G}}}
\newcommand{\calH}{\ensuremath{\mathcal{H}}}
\newcommand{\calX}{\ensuremath{\mathcal{X}}}
\newcommand{\calY}{\ensuremath{\mathcal{Y}}}

% types of indistinguishability
\newcommand{\compind}{\ensuremath{\stackrel{c}{\approx}}}
\newcommand{\statind}{\ensuremath{\stackrel{s}{\approx}}}
\newcommand{\perfind}{\ensuremath{\equiv}}

% font for general-purpose algorithms
\newcommand{\algo}[1]{\ensuremath{\mathsf{#1}}}
% font for general-purpose computational problems
\newcommand{\problem}[1]{\ensuremath{\mathsf{#1}}}
% font for complexity classes
\newcommand{\class}[1]{\ensuremath{\mathsf{#1}}}

% complexity classes and languages
\renewcommand{\P}{\class{P}}
\newcommand{\BPP}{\class{BPP}}
\newcommand{\NP}{\class{NP}}
\newcommand{\coNP}{\class{coNP}}
\newcommand{\AM}{\class{AM}}
\newcommand{\coAM}{\class{coAM}}
\newcommand{\IP}{\class{IP}}

%%% "LEFT-RIGHT" PAIRS OF SYMBOLS


% inner product
\DeclarePairedDelimiter\inner{\langle}{\rangle}
% absolute value
\DeclarePairedDelimiter\abs{\lvert}{\rvert}
% a set
\DeclarePairedDelimiter\set{\{}{\}}
% parens
\DeclarePairedDelimiter\parens{(}{)}
% tuple, alias for parens
\DeclarePairedDelimiter\tuple{(}{)}
% square brackets
\DeclarePairedDelimiter\bracks{[}{]}
% rounding off
\DeclarePairedDelimiter\round{\lfloor}{\rceil}
% floor function
\DeclarePairedDelimiter\floor{\lfloor}{\rfloor}
% ceiling function
\DeclarePairedDelimiter\ceil{\lceil}{\rceil}
% length of some vector, element
\DeclarePairedDelimiter\length{\lVert}{\rVert}
% "lifting" of a residue class
\DeclarePairedDelimiter\lift{\llbracket}{\rrbracket}
\DeclarePairedDelimiter\len{\lvert}{\rvert}

%%% CRYPTO-RELATED NOTATION

% KEYS AND RELATED

\newcommand{\key}[1]{\ensuremath{#1}}

\newcommand{\pk}{\key{pk}}
\newcommand{\vk}{\key{vk}}
\newcommand{\sk}{\key{sk}}
\newcommand{\mpk}{\key{mpk}}
\newcommand{\msk}{\key{msk}}
\newcommand{\fk}{\key{fk}}
\newcommand{\id}{id}
\newcommand{\keyspace}{\ensuremath{\mathcal{K}}}
\newcommand{\msgspace}{\ensuremath{\mathcal{M}}}
\newcommand{\ctspace}{\ensuremath{\mathcal{C}}}
\newcommand{\tagspace}{\ensuremath{\mathcal{T}}}
\newcommand{\idspace}{\ensuremath{\mathcal{ID}}}

\newcommand{\concat}{\ensuremath{\|}}

% GAMES

% advantage
\newcommand{\advan}{\ensuremath{\mathbf{Adv}}}

% different attack models
\newcommand{\attack}[1]{\ensuremath{\text{#1}}}

\newcommand{\atk}{\attack{atk}} % dummy attack
\newcommand{\indcpa}{\attack{ind-cpa}}
\newcommand{\indcca}{\attack{ind-cca}}
\newcommand{\anocpa}{\attack{ano-cpa}} % anonymous
\newcommand{\anocca}{\attack{ano-cca}}
\newcommand{\euacma}{\attack{eu-acma}} % forgery: adaptive chosen-message
\newcommand{\euscma}{\attack{eu-scma}} % forgery: static chosen-message
\newcommand{\suacma}{\attack{su-acma}} % strongly unforgeable

% ADVERSARIES
\newcommand{\attacker}[1]{\ensuremath{\mathcal{#1}}}

\newcommand{\Adv}{\attacker{A}}
\newcommand{\AdvA}{\attacker{A}}
\newcommand{\AdvB}{\attacker{B}}
\newcommand{\Dist}{\attacker{D}}
\newcommand{\Sim}{\attacker{S}}
\newcommand{\Ora}{\attacker{O}}
\newcommand{\Inv}{\attacker{I}}
\newcommand{\For}{\attacker{F}}

% CRYPTO SCHEMES

\newcommand{\scheme}[1]{\ensuremath{\text{#1}}}

% pseudorandom stuff
\newcommand{\prg}{\algo{PRG}}
\newcommand{\prf}{\algo{PRF}}
\newcommand{\prp}{\algo{PRP}}

% symmetric-key cryptosystem
\newcommand{\skc}{\scheme{SKC}}
\newcommand{\skcgen}{\algo{Gen}}
\newcommand{\skcenc}{\algo{Enc}}
\newcommand{\skcdec}{\algo{Dec}}

% public-key cryptosystem
\newcommand{\pkc}{\scheme{PKC}}
\newcommand{\pkcgen}{\algo{Gen}}
\newcommand{\pkcenc}{\algo{Enc}} % can also use \kemenc and \kemdec
\newcommand{\pkcdec}{\algo{Dec}}

% digital signatures
\newcommand{\sig}{\scheme{SIG}}
\newcommand{\siggen}{\algo{Gen}}
\newcommand{\sigsign}{\algo{Sign}}
\newcommand{\sigver}{\algo{Ver}}

% message authentication code
\newcommand{\mac}{\scheme{MAC}}
\newcommand{\macgen}{\algo{Gen}}
\newcommand{\mactag}{\algo{Tag}}
\newcommand{\macver}{\algo{Ver}}

% key-encapsulation mechanism
\newcommand{\kem}{\scheme{KEM}}
\newcommand{\kemgen}{\algo{Gen}}
\newcommand{\kemenc}{\algo{Encaps}}
\newcommand{\kemdec}{\algo{Decaps}}

% identity-based encryption
\newcommand{\ibe}{\scheme{IBE}}
\newcommand{\ibesetup}{\algo{Setup}}
\newcommand{\ibeext}{\algo{Ext}}
\newcommand{\ibeenc}{\algo{Enc}}
\newcommand{\ibedec}{\algo{Dec}}

% hierarchical IBE (as key encapsulation)
\newcommand{\hibe}{\scheme{HIBE}}
\newcommand{\hibesetup}{\algo{Setup}}
\newcommand{\hibeext}{\algo{Extract}}
\newcommand{\hibeenc}{\algo{Encaps}}
\newcommand{\hibedec}{\algo{Decaps}}

% binary tree encryption (as key encapsulation)
\newcommand{\bte}{\scheme{BTE}}
\newcommand{\btesetup}{\algo{Setup}}
\newcommand{\bteext}{\algo{Extract}}
\newcommand{\bteenc}{\algo{Encaps}}
\newcommand{\btedec}{\algo{Decaps}}

% trapdoor functions
\newcommand{\tdf}{\scheme{TDF}}
\newcommand{\tdfgen}{\algo{Gen}}
\newcommand{\tdfeval}{\algo{Eval}}
\newcommand{\tdfinv}{\algo{Invert}}
\newcommand{\tdfver}{\algo{Ver}}

%%% PROTOCOLS

\newcommand{\out}{\text{out}}
\newcommand{\view}{\text{view}}

%%% COMMANDS FOR LECTURES/HOMEWORKS

\RequirePackage{fancyhdr}

\newcommand{\lecheader}{%
  \chead{\large \textbf{Lecture \lecturenum\\\lecturetopic}}

  \lhead{\small \textbf{Theory of Cryptography}\\}

  \rhead{\small \textbf{Instructor:
      \href{http://www.eecs.umich.edu/~cpeikert/}{Chris Peikert}\\Scribe:
      \scribename}}

  \setlength{\headheight}{20pt}
  \setlength{\headsep}{16pt}
}


%%% xsim settings

\RequirePackage{xsim}
\RequirePackage{needspace}

\DeclareExerciseEnvironmentTemplate{QRef}
{%
    \par\vspace{\baselineskip}
    \Needspace*{2\baselineskip}
    \noindent
    \hyperref[sol:\ExerciseID]{\textbf{\XSIMmixedcase{\GetExerciseName}~\GetExerciseProperty{counter}.}}
    \label{ques:\ExerciseID}
}{}

\DeclareExerciseEnvironmentTemplate{ARef}{%
        \Needspace*{2\baselineskip}
        \noindent
        \hyperref[ques:\ExerciseID]{\textbf{\XSIMmixedcase{\GetExerciseParameter{exercise-name}}~\GetExerciseProperty{counter}.}}
        \label{sol:\ExerciseID}
        \GetExerciseBody{exercise}
        \newline
        \textbf{\XSIMmixedcase{\GetExerciseName}.}
}{\par\vspace{\baselineskip}}

\DeclareExerciseHeadingTemplate{simple}{%
    \section*{\XSIMmixedcase{\GetExerciseParameter{solution-name}s}}
}

\DeclareExerciseType{question}{%
    exercise-env = question,
    solution-env = answer,
    exercise-name = Question,
    solution-name = Answer,
    exercise-template = QRef,
    solution-template = ARef,
}

\xsimsetup{%
  path = {questions},
  file-extension = {aux},
  print-solutions/headings-template=simple
}

\AtEndDocument{\pagebreak\printsolutions}


% VARIABLES

\newcommand{\lecturenum}{3}
\newcommand{\lecturetopic}{OWFs, Hardness Amplification}
\newcommand{\scribename}{George P.~Burdell}

% END OF VARIABLES

\lecheader

\pagestyle{plain}               % default: no special header

\begin{document}

\thispagestyle{fancy}           % first page should have special header

% LECTURE MATERIAL STARTS HERE

\section{One-Way Functions}
\label{sec:one-way-functions}

Recall our formal definition of a one-way function:
\begin{definition}
  \label{def:owf}
  A function $f : \bit^{*} \to \bit^{*}$ is \emph{one-way} if it
  satisfies the following conditions.
  \begin{itemize}
  \item \emph{Easy to compute.}  There is an efficient algorithm
    computing $f$.  Formally, there exists a (uniform, deterministic)
    poly-time algorithm $\algo{F}$ such that $\algo{F}(x) = f(x)$ for
    all $x \in \bit^{*}$.
  \item \emph{Hard to invert.}  An efficient algorithm inverts $f$ on
    a random input with only negligible probability.  Formally, for
    any non-uniform PPT algorithm $\Inv$, the \emph{advantage} of
    $\Inv$ \[ \advan_{f}(\Inv) := \Pr_{x \gets \bit^{n}} \bracks*{
      \Inv(1^{n}, f(x)) \in f^{-1}(f(x)) } = \negl(n) \] is a
    negligible function (in the security parameter $n$).
  \end{itemize}
\end{definition}
We also allowed a syntactic relaxation where the domain and range (and
the input distribution over the domain) may be arbitrary, as long as
the input distribution can be sampled efficiently (given the security
parameter $n$).

\subsection{Candidates}
\label{sec:candidates}

At the end of last lecture we defined two functions, with the idea of
investigating whether they could be one-way functions:
\begin{enumerate}
\item Subset-sum: define $f_{\text{ss}} : (\Z_{N})^{n} \times \bit^{n}
  \to (\Z_{N})^{n} \times \Z_{N}$, where $N = 2^{n}$, as \[
  f_{\text{ss}}(a_{1}, \ldots, a_{n}, b_{1}, \ldots, b_{n}) = (a_{1},
  \ldots, a_{n}, S = \sum_{i\; :\; b_{i} = 1} a_{i} \bmod N). \]
  Notice that because the $a_{i}$s are part of the output, inverting
  $f$ means finding a subset of $\set{1,\ldots,n}$ (corresponding to
  those $b_{i}$s equaling $1$) that induces the given sum $S$ modulo
  $N$.
\item Multiplication: define $f_{\text{mult}} : \N^{2} \to \N$
  as\footnote{For security parameter $n$, we use the domain $[1,2^{n}]
    \times [1,2^{n}]$.  The cases $x=1$ and $y=1$ are special to rule
    out the trivial preimages $(1,xy)$ and $(xy, 1)$ for the output
    $xy \neq 1$.}
  \[
  f_{\text{mult}}(x,y) =
  \begin{cases}
    1 & \text{if } x=1 \vee y=1 \\
    x \cdot y & \text{otherwise.}
  \end{cases}
  \]
\end{enumerate}

First consider $f_{\text{ss}}$; clearly, it can be computed in
polynomial time. It syntactically matches subset-sum, the famous
$\NP\text{-complete}$ problem, and is conjectured to be one-way.

\begin{question}
  Recall that subset-sum is an \(\NP\text{-complete}\) problem, so,
  assuming \(\P \neq \NP\), it has no polynomial-time algorithm. Does
  this prove that \(f_{ss}\) is a one-way function?
\end{question}

\begin{answer}
  No. The fact that subset-sum is \(\NP\text{-complete}\) means that,
  assuming \(\P \neq \NP\), no polynomial-time algorithm solves
  subset-sum in the \emph{worst case}, i.e., on \emph{any} given
  instance. However, in the one-way function inversion game, the input
  to \(f_{ss}\) is chosen uniformly \emph{at random}, and is not
  necessarily a ``hardest'' instance of subset-sum. In other words, an
  inverter for \(f_{ss}\) only needs to solve subset-sum \emph{on the
    average} and with good enough probability, not necessarily in the
  worst case and with probability one. While we \emph{conjecture} that
  almost all instances of subset-sum are hard, and hence~\(f_{ss}\) is
  one-way, as far as we know it could be the case that a large
  fraction of subset-sum instances are easy to solve, yet a rare few
  are hard. This would allow a polynomial-time inverter to attain
  non-negligible advantage against \(f_{ss}\) without solving the
  \(\NP\text{-complete}\) subset-sum problem in the worst case.
\end{answer}

Now consider $f_{\text{mult}}$, which also can be computed in
polynomial time. Syntactically, inverting $f_{\text{mult}}$ means
factoring~$xy$ into non-trivial integer factors.

\begin{question}
  Factoring integers is conjectured to be a hard problem. Does this
  assumption imply that \(f_{\text{mult}}\) is a one-way function?
\end{question}

\begin{answer}
  No! In fact, \(f_{\text{mult}}\) is certainly \emph{not} a one-way
  function as we have defined it, for the reasons given following the
  text of this question.
\end{answer}

The answer is no: \(f_{\text{mult}}\) is \emph{not} a one-way function
as we have defined it, and this is due to the difference between
factoring in the worst case versus the average case. While \emph{some}
integers seem hard to factor, a large fraction of integers are very
easy to factor. Concretely, consider the following inverter $\Inv(z)$:
if $z$ is even, then output $(2, z/2)$, otherwise fail.  Notice that
$\Inv$ outputs a valid preimage of $z$ exactly when $z = xy$ is even,
which occurs with non-negligible probability $3/4$ when $x, y$ are
chosen independently and uniformly from $[1,2^{n}]$ (because $xy$ is
even exactly when at least one of $x,y$ is even).

There are two common ways to deal with this difference between the
worst case and average case.  The first is to ``tweak'' the
function---actually, its input distribution---to focus on the
instances that we believe to be hard.  For the factoring problem, it
is believed (but not proved!) that the hardest numbers to factor are
those that are the product of exactly two primes of about the same
size.  More specifically, define
\[ \Pi_{n} = \set{ p\; :\; p \in [1,2^{n}] \text{ is prime}} \] to be
the set of ``$n$-bit prime numbers.''  Then we may make the following
plausible conjecture:

\begin{conjecture}[Factoring Assumption]
  \label{conj:fact-primes-hard}
  For every non-uniform PPT algorithm $\Adv$,
  \[ \Pr_{p,q \gets \Pi_{n}} [\Adv(p \cdot q) = (p,q)] = \negl(n). \]
\end{conjecture}

Let's now change the input distribution of $f_{\text{mult}}$ so that
$x,y$ are chosen uniformly and independently from $\Pi_{n}$.  Now, the
one-wayness property for $f_{\text{mult}}$ is syntactically identical
to \cref{conj:fact-primes-hard}, so for that property we
have a ``proof by assumption.''  The only remaining question is the
distribution over inputs: can we efficiently sample a random $n$-bit
prime, i.e., sample uniformly from $\Pi_{n}$?
\cref{alg:genprime} gives a potential algorithm,
$\algo{GenPrime}$, to do so.

\begin{algorithm}
  \caption{Algorithm $\algo{GenPrime}(1^{n})$ for sampling a uniformly
    random $n$-bit prime.}
  \label{alg:genprime}
  \begin{algorithmic}[1]
    \STATE Choose a fresh uniformly random integer $p \in [1, 2^{n}]$.

    \STATE If $p$ is prime, output $p$.  Otherwise start over.
  \end{algorithmic}
\end{algorithm}

Clearly, if $\algo{GenPrime}$ terminates, it outputs an element of
$\Pi_{n}$.  Moreover, it outputs a \emph{uniformly random} element of
$\Pi_{n}$, because every $\bar{p} \in \Pi_{n}$ is equally likely to be
chosen in the first step.  So the only remaining question is the
running time: can we test primality efficiently?  And can we guarantee
that the algorithm terminates quickly, i.e., does not start over too
many times?

The first question has many answers.  The definitive answer was
provided in 2002 by Agrawal, Kayal, and Saxena, who gave a
\emph{deterministic} primality test that runs in time polynomial in
the bit length of the number (i.e., $\poly(n)$ for a number in
$[1,2^{n}]$).  The running time, while polynomial, is quite large, so
the AKS test is not used in practice.  Instead, \emph{randomized}
tests such as Miller-Rabin are used to quickly test primality;
however, such tests have a \emph{very small} probability of returning
the wrong answer (``prime'' when the number is composite, or vice
versa, or both).  This introduces a tiny bias into the output
distribution of $\algo{GenPrime}$, which we can fold into a negligible
error term that we won't need to worry about.\footnote{Another
  alternative is to use \emph{deterministic} primality tests which
  only have proofs of correctness under unproved (but widely believed)
  number-theoretic assumptions.  These are less common in practice
  because they aren't so fast.}  (Later on we will rigorously define
what ``negligible error term'' means.  We may study primality tests in
more detail later.)

The second question, about the running time of $\algo{GenPrime}$, is
also very interesting.  The probability that a particular iteration
terminates is exactly $\abs{\Pi_{n}} / 2^{n}$ (assuming a perfect
primality test).  How does $\abs{\Pi_{n}}$ compare to $2^{n}$, i.e.,
how ``dense'' are the $n$-bit primes?  It is easy to show that there
are an infinite number of primes, but this is not strong enough for
our purposes---primes could be so vanishingly rare that
$\algo{GenPrime}$ effectively runs forever.  Fortunately, it turns out
that the primes are quite dense.  For an integer $N > 1$, let $\pi(N)$
denote the number of primes not exceeding $N$.  We have the following:

\begin{lemma}[Chebyshev]
  \label{lem:chebyshev-primes}
  For every integer $N > 1$, $\pi(N) > \frac{N}{2 \log_{2} N}$.
\end{lemma}
(In fact, this is a weaker form of the ``prime number theorem,'' which
says that $\pi(N)$ asymptotically approaches $\frac{N}{\ln N}$.  We
will not give a proof of \cref{lem:chebyshev-primes} now, but
there is an elementary one in the Pass-shelat notes.)  Using
\cref{lem:chebyshev-primes}, we have
\[ \frac{\abs{\Pi_{n}}}{2^{n}} = \frac{\pi(2^{n})}{2^{n}} >
  \frac{2^{n}}{2^{n} \cdot 2 \log_{2}(2^{n})} = \frac{1}{2n}. \]
Finally, since all the iterations of $\algo{GenPrime}$ are
independent, the probability that it fails to terminate after (say)
$2n^{2}$ iterations is
\[ (1-\tfrac{1}{2n})^{2n \cdot n} \leq (1/2)^{n}, \] which is
exponentially small.\footnote{To be completely formal, in order to say
  that $\algo{GenPrime}$ is polynomial time, we should modify it so
  that if it has not terminated after $2n^{2}$ iterations, it just
  outputs some fixed value (say, $1$).  This introduces only an
  exponentially small bias in its output distribution.}

\subsection{Weak OWFs and Hardness Amplification}
\label{sec:hardn-ampl}

The second way of dealing with the above difficulty of ``somewhere
hard'' functions is more general, but also most costly and technically
more complex.  Suppose we are not confident where the ``hard'' inputs
to a OWF were likely to be, but we do believe that they aren't too
rare.  We would then have a $\delta$-\emph{weak} one-way function,
where the ``hard to invert'' property from \cref{def:owf} is
relaxed to the following:
\begin{itemize}
\item For any non-uniform PPT algorithm $\Inv$, \[ \advan_{f}(\Inv) :=
  \Pr_{x \gets \bit^{n}} \bracks*{\Inv(f(x)) \in f^{-1}(f(x))} \leq
  1-\delta, \] for some $\delta = \delta(n) = 1/\poly(n)$.
\end{itemize}
The only change is that we have replaced the $\negl(n)$ success
probability by $1-\delta$. (To contrast with the ``$\delta$-weak''
nomenclature, we sometimes say that the original condition defines a
\emph{strong} OWF.) This new definition allows for~$f$ to be easily
invertible for a large fraction of inputs (e.g., all the even numbers,
all those divisible by three, perfect squares, and the like), but not
``almost everywhere:'' a $\delta$-fraction of the inputs remain hard.

Suppose that~$f$ is some arbitrary $\delta$-weak OWF.  Can we always
get a (strong) OWF from it?  The answer is yes; this kind of result
lies within the area called \emph{hardness amplification}.  For a
function~$f$, define its $m$-wise \emph{direct product} as
\[ f'(x_{1}, \ldots, x_{m}) = (f(x_{1}), \ldots, f(x_{m})), \] where
we parse the input so that $\len{x_{1}} = \cdots = \len{x_{m}}$.  We
have the following theorem:

\begin{theorem}
  \label{thm:weak-strong-owf}
  If $f$ is a $\delta$-weak one-way function, then $f'$ is a (strong)
  one-way function for any $\poly(n)$-bounded $m \geq 2n/\delta$.
\end{theorem}

\begin{question}
  Does inverting~$f'(x_{1}, \ldots, x_{m})$ require
  inverting~$f(x_{i})$ for all $i=1,\ldots,m$?  Since an nuPPT
  algorithm can have advantage at most $1-\delta$ in inverting a
  single~$f(x_{i})$, does this imply that an nuPPT algorithm can have
  advantage at most
  $(1-\delta)^{m} \leq (1-\delta)^{2n/\delta} \leq (1/e)^{2n} =
  \negl(n)$ in inverting~$f'$?
\end{question}

\begin{answer}
  The answer to the first question is yes: any preimage of
  $f'(x_{1}, \ldots, x_{m}) = (f(x_{1}), \ldots, f(x_{m}))$ is a tuple
  $(x_{1}^{*}, \ldots, x_{m}^{*})$ of equal-length strings~$x_{i}^{*}$
  where $f(x_{i}^{*}) = f(x_{i})$, i.e., each~$x_{i}^{*}$ is a
  preimage of~$f(x_{i})$. (This argument assumes that the output
  tuple of~$f'$ is encoded so that it is uniquely parseable into
  its~$m$ components.)

  The answer to the second question is no: that reasoning is not
  valid, because while the components $y_{i} = f(x_{i})$ are
  independent, an adversary attacking them \emph{need not treat each
    of the inversion subtasks independently}; it might correlate its
  successes and failures. See the text following the original question
  for further elaboration, and the proof of
  \cref{thm:weak-strong-owf} for a rigorous argument.
\end{answer}

\paragraph{Proof idea:} The intuition here is that inverting~$f'$
requires inverting~$f$ on~$m$ independently chosen values.  If each
component is inverted independently with probability at most
$1-\delta$, then the probability of inverting all of them should be at
most
\[ (1-\delta)^{m} \leq (1-\delta)^{2n/\delta} \leq (1/e)^{2n} =
  \negl(n). \] Rigorously proving that this intuition is correct,
however, turns out to be quite subtle.  The reason is that an
adversary $\Inv'$ attacking~$f'$ does not have to treat each of the
inversion subtasks $y_{1} = f(x_{1}), \ldots, y_{m} = f(x_{m})$
independently.  For example, it might look at all of the $y_{i}$ at
once, and either invert \emph{all} of them (with some non-negligible
probability overall), or none of them.  So the individual events (that
$\Inv'$ succeeds in the sub-task of inverting $y_{i} = f(x_{i})$) can
be highly correlated, and we cannot simply say that the overall
success probability is bounded by $(1-\delta)^{m}$.

To give a correct proof of the theorem, we need to perform a
\emph{reduction} from inverting~$f$ with advantage $> 1-\delta$ to
inverting~$f'$ with some non-negligible advantage.  Unpacking this a
bit, we want to prove the contrapositive: suppose there is some nuPPT
adversary~$\Inv'$ that manages to break the (strong) one-wayness
of~$f'$, i.e., it inverts with some non-negligible advantage
$\advan_{f'}(\Inv') = \alpha = \alpha(n)$.  Then we will construct
another nuPPT adversary~$\Inv$ that manages to break the $\delta$-weak
one-wayness of~$f$, i.e., it inverts~$f$ with advantage
$\advan_{f}(\Inv) > 1-\delta$.  In doing so, we will use~$\Inv'$ as a
``black box,'' invoking it on inputs of our choice and using its
outputs in whatever way may be useful.  However, we cannot assume
that~$\Inv'$ ``works'' in any particular way; we can only rely on the
hypothesis that it violates the strong one-wayness of~$f'$.

The basic idea of the reduction is this: $\Inv$ is given $y = f(x)$
and wants to use $\Inv'$ to invert~$y$.  An obvious first attempt is
to ``plug in'' the value~$y$ as part of the output of~$f'$, i.e., to
run $\Inv'$ on
\[ y' = (y_{1} = y = f(x), y_{2} = f(x_{2}), \ldots, y_{m} = f(x_{m}))
  = f'(x_{1} = x, x_{2}, \ldots, x_{m}) \] for some random
$x_{2}, \ldots, x_{m}$ chosen by $\Inv$.  When $\Inv'$ produces a
potential preimage $x'=(x'_{1},\ldots,x'_{m})$, then $\Inv$ just
outputs $x'_{1}$ as its potential preimage of $y=f(x)$.  Notice that
if~$\Inv'$ happens to invert~$y'$ successfully, then in particular
$x_{1} \in f^{-1}(y)$, as required.

Unfortunately, this simple idea does not quite work.  The problem is
that~$\Inv'$ inverts only with non-negligible probability $\alpha$,
whereas we need $\Inv$ to invert with (typically larger) probability
$> 1-\delta$, to violate the weak one-wayness of~$f$.  We might hope
to fix this problem by running~$\Inv'$ a large number of times, using
the same $y_{1} = f(x)$ but \emph{fresh} random $x_{2}, \ldots, x_{m}$
each time, expecting $\Inv'$ to successfully invert at least once.
(Note that we can check whether~$\Inv'$ succeeded in inverting~$f'$ on
a particular tuple, because~$f'$ is efficiently computable.)  This
strategy is better, but still flawed: the problem is that the runs are
\emph{not independent}, since we use the same $y_{1} = f(x)$ every
time.  Indeed, $\Inv'$ might behave in a ``sneaky'' way to make this
very strategy fail.  For example, $\Inv'$ might fix a certain
$\alpha$-fraction of ``good'' values of $x_{1}$ for which it always
inverts (ignoring $x_{2}, \ldots, x_{m}$), and fail for \emph{all}
other values of $x_{1}$.  Notice that such $\Inv'$ has overall
advantage $\alpha$, as required.  But no matter how many time we
repeatedly run $\Inv'$ with $y_{1}=y=f(x)$, we will eventually succeed
only if the original $x$ is ``good,'' which happens with probability
only $\alpha$.

A better reduction (which will turn out to work) just applies the
above ``plug-and-repeat'' strategy for \emph{every} position
$i = 1, \ldots, m$.  As it turns out, we can show that \emph{some}
position must have a very large ($> 1-\delta/2$) fraction of ``good''
$x_{i}$'s, for a suitable definition of good.  Conditioned on the~$x$
from our given instance $y = f(x)$ being good, by repeatedly
calling~$\Inv'$ a polynomial number of times (using fresh
other~$x_{i}$ values each time), the probability that~$\Inv'$ inverts
successfully in at least one call will be extremely close to 1, so our
overall advantage in inverting~$f$ will be
$\approx 1-\delta/2 > 1-\delta$, as desired.

We now give the formal proof.

\begin{proof}[Proof of \cref{thm:weak-strong-owf}]
  Because $m = \poly(n)$ by assumption (and because $2n/\delta = 2n
  \cdot \poly(n) = \poly(n)$), if $f$ is efficiently computable, then
  so is $f'$.  Now suppose that some efficient adversary $\Inv'$
  violates the one-wayness of $f'$, with non-negligible advantage
  $\alpha = \alpha(n) = \advan_{\Inv'}(f')$.  We wish to construct an
  efficient adversary $\Inv$ that uses $\Inv'$ (efficiently) to
  violate the $\delta$-weak one-wayness of $f$.

  We first establish a crucial property of $\Inv'$ that will help us
  obtain the desired reduction.  Define the ``good'' set for position
  $i$ to be the set of all $x_{i}$ such that $\Inv'$ successfully
  inverts with ``good enough'' probability, over the random choice of
  all the remaining $x_{j}$.  Formally, \[ G_{i} = \set*{ x_{i}\;
    :\; \Pr_{x_{j} \text{ for } j \neq i} \bracks*{ \Inv'(f'(x'))
      \text{ succeeds} } \geq \frac{\alpha}{2m}}. \] Note that because
  $\Inv'$ succeeds with at least $\alpha$ probability overall, it is
  easy to see that $G_{i}$ is at least $(\alpha/2)$-dense for every
  $i$.  However, we will show a very different fact: that $G_{i}$ is
  at least $(1-\delta/2)$-dense for \emph{some} $i$.  (Note that
  typically, $(1-\delta/2)$ is much larger than $\alpha$.)

  \begin{claim}
    \label{claim:good-dense}
    For some $i$, we have $\frac{\abs{G_{i}}}{2^{n}} \geq
    (1-\delta/2)$.
  \end{claim}

  \begin{proof}
    Suppose not.  Then we have
    \begin{align*}
      \lefteqn{\Pr_{x'}[\Inv'(f'(x')) \text{ succeeds]}} \\
      &\leq \Pr[\Inv' \text{ succeeds} \wedge \text{every } x_{i} \in
      G_{i}] + \sum_{i=1}^{m} \Pr[\Inv' \text{ succeeds} \wedge
      x_{i} \not\in G_{i}] & \text{(union bound)} \\
      &\leq \Pr[\text{every } x_{i} \in G_{i}] + \sum_{i=1}^{m}
      \Pr[\Inv' \text{ succeeds}\; |\; x_{i} \not\in
      G_{i}] & \text{(probability)} \\
      &< (1-\tfrac{\delta}{2})^{2n/\delta} + m \cdot \tfrac{\alpha}{2m}
      & \text{(def \& size of $G_{i}$)} \\
      &\leq 2^{-n} + \alpha/2 < \alpha,
    \end{align*}
    contradicting our assumption on $\Inv'$.
  \end{proof}

  The remainder of the proof is a simple exercise, following the
  ``plug-and-repeat'' strategy described above, and using
  \cref{claim:good-dense}.
\end{proof}

\end{document}

%%% Local Variables: 
%%% mode: latex
%%% TeX-master: t
%%% End: 
