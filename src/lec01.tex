\documentclass[11pt]{article}

\usepackage{fullpage}
\usepackage{newtxtext}
\usepackage[T1]{fontenc}
\usepackage{hyperref,microtype,pdfsync}
\usepackage{amsmath,amsfonts,amssymb,amsthm}
\usepackage{mathtools}
\usepackage{fancyhdr}
\usepackage{xsim}
\usepackage{needspace}

\RequirePackage{mathtools}

%%% BLACKBOARD SYMBOLS

% hyperref package defines C and G, undefined to avoid conflicts
\let\C\relax
\let\G\relax
\newcommand{\C}{\ensuremath{\mathbb{C}}}
\newcommand{\D}{\ensuremath{\mathbb{D}}}
\newcommand{\F}{\ensuremath{\mathbb{F}}}
\newcommand{\G}{\ensuremath{\mathbb{G}}}
\newcommand{\J}{\ensuremath{\mathbb{J}}}
\newcommand{\N}{\ensuremath{\mathbb{N}}}
\newcommand{\Q}{\ensuremath{\mathbb{Q}}}
\newcommand{\R}{\ensuremath{\mathbb{R}}}
\newcommand{\T}{\ensuremath{\mathbb{T}}}
\newcommand{\Z}{\ensuremath{\mathbb{Z}}}
\newcommand{\QR}{\ensuremath{\mathbb{QR}}}

\newcommand{\Zt}{\ensuremath{\Z_t}}
\newcommand{\Zp}{\ensuremath{\Z_p}}
\newcommand{\Zq}{\ensuremath{\Z_q}}
\newcommand{\ZN}{\ensuremath{\Z_N}}
\newcommand{\Zps}{\ensuremath{\Z_p^*}}
\newcommand{\ZNs}{\ensuremath{\Z_N^*}}
\newcommand{\JN}{\ensuremath{\J_N}}
\newcommand{\QRN}{\ensuremath{\QR_{N}}}
\newcommand{\QRp}{\ensuremath{\QR_{p}}}

%%% THEOREM COMMANDS

\RequirePackage{amsthm}

\theoremstyle{plain}            % following are "theorem" style
\newtheorem{theorem}{Theorem}[section]
\newtheorem{lemma}[theorem]{Lemma}
\newtheorem{corollary}[theorem]{Corollary}
\newtheorem{proposition}[theorem]{Proposition}
\newtheorem{claim}[theorem]{Claim}
\newtheorem{fact}[theorem]{Fact}

\theoremstyle{definition}       % following are def style
\newtheorem{definition}[theorem]{Definition}
\newtheorem{conjecture}[theorem]{Conjecture}
\newtheorem{example}[theorem]{Example}
\newtheorem{protocol}[theorem]{Protocol}

\theoremstyle{remark}           % following are remark style
\newtheorem{remark}[theorem]{Remark}
\newtheorem{note}[theorem]{Note}

% equation numbering style
\numberwithin{equation}{section}

%%% GENERAL COMPUTING

\newcommand{\bit}{\ensuremath{\set{0,1}}}
\newcommand{\pmone}{\ensuremath{\set{-1,1}}}

% asymptotics
\DeclareMathOperator{\poly}{poly}
\DeclareMathOperator{\polylog}{polylog}
\DeclareMathOperator{\negl}{negl}
\newcommand{\Otil}{\ensuremath{\tilde{O}}}

% probability/distribution stuff
\DeclareMathOperator*{\E}{E}
\DeclareMathOperator*{\Var}{Var}

% sets in calligraphic type
\newcommand{\calD}{\ensuremath{\mathcal{D}}}
\newcommand{\calF}{\ensuremath{\mathcal{F}}}
\newcommand{\calG}{\ensuremath{\mathcal{G}}}
\newcommand{\calH}{\ensuremath{\mathcal{H}}}
\newcommand{\calX}{\ensuremath{\mathcal{X}}}
\newcommand{\calY}{\ensuremath{\mathcal{Y}}}

% types of indistinguishability
\newcommand{\compind}{\ensuremath{\stackrel{c}{\approx}}}
\newcommand{\statind}{\ensuremath{\stackrel{s}{\approx}}}
\newcommand{\perfind}{\ensuremath{\equiv}}

% font for general-purpose algorithms
\newcommand{\algo}[1]{\ensuremath{\mathsf{#1}}}
% font for general-purpose computational problems
\newcommand{\problem}[1]{\ensuremath{\mathsf{#1}}}
% font for complexity classes
\newcommand{\class}[1]{\ensuremath{\mathsf{#1}}}

% complexity classes and languages
\renewcommand{\P}{\class{P}}
\newcommand{\BPP}{\class{BPP}}
\newcommand{\NP}{\class{NP}}
\newcommand{\coNP}{\class{coNP}}
\newcommand{\AM}{\class{AM}}
\newcommand{\coAM}{\class{coAM}}
\newcommand{\IP}{\class{IP}}

%%% "LEFT-RIGHT" PAIRS OF SYMBOLS


% inner product
\DeclarePairedDelimiter\inner{\langle}{\rangle}
% absolute value
\DeclarePairedDelimiter\abs{\lvert}{\rvert}
% a set
\DeclarePairedDelimiter\set{\{}{\}}
% parens
\DeclarePairedDelimiter\parens{(}{)}
% tuple, alias for parens
\DeclarePairedDelimiter\tuple{(}{)}
% square brackets
\DeclarePairedDelimiter\bracks{[}{]}
% rounding off
\DeclarePairedDelimiter\round{\lfloor}{\rceil}
% floor function
\DeclarePairedDelimiter\floor{\lfloor}{\rfloor}
% ceiling function
\DeclarePairedDelimiter\ceil{\lceil}{\rceil}
% length of some vector, element
\DeclarePairedDelimiter\length{\lVert}{\rVert}
% "lifting" of a residue class
\DeclarePairedDelimiter\lift{\llbracket}{\rrbracket}
\DeclarePairedDelimiter\len{\lvert}{\rvert}

%%% CRYPTO-RELATED NOTATION

% KEYS AND RELATED

\newcommand{\key}[1]{\ensuremath{#1}}

\newcommand{\pk}{\key{pk}}
\newcommand{\vk}{\key{vk}}
\newcommand{\sk}{\key{sk}}
\newcommand{\mpk}{\key{mpk}}
\newcommand{\msk}{\key{msk}}
\newcommand{\fk}{\key{fk}}
\newcommand{\id}{id}
\newcommand{\keyspace}{\ensuremath{\mathcal{K}}}
\newcommand{\msgspace}{\ensuremath{\mathcal{M}}}
\newcommand{\ctspace}{\ensuremath{\mathcal{C}}}
\newcommand{\tagspace}{\ensuremath{\mathcal{T}}}
\newcommand{\idspace}{\ensuremath{\mathcal{ID}}}

\newcommand{\concat}{\ensuremath{\|}}

% GAMES

% advantage
\newcommand{\advan}{\ensuremath{\mathbf{Adv}}}

% different attack models
\newcommand{\attack}[1]{\ensuremath{\text{#1}}}

\newcommand{\atk}{\attack{atk}} % dummy attack
\newcommand{\indcpa}{\attack{ind-cpa}}
\newcommand{\indcca}{\attack{ind-cca}}
\newcommand{\anocpa}{\attack{ano-cpa}} % anonymous
\newcommand{\anocca}{\attack{ano-cca}}
\newcommand{\euacma}{\attack{eu-acma}} % forgery: adaptive chosen-message
\newcommand{\euscma}{\attack{eu-scma}} % forgery: static chosen-message
\newcommand{\suacma}{\attack{su-acma}} % strongly unforgeable

% ADVERSARIES
\newcommand{\attacker}[1]{\ensuremath{\mathcal{#1}}}

\newcommand{\Adv}{\attacker{A}}
\newcommand{\AdvA}{\attacker{A}}
\newcommand{\AdvB}{\attacker{B}}
\newcommand{\Dist}{\attacker{D}}
\newcommand{\Sim}{\attacker{S}}
\newcommand{\Ora}{\attacker{O}}
\newcommand{\Inv}{\attacker{I}}
\newcommand{\For}{\attacker{F}}

% CRYPTO SCHEMES

\newcommand{\scheme}[1]{\ensuremath{\text{#1}}}

% pseudorandom stuff
\newcommand{\prg}{\algo{PRG}}
\newcommand{\prf}{\algo{PRF}}
\newcommand{\prp}{\algo{PRP}}

% symmetric-key cryptosystem
\newcommand{\skc}{\scheme{SKC}}
\newcommand{\skcgen}{\algo{Gen}}
\newcommand{\skcenc}{\algo{Enc}}
\newcommand{\skcdec}{\algo{Dec}}

% public-key cryptosystem
\newcommand{\pkc}{\scheme{PKC}}
\newcommand{\pkcgen}{\algo{Gen}}
\newcommand{\pkcenc}{\algo{Enc}} % can also use \kemenc and \kemdec
\newcommand{\pkcdec}{\algo{Dec}}

% digital signatures
\newcommand{\sig}{\scheme{SIG}}
\newcommand{\siggen}{\algo{Gen}}
\newcommand{\sigsign}{\algo{Sign}}
\newcommand{\sigver}{\algo{Ver}}

% message authentication code
\newcommand{\mac}{\scheme{MAC}}
\newcommand{\macgen}{\algo{Gen}}
\newcommand{\mactag}{\algo{Tag}}
\newcommand{\macver}{\algo{Ver}}

% key-encapsulation mechanism
\newcommand{\kem}{\scheme{KEM}}
\newcommand{\kemgen}{\algo{Gen}}
\newcommand{\kemenc}{\algo{Encaps}}
\newcommand{\kemdec}{\algo{Decaps}}

% identity-based encryption
\newcommand{\ibe}{\scheme{IBE}}
\newcommand{\ibesetup}{\algo{Setup}}
\newcommand{\ibeext}{\algo{Ext}}
\newcommand{\ibeenc}{\algo{Enc}}
\newcommand{\ibedec}{\algo{Dec}}

% hierarchical IBE (as key encapsulation)
\newcommand{\hibe}{\scheme{HIBE}}
\newcommand{\hibesetup}{\algo{Setup}}
\newcommand{\hibeext}{\algo{Extract}}
\newcommand{\hibeenc}{\algo{Encaps}}
\newcommand{\hibedec}{\algo{Decaps}}

% binary tree encryption (as key encapsulation)
\newcommand{\bte}{\scheme{BTE}}
\newcommand{\btesetup}{\algo{Setup}}
\newcommand{\bteext}{\algo{Extract}}
\newcommand{\bteenc}{\algo{Encaps}}
\newcommand{\btedec}{\algo{Decaps}}

% trapdoor functions
\newcommand{\tdf}{\scheme{TDF}}
\newcommand{\tdfgen}{\algo{Gen}}
\newcommand{\tdfeval}{\algo{Eval}}
\newcommand{\tdfinv}{\algo{Invert}}
\newcommand{\tdfver}{\algo{Ver}}

%%% PROTOCOLS

\newcommand{\out}{\text{out}}
\newcommand{\view}{\text{view}}

%%% COMMANDS FOR LECTURES/HOMEWORKS

\RequirePackage{fancyhdr}

\newcommand{\lecheader}{%
  \chead{\large \textbf{Lecture \lecturenum\\\lecturetopic}}

  \lhead{\small \textbf{Theory of Cryptography}\\}

  \rhead{\small \textbf{Instructor:
      \href{http://www.eecs.umich.edu/~cpeikert/}{Chris Peikert}\\Scribe:
      \scribename}}

  \setlength{\headheight}{20pt}
  \setlength{\headsep}{16pt}
}


%%% xsim settings

\RequirePackage{xsim}
\RequirePackage{needspace}

\DeclareExerciseEnvironmentTemplate{QRef}
{%
    \par\vspace{\baselineskip}
    \Needspace*{2\baselineskip}
    \noindent
    \hyperref[sol:\ExerciseID]{\textbf{\XSIMmixedcase{\GetExerciseName}~\GetExerciseProperty{counter}.}}
    \label{ques:\ExerciseID}
}{}

\DeclareExerciseEnvironmentTemplate{ARef}{%
        \Needspace*{2\baselineskip}
        \noindent
        \hyperref[ques:\ExerciseID]{\textbf{\XSIMmixedcase{\GetExerciseParameter{exercise-name}}~\GetExerciseProperty{counter}.}}
        \label{sol:\ExerciseID}
        \GetExerciseBody{exercise}
        \newline
        \textbf{\XSIMmixedcase{\GetExerciseName}.}
}{\par\vspace{\baselineskip}}

\DeclareExerciseHeadingTemplate{simple}{%
    \section*{\XSIMmixedcase{\GetExerciseParameter{solution-name}s}}
}

\DeclareExerciseType{question}{%
    exercise-env = question,
    solution-env = answer,
    exercise-name = Question,
    solution-name = Answer,
    exercise-template = QRef,
    solution-template = ARef,
}

\xsimsetup{%
  path = {questions},
  file-extension = {aux},
  print-solutions/headings-template=simple
}

\AtEndDocument{\pagebreak\printsolutions}


% VARIABLES

\newcommand{\lecturenum}{1}
\newcommand{\lecturetopic}{Overview, Perfect Secrecy}
\newcommand{\scribename}{George P.~Burdell}

% END OF VARIABLES

\lecheader                      % execute lecture commands

\pagestyle{plain}               % default: no special header

\begin{document}

\thispagestyle{fancy}           % first page should have special header

% LECTURE MATERIAL STARTS HERE

\section{Overview of Course}
\label{sec:overview-course}

Cryptography is about \emph{communicating} and \emph{computing}
securely in the presence of \emph{malicious} behavior.  Its use goes
back to antiquity, but only in the past few decades has it become a
rigorous discipline.  Central to the modern approach are precise
mathematical \emph{models} and strong \emph{definitions} of security,
as well as rigorous \emph{proofs} based on well-formed (and often
mild) assumptions.

\medskip \noindent
Our inquiry will be centered around several core themes: 
\begin{itemize}
\item \textbf{``Perfect'' security.}  The ideal and its limitations.

\item \textbf{Computational hardness.}  What it means for a problem to
  be ``hard.''  The kinds of hardness needed for cryptographic
  applications.  Where we might hope to find such hardness.

\item \textbf{Indistinguishability and pseudorandomness.}  How very
  different objects can appear ``essentially the same,'' and how
  hardness can be used to achieve this.  Applying these concepts to
  secret communication (encryption).

\item \textbf{Authentication.}  How to ensure that a message came from
  an expected source, not an impersonator.  How to identify yourself
  remotely.

\item \textbf{Interaction and knowledge.}  Proving that a statement is
  true, without giving any reason why.  Keeping a secret even if
  pieces of it are revealed.  Running a program in public without
  revealing its inputs.

\item \textbf{Advanced encryption.}  Secrecy even given a decryption
  oracle.  Your name as your public key.

\item \textbf{Special topics.}  Querying a database privately.
  Implications of quantum computers.  Lattice-based cryptography.  The
  frontier.
\end{itemize}

See the Course Information handout for policies and procedures.

\section{Hidden Writing and Perfect Secrecy}
\label{sec:hidd-writ-perf}

Suppose a sender (call her Alice) wants to send a secret message to a
receiver (Bob), but she doesn't want a potential eavesdropper (Eve) to
be privy to its contents.  Can this be achieved?

In addressing this question (and almost all others in this course!),
we will apply the \emph{cryptographic methodology}:
\begin{enumerate}
\item Form a realistic \emph{model} of the problem (adjusting
  as necessary to allow for the possibility of a solution).
\item Next, \emph{precisely define} the desired functionality and
  security properties of a potential solution.
\item Finally, \emph{construct} and \emph{analyze} a solution,
  (ideally) proving that it satisfies all the desired properties.
\end{enumerate}

\subsection{A First Attempt at a Model}
\label{sec:first-attempt}

In any scientific discipline, one of the earliest tasks in addressing
a question is to form a precise \emph{model} of the problem.  This is
one of the most important steps of the process, because all else
depends on it: we need the model both to admit a useful solution to
our problem, and to be as close to ``reality'' as possible (even
though it will necessarily only be an approximation).

Let's try to define a model for the above ``hidden writing'' problem.
In cryptography, we generally model all the potential actors in the
system by \emph{algorithms}.  These algorithms have precisely defined
interfaces, i.e., what inputs they take and what outputs they produce.
In this case, we might choose the following model:
\begin{itemize}
\item The sender Alice is represented by an algorithm $A(\cdot)$ that
  takes as input a ``plaintext'' $m$ from some (finite) set of
  possible messages $\msgspace$, and outputs some ``ciphertext'' $c$
  from some (finite) set $\ctspace$.
\item The receiver Bob is an algorithm $B(\cdot)$ that takes as input
  a ciphertext $c \in \ctspace$, and outputs a message $m \in
  \msgspace$.
\item The eavesdropper Eve is some algorithm $E(\cdot)$ that takes as
  input (i.e., is allowed to see) a ciphertext $c$, and outputs\ldots
  what, exactly?  Because $E$ is an adversary, we have little control
  over what it does, so let's not impose any constraint on the form of
  its output.  

  Note, however, that we have specified (though somewhat imprecisely
  at this point) Eve's privileges in attacking the system: she gets to
  see a ciphertext $c$, and nothing else.
\end{itemize}
So far, so good.  Notice that we have so far defined only the
\emph{interfaces} of the algorithms, but not any of the properties we
would like them to have.  One obvious property we'd want is that Bob
should correctly recover (i.e., output) the message that Alice
intended.  More precisely: for every $m \in \msgspace$, we should have
$B(A(m)) = m$.  (For jargon lovers, this is often called the
``completeness'' property.)

Our next task is to try to precisely define the desired
\emph{security} property we want our scheme to have.  This is usually
one of the most difficult and subtle tasks to get right, perhaps
because it is a ``negative goal:'' we want to say that Eve
\emph{cannot} do something --- but what, exactly?  For the moment, we
can certainly agree that even a minimally secure scheme should
\emph{at least} prevent Eve from always discovering Alice's message
$m$.  But a moment's thought reveals a problem: if the eavesdropper
simply runs Bob's algorithm $B$ (i.e., if $E = B$), then by the
completeness property, $E$ will \emph{always} output the correct $m$.
The problem is with our model --- it is too strong to allow for a
meaningful solution to the problem.  We need to change it.

\subsection{Fixing the Model}
\label{sec:fixing-model}

To avoid the problem described above, we need to introduce something
that distinguishes Bob (and perhaps Alice as well) from Eve.  One
immediate idea is to make Bob's algorithm $B$ \emph{secret}, so that
Eve cannot run it (often called ``security by obscurity'').  This
turns out to be a \emph{terrible idea}: the history of cryptography
(and security in general) is littered with the discarded remains of
``secret'' algorithms/mechanisms that were anything but, or were
broken even without discovering the mechanism at all.  Furthermore, it
is impossible to evaluate the security or effectiveness of an
algorithm without knowing what it is!  Therefore, a central tenet of
modern cryptography is that
\begin{center}
  \emph{the system should be secure even if all its algorithms are
    public}.
\end{center}
(This maxim is often called Kerckhoff's Law.)  Of course, we need not
go out of our way to disclose our algorithms to our enemies, but we
should play it safe and assume that they will somehow learn what they
are.  And in practice, designing security mechanisms to be public
often ``keeps us honest'' and ends up leading to better solutions.

OK, back to the problem at hand.  Instead of using a secret algorithm,
to distinguish Bob from Eve we use a secret \emph{input}, typically
called a ``key'' (by analogy to a lock mechanism that only the key can
open).  We augment our model above with an additional algorithm
$\skcgen$ that creates a key, which then becomes an input to Alice and
Bob, but not to Eve.\footnote{This is called the ``shared-key'' or
  ``symmetric-key'' model, for obvious reasons.  Later in the course
  we will see other models that can be both more flexible to use, and
  allow for amazing functionality and security properties.} For more
evocative notation, we also represent Alice by $\skcenc$ (for
``encrypt'') and Bob by $\skcdec$ (for ``decrypt'').  Our new model is
as follows:
\begin{itemize}
\item $\skcgen$ is a \emph{randomized} algorithm that takes no input,
  and outputs a key $k$ in some (finite) set $\keyspace$.
\item $\skcenc_{k}(m) = \skcenc(k,m)$ takes a key $k \in \keyspace$
  and message $m \in \msgspace$, and outputs a ciphertext $c \in
  \ctspace$.
\item $\skcdec_{k}(c) = \skcdec(k,c)$ takes a key $k \in \keyspace$
  and ciphertext $c \in \ctspace$, and outputs a message $m \in
  \msgspace$.
\end{itemize}
Notice that $\skcgen$ \emph{cannot} be deterministic (i.e., it must be
able to ``flip coins''), or else we would have gained nothing: Eve
could just run $\skcgen$ herself and learn the key.  Also note that
our model is still realistic, though somewhat less usable than before:
Alice and Bob both need to get ahold of $\skcgen$'s output, so they
may need to meet in advance or have some trusted communication path to
obtain the key without Eve intercepting it.

For completeness (pun not intended), let's update our correctness
condition: for every $k \in \keyspace$ and $m \in \msgspace$, we
should have $\skcdec_{k}(\skcenc_{k}(m)) = m$.

\subsection{Shannon / Perfect Secrecy}
\label{sec:shannon-perfect-secrecy}

Now we have a model that (hopefully) allows for a secure solution.
But what does ``secure'' \emph{mean}?  We can imagine many desirable
properties:
\begin{itemize}
\item Eve should not learn the key.
\item Eve should not be able to output the message, given the
  ciphertext.
\item The ciphertext should look like ``random gibberish'' without the
  key.
\item \ldots
\end{itemize}
These all seem nice enough, but where does the list end?  And how do
we define them in a precise, mathematical way?

Let's go back and consider what we really want our scheme to
accomplish: \emph{it should conceal the message}.  (The key and the
ciphertext are only means to this end, so our security notion should
not be ``about'' them.)  Ideally, we would like the ciphertext to
convey \emph{no (new) information the message} to the adversary, i.e.,
\begin{center}
  seeing the ciphertext should be no better than \emph{seeing nothing
    at all}.
\end{center}
This principle is a crucial insight that will appear again and again
(in various guises) throughout our study of cryptography.

In his seminal 1949 work on information theory and cryptography,
Claude Shannon precisely expressed the above principle in the language
of probability theory, giving the following definition.

\begin{definition}[Shannon secrecy]
  \label{def:shannon-secrecy}
  A shared-key encryption scheme $(\skcgen,\skcenc,\skcdec)$ with
  message space $\msgspace$ and ciphertext space $\ctspace$ is
  \emph{Shannon secret with respect to a probability distribution $D$}
  over $\msgspace$ if for all $\bar{m} \in \msgspace$ and all $\bar{c}
  \in \ctspace$,
  \begin{equation}
    \label{eq:shannon}
    \Pr_{m \gets D,\; k \gets \skcgen}[m = \bar{m}\; |\;
    \skcenc_{k}(m) = \bar{c} ] = \Pr_{m \gets D}[ m = \bar{m} ].
  \end{equation}
  The scheme is \emph{Shannon secret} if it is Shannon secret with
  respect to every distribution $D$ over $\msgspace$.
\end{definition}

Let's consider this definition.  First, the distribution $D$
represents how Alice chooses her message, and can be arbitrary --- but
we (conservatively) imagine that the distribution itself is publicly
known.  The right-hand side of Equation~\eqref{eq:shannon} is simply
the \textit{a priori} probability that Alice chooses the message
$\bar{m}$; this represents what Eve already knows about the message
\emph{without seeing the ciphertext}.  The left-hand side is the
\textit{a posteriori} probability that Alice chose the message
$\bar{m}$, \emph{conditioned} on the fact that the ciphertext (which
Eve gets to see) was $\bar{c}$.  The definition says that the two
probabilities are exactly the same, or in other words, that no matter
the value of the ciphertext, it reveals nothing new about the
underlying message that was encrypted.

Let's now rewrite the Shannon secrecy condition in another way, to
eliminate the (somewhat cumbersome) conditional probability.  Using
the definition of conditional probability, then substituting $m$ for
$\bar{m}$ in $\skcenc_{k}(m)$ (because $m=\bar{m}$ is the other event
in the conjunction), and then using the independence of $m$ and $k$,
we can expand the left side of Equation~\eqref{eq:shannon} as follows:
\[ \Pr_{m,k}[m=\bar{m}\; |\; \skcenc_{k}(m)=\bar{c}] =
\frac{\Pr_{m,k}[m = \bar{m} \wedge
  \skcenc_{k}(\bar{m})=\bar{c}]}{\Pr_{m,k}[\skcenc_{k}(m)=\bar{c}]} =
\frac{\Pr_{m}[m=\bar{m}] \cdot
  \Pr_{k}[\skcenc_{k}(\bar{m})=\bar{c}]}{\Pr_{m,k}[\skcenc_{k}(m)=\bar{c}]}. \]
Shannon secrecy says that the above expression must equal
$\Pr[m=\bar{m}]$, which is one of the terms in the numerator.  As long
as this probability is positive (i.e., $\bar{m}$ is in the support of
$D$), we can cancel it from both sides.  (Note that if $\bar{m}$ is
not in the support of $D$, then Equation~\eqref{eq:shannon} always
holds, regardless of how $\skcenc$ works!)  We therefore get an
equivalent form of Shannon secrecy, which is: for every distribution
$D$ over $\msgspace$, every $\bar{m}$ in the support of $D$, and every
$\bar{c} \in \ctspace$, \[ \Pr_{k \gets
  \skcgen}[\skcenc_{k}(\bar{m})=\bar{c}] = \Pr_{m \gets D,k \gets
  \skcgen}[\skcenc_{k}(m)=\bar{c}]. \] In words, every
\emph{particular} message $\bar{m}$ is equally likely to encrypt to
$\bar{c}$ as a \emph{random} message $m$ (chosen from~$D$) is.

While we have simplified the Shannon secrecy condition, it can still
be somewhat complicated to analyze, because of the arbitrary message
distribution $D$.  Here we give a simpler definition which eliminates
this distribution.  It says that no matter what message is encrypted,
the ciphertext's distribution (solely over the random choices of
$\skcgen$ and $\skcenc$) is exactly the same.

\begin{definition}[Perfect secrecy]
  \label{def:perfect-secrecy}
  A shared-key encryption scheme $(\skcgen,\skcenc,\skcdec)$ with
  message space $\msgspace$ and ciphertext space $\ctspace$ is
  \emph{perfectly secret} if for all $m_{0}, m_{1} \in \msgspace$ and
  all $\bar{c} \in \ctspace$,
  \begin{equation}
    \label{eq:perfect}
    \Pr_{k \gets \skcgen}[\skcenc_{k}(m_{0}) = \bar{c}] = \Pr_{k \gets
      \skcgen}[\skcenc_{k}(m_{1}) = \bar{c}].
  \end{equation}
\end{definition}

Using elementary probability facts, it is straightforward to prove
that perfect secrecy is \emph{equivalent} to Shannon secrecy (in
either of the forms above).  That is, a scheme is Shannon secret if
and only if it is perfectly secret, so we can use whichever definition
we prefer.  (As an exercise, give the proof.)  The fact that such
syntactically different definitions have exactly the same meaning also
gives us further confidence that our definition is meaningful and
robust.  Later on we will see other examples where seemingly very
different security definitions end up being equivalent.

\subsection{A Perfectly Secret Scheme: The One-Time Pad}
\label{sec:one-time-pad}

Now that we have a model and a (pair of) good definition(s), it's time
to move to the third step of our methodology: constructing and
analyzing a scheme.

Interestingly, the encryption scheme we use predates Shannon's
definition by 30 years!  (Nowadays it is less common for a scheme to
end up being proved secure according to a definition that was
formulated later on, but those were simpler times.)  Today the scheme
is is called the \emph{one-time pad}, or sometimes the \emph{Vernam
  cipher} after its inventor.  The intuition is that the key is used
to completely ``randomize'' the message, but in a reversible way
(using the key).

\begin{definition}[One-Time Pad]
  Let $n \geq 1$ be an integer.  The key, message, and ciphertext
  spaces are each the set of $n$-bit strings: $\keyspace = \msgspace =
  \ctspace = \bit^{n}$.  The scheme is defined as follows:
  \begin{itemize}
  \item $\skcgen$ outputs a uniformly random $k \gets \bit^{n}$.
  \item $\skcenc_{k}(m)$ outputs $c = m \oplus k \in \bit^{n}$, where
    $\oplus$ denotes the bitwise exclusive-or.
  \item $\skcdec_{k}(c)$ outputs $m = c \oplus k \in \bit^{n}$.
  \end{itemize}
\end{definition}

\begin{theorem}
  \label{thm:otp-perfect}
  The one-time pad is a perfectly secret shared-key encryption scheme.
\end{theorem}

\begin{proof}
  First, the one-time pad is well-defined as a shared-key encryption
  scheme: we have defined the sets $\keyspace$, $\msgspace$,
  $\ctspace$, and the algorithms $\skcgen$, $\skcenc$, $\skcdec$ in a
  manner consistent with our model.

  We now prove completeness.  Observe that for any $m \in \msgspace$
  and $k \in \keyspace$, we have \[ \skcdec_{k}(\skcenc_{k}(m)) = (m
  \oplus k) \oplus k = m \oplus (k \oplus k) = m \oplus 0^{n} = m, \]
  as required.  (Here we have used standard facts about the $\oplus$
  operation, i.e., associativity and that $x \oplus x = 0$ for all
  $x$.)

  Finally, we prove perfect secrecy according to
  Definition~\ref{def:perfect-secrecy}.  ``Plugging in'' the scheme to
  the definition, observe that for any $\bar{m} \in \msgspace$ and
  $\bar{c} \in \ctspace$, we have \[ \Pr_{k \gets
    \skcgen}[\skcenc_{k}(\bar{m}) = \bar{c}] = \Pr_{k \gets
    \bit^{n}}[\bar{m} \oplus k = \bar{c}] = \Pr_{k \gets \bit^{n}}[k =
  \bar{m} \oplus \bar{c}] = 2^{-n}, \] where the last equality holds
  because $\bar{m} \oplus \bar{c} \in \bit^{n}$ is fixed. It follows
  that for any $m_{0}, m_{1} \in \msgspace$ and $\bar{c} \in
  \ctspace$, we have \[ \Pr_{k \gets \skcgen}[\skcenc_{k}(m_{0}) =
  \bar{c}] = 2^{-n} = \Pr_{k \gets \skcgen}[\skcenc_{k}(m_{1}) =
  \bar{c}], \] as required by the definition.  This completes the
  proof.
\end{proof}

\end{document}

%%% Local Variables: 
%%% mode: latex
%%% TeX-master: t
%%% End: 
