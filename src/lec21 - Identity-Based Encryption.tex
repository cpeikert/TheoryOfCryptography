\documentclass[11pt]{article}

\usepackage{fullpage,microtype,pdfsync,hyperref}
\usepackage{newtxtext}
\usepackage[T1]{fontenc}

\usepackage{mathtools,amsfonts,amssymb}
\usepackage[amsmath,amsthm,thmmarks,hyperref]{ntheorem}
\usepackage[capitalize,nameinlink,noabbrev]{cleveref}

\RequirePackage{mathtools}

%%% BLACKBOARD SYMBOLS

% hyperref package defines C and G, undefined to avoid conflicts
\let\C\relax
\let\G\relax
\newcommand{\C}{\ensuremath{\mathbb{C}}}
\newcommand{\D}{\ensuremath{\mathbb{D}}}
\newcommand{\F}{\ensuremath{\mathbb{F}}}
\newcommand{\G}{\ensuremath{\mathbb{G}}}
\newcommand{\J}{\ensuremath{\mathbb{J}}}
\newcommand{\N}{\ensuremath{\mathbb{N}}}
\newcommand{\Q}{\ensuremath{\mathbb{Q}}}
\newcommand{\R}{\ensuremath{\mathbb{R}}}
\newcommand{\T}{\ensuremath{\mathbb{T}}}
\newcommand{\Z}{\ensuremath{\mathbb{Z}}}
\newcommand{\QR}{\ensuremath{\mathbb{QR}}}

\newcommand{\Zt}{\ensuremath{\Z_t}}
\newcommand{\Zp}{\ensuremath{\Z_p}}
\newcommand{\Zq}{\ensuremath{\Z_q}}
\newcommand{\ZN}{\ensuremath{\Z_N}}
\newcommand{\Zps}{\ensuremath{\Z_p^*}}
\newcommand{\ZNs}{\ensuremath{\Z_N^*}}
\newcommand{\JN}{\ensuremath{\J_N}}
\newcommand{\QRN}{\ensuremath{\QR_{N}}}
\newcommand{\QRp}{\ensuremath{\QR_{p}}}

%%% THEOREM COMMANDS

\RequirePackage{amsthm}

\theoremstyle{plain}            % following are "theorem" style
\newtheorem{theorem}{Theorem}[section]
\newtheorem{lemma}[theorem]{Lemma}
\newtheorem{corollary}[theorem]{Corollary}
\newtheorem{proposition}[theorem]{Proposition}
\newtheorem{claim}[theorem]{Claim}
\newtheorem{fact}[theorem]{Fact}

\theoremstyle{definition}       % following are def style
\newtheorem{definition}[theorem]{Definition}
\newtheorem{conjecture}[theorem]{Conjecture}
\newtheorem{example}[theorem]{Example}
\newtheorem{protocol}[theorem]{Protocol}

\theoremstyle{remark}           % following are remark style
\newtheorem{remark}[theorem]{Remark}
\newtheorem{note}[theorem]{Note}

% equation numbering style
\numberwithin{equation}{section}

%%% GENERAL COMPUTING

\newcommand{\bit}{\ensuremath{\set{0,1}}}
\newcommand{\pmone}{\ensuremath{\set{-1,1}}}

% asymptotics
\DeclareMathOperator{\poly}{poly}
\DeclareMathOperator{\polylog}{polylog}
\DeclareMathOperator{\negl}{negl}
\newcommand{\Otil}{\ensuremath{\tilde{O}}}

% probability/distribution stuff
\DeclareMathOperator*{\E}{E}
\DeclareMathOperator*{\Var}{Var}

% sets in calligraphic type
\newcommand{\calD}{\ensuremath{\mathcal{D}}}
\newcommand{\calF}{\ensuremath{\mathcal{F}}}
\newcommand{\calG}{\ensuremath{\mathcal{G}}}
\newcommand{\calH}{\ensuremath{\mathcal{H}}}
\newcommand{\calX}{\ensuremath{\mathcal{X}}}
\newcommand{\calY}{\ensuremath{\mathcal{Y}}}

% types of indistinguishability
\newcommand{\compind}{\ensuremath{\stackrel{c}{\approx}}}
\newcommand{\statind}{\ensuremath{\stackrel{s}{\approx}}}
\newcommand{\perfind}{\ensuremath{\equiv}}

% font for general-purpose algorithms
\newcommand{\algo}[1]{\ensuremath{\mathsf{#1}}}
% font for general-purpose computational problems
\newcommand{\problem}[1]{\ensuremath{\mathsf{#1}}}
% font for complexity classes
\newcommand{\class}[1]{\ensuremath{\mathsf{#1}}}

% complexity classes and languages
\renewcommand{\P}{\class{P}}
\newcommand{\BPP}{\class{BPP}}
\newcommand{\NP}{\class{NP}}
\newcommand{\coNP}{\class{coNP}}
\newcommand{\AM}{\class{AM}}
\newcommand{\coAM}{\class{coAM}}
\newcommand{\IP}{\class{IP}}

%%% "LEFT-RIGHT" PAIRS OF SYMBOLS


% inner product
\DeclarePairedDelimiter\inner{\langle}{\rangle}
% absolute value
\DeclarePairedDelimiter\abs{\lvert}{\rvert}
% a set
\DeclarePairedDelimiter\set{\{}{\}}
% parens
\DeclarePairedDelimiter\parens{(}{)}
% tuple, alias for parens
\DeclarePairedDelimiter\tuple{(}{)}
% square brackets
\DeclarePairedDelimiter\bracks{[}{]}
% rounding off
\DeclarePairedDelimiter\round{\lfloor}{\rceil}
% floor function
\DeclarePairedDelimiter\floor{\lfloor}{\rfloor}
% ceiling function
\DeclarePairedDelimiter\ceil{\lceil}{\rceil}
% length of some vector, element
\DeclarePairedDelimiter\length{\lVert}{\rVert}
% "lifting" of a residue class
\DeclarePairedDelimiter\lift{\llbracket}{\rrbracket}
\DeclarePairedDelimiter\len{\lvert}{\rvert}

%%% CRYPTO-RELATED NOTATION

% KEYS AND RELATED

\newcommand{\key}[1]{\ensuremath{#1}}

\newcommand{\pk}{\key{pk}}
\newcommand{\vk}{\key{vk}}
\newcommand{\sk}{\key{sk}}
\newcommand{\mpk}{\key{mpk}}
\newcommand{\msk}{\key{msk}}
\newcommand{\fk}{\key{fk}}
\newcommand{\id}{id}
\newcommand{\keyspace}{\ensuremath{\mathcal{K}}}
\newcommand{\msgspace}{\ensuremath{\mathcal{M}}}
\newcommand{\ctspace}{\ensuremath{\mathcal{C}}}
\newcommand{\tagspace}{\ensuremath{\mathcal{T}}}
\newcommand{\idspace}{\ensuremath{\mathcal{ID}}}

\newcommand{\concat}{\ensuremath{\|}}

% GAMES

% advantage
\newcommand{\advan}{\ensuremath{\mathbf{Adv}}}

% different attack models
\newcommand{\attack}[1]{\ensuremath{\text{#1}}}

\newcommand{\atk}{\attack{atk}} % dummy attack
\newcommand{\indcpa}{\attack{ind-cpa}}
\newcommand{\indcca}{\attack{ind-cca}}
\newcommand{\anocpa}{\attack{ano-cpa}} % anonymous
\newcommand{\anocca}{\attack{ano-cca}}
\newcommand{\euacma}{\attack{eu-acma}} % forgery: adaptive chosen-message
\newcommand{\euscma}{\attack{eu-scma}} % forgery: static chosen-message
\newcommand{\suacma}{\attack{su-acma}} % strongly unforgeable

% ADVERSARIES
\newcommand{\attacker}[1]{\ensuremath{\mathcal{#1}}}

\newcommand{\Adv}{\attacker{A}}
\newcommand{\AdvA}{\attacker{A}}
\newcommand{\AdvB}{\attacker{B}}
\newcommand{\Dist}{\attacker{D}}
\newcommand{\Sim}{\attacker{S}}
\newcommand{\Ora}{\attacker{O}}
\newcommand{\Inv}{\attacker{I}}
\newcommand{\For}{\attacker{F}}

% CRYPTO SCHEMES

\newcommand{\scheme}[1]{\ensuremath{\text{#1}}}

% pseudorandom stuff
\newcommand{\prg}{\algo{PRG}}
\newcommand{\prf}{\algo{PRF}}
\newcommand{\prp}{\algo{PRP}}

% symmetric-key cryptosystem
\newcommand{\skc}{\scheme{SKC}}
\newcommand{\skcgen}{\algo{Gen}}
\newcommand{\skcenc}{\algo{Enc}}
\newcommand{\skcdec}{\algo{Dec}}

% public-key cryptosystem
\newcommand{\pkc}{\scheme{PKC}}
\newcommand{\pkcgen}{\algo{Gen}}
\newcommand{\pkcenc}{\algo{Enc}} % can also use \kemenc and \kemdec
\newcommand{\pkcdec}{\algo{Dec}}

% digital signatures
\newcommand{\sig}{\scheme{SIG}}
\newcommand{\siggen}{\algo{Gen}}
\newcommand{\sigsign}{\algo{Sign}}
\newcommand{\sigver}{\algo{Ver}}

% message authentication code
\newcommand{\mac}{\scheme{MAC}}
\newcommand{\macgen}{\algo{Gen}}
\newcommand{\mactag}{\algo{Tag}}
\newcommand{\macver}{\algo{Ver}}

% key-encapsulation mechanism
\newcommand{\kem}{\scheme{KEM}}
\newcommand{\kemgen}{\algo{Gen}}
\newcommand{\kemenc}{\algo{Encaps}}
\newcommand{\kemdec}{\algo{Decaps}}

% identity-based encryption
\newcommand{\ibe}{\scheme{IBE}}
\newcommand{\ibesetup}{\algo{Setup}}
\newcommand{\ibeext}{\algo{Ext}}
\newcommand{\ibeenc}{\algo{Enc}}
\newcommand{\ibedec}{\algo{Dec}}

% hierarchical IBE (as key encapsulation)
\newcommand{\hibe}{\scheme{HIBE}}
\newcommand{\hibesetup}{\algo{Setup}}
\newcommand{\hibeext}{\algo{Extract}}
\newcommand{\hibeenc}{\algo{Encaps}}
\newcommand{\hibedec}{\algo{Decaps}}

% binary tree encryption (as key encapsulation)
\newcommand{\bte}{\scheme{BTE}}
\newcommand{\btesetup}{\algo{Setup}}
\newcommand{\bteext}{\algo{Extract}}
\newcommand{\bteenc}{\algo{Encaps}}
\newcommand{\btedec}{\algo{Decaps}}

% trapdoor functions
\newcommand{\tdf}{\scheme{TDF}}
\newcommand{\tdfgen}{\algo{Gen}}
\newcommand{\tdfeval}{\algo{Eval}}
\newcommand{\tdfinv}{\algo{Invert}}
\newcommand{\tdfver}{\algo{Ver}}

%%% PROTOCOLS

\newcommand{\out}{\text{out}}
\newcommand{\view}{\text{view}}

%%% COMMANDS FOR LECTURES/HOMEWORKS

\RequirePackage{fancyhdr}

\newcommand{\lecheader}{%
  \chead{\large \textbf{Lecture \lecturenum\\\lecturetopic}}

  \lhead{\small \textbf{Theory of Cryptography}\\}

  \rhead{\small \textbf{Instructor:
      \href{http://www.eecs.umich.edu/~cpeikert/}{Chris Peikert}\\Scribe:
      \scribename}}

  \setlength{\headheight}{20pt}
  \setlength{\headsep}{16pt}
}


%%% xsim settings

\RequirePackage{xsim}
\RequirePackage{needspace}

\DeclareExerciseEnvironmentTemplate{QRef}
{%
    \par\vspace{\baselineskip}
    \Needspace*{2\baselineskip}
    \noindent
    \hyperref[sol:\ExerciseID]{\textbf{\XSIMmixedcase{\GetExerciseName}~\GetExerciseProperty{counter}.}}
    \label{ques:\ExerciseID}
}{}

\DeclareExerciseEnvironmentTemplate{ARef}{%
        \Needspace*{2\baselineskip}
        \noindent
        \hyperref[ques:\ExerciseID]{\textbf{\XSIMmixedcase{\GetExerciseParameter{exercise-name}}~\GetExerciseProperty{counter}.}}
        \label{sol:\ExerciseID}
        \GetExerciseBody{exercise}
        \newline
        \textbf{\XSIMmixedcase{\GetExerciseName}.}
}{\par\vspace{\baselineskip}}

\DeclareExerciseHeadingTemplate{simple}{%
    \section*{\XSIMmixedcase{\GetExerciseParameter{solution-name}s}}
}

\DeclareExerciseType{question}{%
    exercise-env = question,
    solution-env = answer,
    exercise-name = Question,
    solution-name = Answer,
    exercise-template = QRef,
    solution-template = ARef,
}

\xsimsetup{%
  path = {questions},
  file-extension = {aux},
  print-solutions/headings-template=simple
}

\AtEndDocument{\pagebreak\printsolutions}


% VARIABLES

\newcommand{\lecturenum}{21}
\newcommand{\lecturetopic}{Identity-Based Encryption}
\newcommand{\scribename}{Pushkar Tripathi}

% END OF VARIABLES

\lecheader

\pagestyle{plain}               % default: no special header

\begin{document}

\thispagestyle{fancy} % first page should have special header

% LECTURE MATERIAL STARTS HERE

\section{Identity-Based Encryption}
\label{sec:ident-based-encrypt}

Shamir, 1984: ``What if your name could be your public key?''

\medskip
\noindent Boneh and Franklin, 2001: ``It can!''

\medskip
\noindent Cocks, 2001: ``I did it years ago \ldots (again).''

\bigskip

Encryption in which ``your name is your public key'' is called
Identity-Based Encryption (IBE).  Shamir's original motivation for
identity-based encryption was to simplify the management of public
keys.  Frequently, this process can be unwieldy because of the
following problem: if Bob obtains Alice's public key over the network,
how does Bob know that the key really belongs to Alice, and not to a
malicious adversary-in-the-middle who replaced Alice's key with his
own as it traveled over the network?  The usual solution is to rely
upon ``certificate authorities'' (CAs), who are trusted to implement
the following process: when Alice creates a key pair, she presents her
public key $\pk_{A}$ and some valid form of identification to the CA.
The CA then generates a signature, under its own verification key
$\vk_{CA}$, attesting to the statement ``this public key $\pk_{A}$
belongs to Alice,'' which Alice appends to her own public key.  Later
on, Bob can verify the CA's signature and gain confidence that the key
really is Alice's.  Of course, in order to do this Bob needs to know
the CA's true public key (and to trust that the CA implemented its
policy correctly), so at first sight it seems that we have gained very
little.  However, there need only be a few CAs in the world, and they
can make their keys widely known through alternative means, such as
bundling with common cryptographic software or widespread publication.
Still, this whole process is rather heavyweight: every time a person
generates a new public key, she must register that key with a CA, and
Bob must store and manage a large number of keys for all the people
with whom he corresponds.  (Things become even more complicated when
we introduce \emph{revocation}, where Alice may desire for her key to
be declared invalid if, for example, it becomes compromised.)

Identity-based encryption can simplify the above scenario in the
following way: for a particular administrative domain (e.g.,
\texttt{domain.com}), there is an ``authority'' who generates a
``master'' public/secret key pair for the entire domain.  When Bob
wants to send a message to Alice at \texttt{alice@domain.com}, he
simply encrypts using the public key string \texttt{alice} and the
master public key for the domain (which should be certified as usual).
There is no need for Bob to obtain Alice's individual public key or
certificate, or even for Alice to generate a key at all!  Meanwhile,
Alice identifies herself to the domain authority, who uses its master
secret key to ``extract'' a secret key that works just for the
identity \texttt{alice}.  Alice uses this to decrypt her message as
usual.  Notice here that the authority implicitly knows secret keys
for every user in the system, so it can read everyone's encrypted
messages.  Depending on the scenario, this can be a blessing or a
curse, but in any case authority's master secret key is very powerful,
so it must be guarded carefully.

IBE also provides additional useful properties like implicit
revocation (via identities that have an ``expiration date'' appended
to them) and delegation (only a subset of messages can be decrypted by
a particular agent).  More generally, IBE-related techniques and
implications have proved to be very versatile and powerful in the
design of other important and seemingly unrelated protocols.

The problem of constructing an IBE remained open for many years
following its initial conception by Shamir.  Finally in 2001, Boneh
and Franklin described a scheme that used a relatively new
mathematical object called a bilinear map (which has since been shown
to have numerous other related applications).  Soon after Boneh and
Franklin's announcement, it was revealed that Clifford Cocks, a
mathematician in the United Kingdom's cryptography agency GCHQ, had
years earlier devised a simple IBE based on a standard and elementary
assumption (but his scheme was classified by the UK government).  This
was an interesting case of ``history repeating itself,'' as Cocks
himself had also discovered the famed RSA algorithm in 1974, three
years before Rivest, Shamir and Adelman, but this discovery was also
classified!

In this lecture we will present the Cocks IBE scheme.

\subsection{Model}
\label{sec:model}

The model for IBE consists of an authority that generates a single
master public key $\mpk$ for all its users, along with a master secret
key $\msk$.  To encrypt a message, one needs to know the master public
key $\mpk$ and the receiver's identity $\id$, which can be any string.
A user with identity $\id$ obtains a secret key $\sk_{id}$ from the
authority (typically after proving his identity in some way).  He can
use this secret key to decrypt messages sent to him.

Formally, an $\ibe$ scheme consists of four ppt algorithms:
\begin{itemize}
\item $\ibesetup(1^{n})$ outputs a master public key $\mpk$ and master
  secret key $\msk$.  This algorithm is executed once by the authority
  when setting up the system.
\item $\ibeext(\msk, \id)$ outputs a secret key for every user based
  on his $\id$.  Since only the authority should have access to
  $\msk$, this algorithm is executed only by the authority after
  checking a user's identity.
\item $\ibeenc(\mpk, \id, m)$ outputs a ciphertext corresponding to a
  message $m$.
\item $\ibedec(\sk_{id}, c)$ decrypts a ciphertext and outputs a
  message.
\end{itemize}

\subsection{Security Definition}
\label{sec:security-definition}

Here we present a formal definion of the notion of IND-CPA security
for an $\ibe$ scheme.  The intuition behind this definition is that a
message encrypted to a particular ``target'' identity $\id^*$ (chosen
by the adversary) should remain secret, \emph{even if} the adversary
gets the secret keys for a polynomial number of other identities that
are different from $\id^*$.

\begin{definition}
  \label{def:ibe-ind-cpa}
  An $\ibe$ scheme is IND-CPA-secure if:
  \[\left\langle \mpk, \algo{E}(\cdot),
    C^0(\id^*,m_0,m_1)\right\rangle \compind \left\langle \mpk,
    \algo{E}(\cdot), C^1(\id^*,m_0,m_1)\right\rangle,\] where $(\mpk,
  \msk) \gets \ibesetup$, and $\algo{E}(\id)$ returns
  $\ibeext(\msk,\id)$, except if queried on $\id^*$ in which case it
  returns $\bot$.  Furthermore, for an $\id^{*}$ different from all
  prior queries to $\algo{E}$, the oracle $C^b(\id^*,m_0,m_1)$,
  returns $\ibeenc(\mpk,\id^{*},m_b)$, defines $\id^{*}$ as the target
  identity, and terminates.
\end{definition}

The definition can be strengthened in various ways, such as adding
chosen-ciphertext security.  In such a definition the adversary would
additionally get a decryption oracle for the identity $\id^{*}$ that
it is allowed to query on any string except for the challenge
ciphertext that was output by $C^{b}$.

\section{Number Theory Background}
\label{sec:numb-theory-backgr}

Throughout this lecture we will let $N$ be a \textit{Blum integer},
i.e., $N = pq$ for primes $p$ and $q$ that are both $3$ modulo $4$.
We use $\QRN^*$ to denote the set of quadratic residues modulo $N$.
We use $\left( \frac{a}{p} \right) = a^{p-1/2} = \pm 1 \bmod p$ to
denote the Legendre symbol.  (The symbol itself is defined as an
integer, not a quantity modulo $p$).

A natural generalization of the Legendre symbol is the \emph{Jacobi
  symbol}, defined as $\left( \frac{a}{pq}\right) = \left(
  \frac{a}{p}\right)\cdot \left( \frac{a}{q}\right)$.  Note that
$\left( \frac{-1}{p}\right) = \left( \frac{-1}{q} \right) = -1$, hence
$\left( \frac{-1}{N}\right) = 1$ for a Blum integer $N$.  That is, $-1
\in \JN^{*}$ but is a non-square.  Also note that $\left(
  \frac{ab}{N}\right) = \left( \frac{a}{N}\right) \cdot \left(
  \frac{b}{N}\right)$, by the multiplicative property of the Legendre
symbol.  Finally, we define $\JN^{*} = \left\{ a \in \Z^*_N :
  \left(\frac{a}{N}\right) = 1 \right\}$, which is a multiplicative
subgroup of $\ZN^{*}$.  The following claim can be established easily.

\begin{claim}
  \label{compute_claim}
  Given $a$ and $N$ (in binary), we can efficiently compute $\left(
    \frac{a}{N}\right)$ without knowing the factoring of $N$.
\end{claim}

\begin{proof}
  The following \emph{reciprocity laws} can be proved with some
  effort.  We omit their proofs.
  \begin{eqnarray}
    \label{rec_1} \text{for odd $a$ and $N$}, \quad \left(\frac{a}{N}
    \right) &=& \left(\frac{N}{a} \right) \cdot \left( -1
    \right)^{\frac{a-1}{2}\cdot\frac{N-1}{2}} \\
    \label{rec_2} \text{for any } N, \quad \left(\frac{-1}{N} \right) &=&
    \left( -1 \right)^{\frac{N-1}{2}} \\
    \label{rec_3}\left(\frac{2}{N} \right) &=& \left( -1
    \right)^{\frac{N^2-1}{8}}
  \end{eqnarray}

  For any even number $a$, we can reduce the problem to the case when
  $a$ is odd by using \cref{rec_3}.  Using \cref{rec_1,rec_2}, we can
  iteratively reduce $a$ or $N$ until one of them becomes eauls to
  $1$, when $\left(\frac{a}{N} \right)$ can be calculated trivially.
  The overall algorithm and its complexity is analogous to Euclid's
  algorithm for finding the gcd of two numbers.
\end{proof}

We also present another claim that would be useful in the following
section.

\begin{claim}
  \label{equal_claim}
  $\abs{\QRN^*} = \tfrac12 \abs{\JN^*}$.
\end{claim}

\begin{proof}
  Let $a \in \Z_N^*$.  Note that $a \in \QRN^*$ iff $\left(
    \frac{a}{p} \right) = \left( \frac{a}{q} \right) = 1$, so
  $\abs{\QRN^{*}} = (p-1)(q-1)/4$.  Also, $a \in \JN^*$ iff $\left(
    \frac{a}{p}\right) = \left( \frac{a}{q}\right) = \pm 1$, so
  $\abs{\JN^*} = \frac{(p-1)(q-1)}{2}$.
\end{proof}

The security of Cocks' IBE is based on the following assumption which
was proposed in the seminal work of Goldwasser and Micali on
public-key encryption.

\begin{conjecture}[Quadratic Residuosity Assumption]
  \label{QR_conjecture}
  For a modulus generator $S$,
  \[ \left\langle N \leftarrow S(1^*), a \leftarrow \QRN^*
  \right\rangle \compind \left\langle N \leftarrow S(1^*), a
    \leftarrow \JN^* \setminus \QRN^* \right\rangle\]
\end{conjecture}

In words, the assumption says that given $N$, it is hard to
distinguish between a random quadratic residue and a random quadratic
nonresidue \emph{with Jacobi symbol $1$}.  Note that it is crucial
that the element have Jacobi symbol $1$, otherwise the tuples could be
distinguished trivially by computing the Jacobi symbol $\left(
  \frac{a}{N} \right)$ using \cref{compute_claim}.

\subsection{Interlude: Goldwasser-Micali Encryption Scheme}
\label{sec:goldw-micali-pkc}

As a warm-up to the IBE, we briefly recall the \emph{original}
application of the quadratic residuosity assumption to standard
public-key encryption.  The Goldwasser-Micali scheme for message space
$\bit$ is defined as follows.
\begin{itemize}
\item $\pkcgen(1^n)$: Generate a Blum integer $\pk = N$ as a public
  key with known factorization $\sk = (p,q)$.
\item $\pkcenc(N,b \in \bit)$: Choose $r \gets \Z_N^*$ and output the
  encryption of $b$ as $c = r^2 \cdot (-1)^b \bmod N$.
\item $\pkcdec(\sk = (p,q), c)$: Return $m = 0$ if $c \in \QRN^*$, and
  return $1$ otherwise.  This test can be done efficiently by testing
  whether both $\left( \frac{c}{p} \right) = 1$ and $\left(
    \frac{c}{q} \right) = 1$.
\end{itemize}

The proof of security follows directly from
\cref{QR_conjecture}, which yields the following theorem.

\begin{theorem}
  Under QRA, the Goldwasser-Micali scheme is IND-CPA-secure.
\end{theorem}

\section{The Cocks IBE Scheme}
\label{sec:cocks-scheme}

In this section we define Cocks' $\ibe$ scheme and prove its security.
To begin, we give a basic public key encryption scheme that works
relative to a modulus $N$, but where decryption can be performed
\emph{without needing the factorization of $N$}.  Instead, decryption
will only require knowledge of a square root of a user's public key.
Looking ahead, the factorization of $N$ will be the \emph{master}
secret key of the system, which will allow the authority to extract a
secret key for any user.  Later in \cref{sec:making-id-based},
we will convert this PKE into an $\ibe$ scheme.

\subsection{Warm-Up: Public-Key Encryption Scheme}
\label{sec:warm-up:-public}

The PKE is for the message space $\set{\pm 1}$, and is defined by the
following algorithms, which are all implicitly provided with the
modulus $N$.
\begin{itemize}
\item $\pkcgen(1^n)$: Generate secret key $\sk = r \gets \ZN^{*}$ and
  public key $\pk = r^2 = a \in \QRN^{*} \subseteq \ZN^{*}$.
\item $\pkcenc(a, m \in \set{\pm 1})$: Choose uniformly random $t \in
  \Z_N^*$ such that $\left( \frac{t}{N}\right) = m$.  (This can be
  done by repetition, since the probability of success for a random
  choice of $t$ is $1/2$.)  Output the ciphertext $c = t + a/t \bmod
  N$.
\item $\pkcdec(r,c)$: Output $\left( \frac{c+2r}{N} \right)$.
\end{itemize}

From inspection it is not immediately apparent that decryption is
correct, but the following claim establishes that this is so.

\begin{claim}
  \label{dec_correct}
  For any $r \in \sqrt{a} \bmod N$ and any message $m \in \set{\pm
    1}$, we have $\pkcdec(r, \pkcenc(a, m)) = m$.
\end{claim}

\begin{proof}
  Let $t$ be the random value chosen by $\pkcenc$ where $\left(
    \frac{t}{N} \right) = m$.  The claim follows from the equations
  given below.
  \begin{align*}
    \left( \frac{c+2r}{N} \right) &= \left( \frac{t + r^2/t + 2r}{N}
    \right) \cdot \left( \frac{t}{N} \right)^2  \\
    &= \left( \frac{t^2 + r^2 + 2rt}{N} \right) \cdot \left(
      \frac{t}{N} \right)  \\
    &= \left( \frac{(t+r)^2}{N} \right) \cdot \left( \frac{t}{N}
    \right) \\
    &= \left( \frac{t}{N} \right).
  \end{align*}
\end{proof}

In the remainder of this section we will argue that the above scheme
is IND-CPA secure under the QRA by showing that if the public key $\pk
= a$ is instead an element of $\JN^{*} \setminus \QRN^{*}$, then the
encryption algorithm $\pkcenc$ is ``lossy,'' i.e., the distributions
of $\pkcenc(a, +1)$ and $\pkcenc(a, -1)$ are \emph{identical}.  The
following claim establishes this lossy property.  Following the claim,
we easily prove that this lossiness property implies IND-CPA security.

\begin{claim}
  \label{claim:lossy}
  If $\pk = a \in \JN^{*} \setminus \QRN^{*}$, then $\pkcenc(a, +1)
  \equiv \pkcenc(a, -1)$.
\end{claim}

\begin{proof}
  Note that $\left( \frac{a}{p} \right) = \left( \frac{a}{q} \right) =
  -1$.  Let $c = t+a/t$ (for some $t \in \ZN^{*}$) be some fixed
  ciphertext that could be output by $\pkcenc(a, \cdot)$.  For this
  fixed $c$ and public key $a$, consider all solutions $t$ to this
  equation $c = t+a/t$.  Let $t = t_0 \in \Z_N^*$ be an arbitrary such
  solution.  Written using the Chinese remainder representation of
  $\ZN^{*}$, there are a total of four solutions:
  \begin{itemize}
  \item $\left\langle t_0 \bmod p ; t_0 \bmod q \right\rangle$ has
    Jacobi symbol $\left( \frac{t_0}{N} \right)$.
  \item $\left\langle a/t_0 \bmod p ; t_0 \bmod q \right\rangle$ has
    Jacobi symbol $-1 \cdot \left( \frac{t_0}{N} \right)$, because
    $\left( \frac{a/t_{0}}{p} \right) = \left( \frac{a}{p} \right)
    \left( \frac{t_{0}}{p} \right)^{-1} = -1 \left( \frac{t_{0}}{p}
    \right)$.
  \item $\left\langle t_0 \bmod p ; a/t_0 \bmod q \right\rangle$ has
    Jacobi symbol $-1 \cdot \left( \frac{t_0}{N} \right)$, for similar
    reasons.
  \item $\left\langle a/t_0 \bmod p ; a/t_0 \bmod q \right\rangle$ has
    Jacobi symbol $\left( \frac{t_0}{N} \right)$.
  \end{itemize}
  Hence, $\Pr \left[ \pkcenc(a, 1) = c \right] = \Pr \left[ \pkcenc(a,
    -1) = c \right]$, and the proof is complete.
\end{proof}

\begin{corollary}
  \label{cor:ind-cpa-cocks-basic}
  Under the QRA, the basic Cocks public-key encryption scheme is
  IND-CPA secure.
\end{corollary}

\begin{proof}
  By the QRA, composition lemma, and \cref{claim:lossy}, we have
  the following:
  \begin{align*}
    \langle N, a \gets \QRN^{*}, \pkcenc(a, +1) \rangle & \compind
    \langle N, a \gets \JN^{*} \setminus \QRN^{*}, \pkcenc(a, +1)
    \rangle \\
    &\equiv \langle N, a \gets \JN^{*} \setminus \QRN^{*},
    \pkcenc(a, -1) \rangle \\
    &\compind \langle N, a \gets \QRN^{*}, \pkcenc(a, -1) \rangle.
  \end{align*}
\end{proof}

\subsection{Making It Identity-Based}
\label{sec:making-id-based}

% hashing $\id$ to two public keys: one in $\QRN^{*}$, on in $\JN^{*}
% \backslash \QRN^{*}$.  extracting secret keys using $\msk$, the
% factorization of $N$.

We make the scheme identity-based in the random oracle model.  Let $H
\colon \bit^{*} \to \JN^{*}$ be modelled as a truly random function,
which will maps identities to public keys in the basic scheme.  (Note
that such a function can be obtained from a random function $H'
\bit^{*} \to \bit$ by using the random output bits of $H'$ to sample
from $\JN^{*}$.)  There is only one subtlety: we cannot detect whether
the output of the hash function $H$ is a ``proper'' or ``lossy''
public key.  The solution is to encrypt (using independent randomness)
to both $a_{0} = H(\id)$ and $a_{1} = -H(\id)$, both of which have
Jacobi symbol $1$, and exactly one of which is a quadratic residue
(because $-1 \in \JN^{*} \setminus \QRN^{*}$).

\begin{itemize}
\item $\ibesetup(1^{n})$ chooses a Blum integer $\mpk = N = pq$ as the
  master public key, and lets $\msk = (p,q)$ be the master secret key.
\item $\ibeext^{H}(\msk, \id)$ lets $a_0 = H(\id)$ and $a_1 = -1 \cdot
  H(\id)$.  It returns $r_{\id} \in \ZN^{*}$ as a \emph{random} square
  root of $a_0$ or $a_1$, whichever is a quadratic residue.
\item $\ibeenc^{H}(\mpk = N, \id, m \in \set{\pm 1})$ lets $a_0 =
  H(\id)$ and $a_1 = -1 \cdot H(\id)$, and outputs $c_0 =
  \pkcenc(a_0,m)$ and $c_1 = \pkcenc(a_1,m)$ where $\pkcenc(\cdot,
  \cdot)$ is the encryption algorithm of the basic scheme as defined
  in \cref{sec:warm-up:-public}
\item $\ibedec(\sk_{id} = \sqrt{a_b}, c = (c_0,c_1))$ outputs
  $\pkcdec(\sk_{id}, c_b)$, where $\pkcdec(\cdot,\cdot)$ is the
  decryption algorithm of the basic scheme as defined in
  \cref{sec:warm-up:-public}.
\end{itemize}

We now provide an outline of the proof of security for the above
scheme; a good exercise is to fill in the details carefully.  In order
to prove security, we need to give a simulator $\Sim(N,a)$ that
attempts to solve the QR problem using an adversary for the IBE
scheme.  The simulator must provide an $\msk$, and answer oracle
queries $H(\cdot)$ and extraction queries $\ibeext_{\msk}(\cdot)$ for
arbitrary identities (but without knowing the $\msk$!).  The basic
idea is that $\Sim$ sets $\msk = N$, and ``programs'' the random
oracle $H$ so that it knows a random square root for $\pm H(\id)$ for
all but a single $H(\id^*) := \pm a$, where $\id^{*}$ is a randomly
chosen query to $H$ that $\Sim$ ``guesses'' will be the adversary's
declared target identity.  If $\Sim$ does happen to guess this
identity correctly, then it prepares a challenge ciphertext by using
the basic scheme to encrypt two \emph{opposite} bits in random order
(i.e., either $+1, -1$ or $-1, +1$) under $a$ and $-a$, respectively.
Note that by the lossiness property, such a ciphertext is distributed
identically as either $\ibeenc(\id^{*}, +1)$ or $\ibeenc(\id^{*},
-1)$, depending on whether $a$ or $-a$ is a quadratic residue.  Based
on whether the adversary identifies the ciphertext as an encryption of
$+1$ or $-1$, then, the simulator answers whether $a \in \QRN^{*}$ or
not.

\end{document}

%%% Local Variables: 
%%% mode: latex
%%% TeX-master: t
%%% End: 
